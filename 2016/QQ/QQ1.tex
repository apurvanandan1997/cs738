\documentclass[11pt,draft]{article}
\usepackage{anysize}
\usepackage{amssymb}
\usepackage{amsmath}
\usepackage{paralist}
\usepackage{pstricks}
\usepackage{pst-text}
\usepackage{pst-node}
\usepackage{pst-rel-points}
\usepackage{multirow}
\usepackage{xspace}
\usepackage{wrapfig}
\usepackage{fancyvrb}

%% Page layout
\pagestyle{headings}
\markboth{}{Your Name and Roll Number:}


\begin{document}
\noindent \rule{\textwidth}{.2mm}
\begin{center}
{\large {\bf Quiz-1 CS618} }

Duration: 45 Minutes \hfill Max Marks: 45
\end{center} 
\noindent \rule{\textwidth}{.2mm}
\begin{itemize}
\item Write your name and roll number on the question paper and the
  answer book.
\item No explanations will be provided. In case of a doubt, make
  suitable assumptions and justify.
\end{itemize}
\noindent \rule{\textwidth}{.2mm}

\noindent\begin{tabular}{|l|@{}l}%\cline{1-1}
\begin{minipage}[t]{.33\textwidth}
\begin{verbatim}
/* ------ entry ------- */

    x = ... /*something*/
    y = x + 3

L2:
    if (y > 20) goto L1
    x = x + 2
    y = x + 3
    goto L2

L1:
    v = x + 3
    w = x + 2

L3:  
    u = v + w
    goto L5

L4:
    w = v + w
    goto L7

L5:
    z = w
    u = x * y
    if (u < z) goto L4
    goto L6
    
L6:
    goto L7

L7:

/* ----- exit --------- */
    
\end{verbatim}
\end{minipage}&
\begin{minipage}[t]{.63\textwidth}
  \begin{enumerate}
  \item (10) Convert the given 3-address program to a CFG. Note that:
    \begin{enumerate}
    \item The basic blocks should be maximal basic blocks.
    \item Clearly show the statements inside each basic block.
    \item For edges corresponding to the conditional {\tt goto}-s,
      label the true edge as {\bf T} and false edge as {\bf F}.
    \end{enumerate}
    
  \item (15) Perform {\bf Very Busy Expression} analysis for the given
    program. In particular,
    \begin{enumerate}
    \item Compute {\bf gen} and {\bf kill} for each basic block.
    \item Give initial values of  {\bf IN} and {\bf OUT} for each basic block.
    \item Give final {\bf IN} and {\bf OUT} for each basic block. You do
      not need to show the intermediate passes.
    \end{enumerate}

  \item (20) An expression {\bf e} is {\bf Partially Available} at a
    program point $\mathbf{\pi}$ if there is {\em some} path from {\em
      entry} to $\mathbf{\pi}$ that contains a computation of {\bf e}
    which is not followed by an assignment to any of its operand.
    
    Perform {\bf Partially Available Expression} analysis for the
    given program. In particular,
    \begin{enumerate}
    \item Compute {\bf gen} and {\bf kill} for each basic block.
    \item Give {\bf equations} for computing {\bf IN} and {\bf OUT}
      for a basic block.
    \item Give initial values of {\bf IN} and {\bf OUT} for each basic
      block.
    \item Give final {\bf IN} and {\bf OUT} for each basic block. You
      do not need to show the intermediate passes.
    \end{enumerate}    
  \end{enumerate}
\end{minipage}\\ \cline{1-1}
\end{tabular}
\end{document}
