\documentclass[12pt]{article}
\usepackage{graphicx}
\usepackage{graphics}
\usepackage{anysize}
\usepackage{fancyhdr}
\usepackage{pstricks}
\usepackage{pst-node}
\usepackage{pst-rel-points}

\setlength{\headheight}{15pt}
\pagestyle{fancyplain}
\lhead{\fancyplain{}{{\em Page \#\/}\thepage}}
\chead{CS738}
\rhead{\fancyplain{}{\textit{Roll No:}\rule{30mm}{0pt}}}
\lfoot{}
\cfoot{}
\rfoot{}

\newcommand{\tbd}[1]{{\red #1}}
\newcommand{\answer}[1]{{{\blue #1}}}
%%\renewcommand{\answer}[1]{\mbox{}}
\newcommand{\symbox}[2]{S#1\psframebox{#2}\phantom{S#1}}
\newcommand{\rsymbox}[2]{\phantom{S#1}\psframebox{#2} S#1}


%% \textwidth 7in
%% \addtolength{\oddsidemargin}{-1in}
%% \textheight 9.5in
%% \addtolength{\topmargin}{-1in}
%% \setlength{\parindent}{0pt}
%% \setlength{\parskip}{0.5cm}
%% \topskip 0.0in
\thispagestyle{empty}

\begin{document}

{\Large\bf NAME: \hspace*{3in} ROLL NO:}

\begin{center}
CS738: Advanced Compiler Optimizations \\
End Semester Examination, 2017-18 I\\
Max Time: 3 Hours  \hfill Max Marks: \tbd{160} \\

{\bf NOTE: }
\begin{itemize}
\item There are total {\bf 6} questions on {\bf 5 pages}.
\item Write your name and roll number on the question paper and the
  answer book.
\item Presenting your answers properly is your
  responsibility. You lose credit if you can not present your
  ideas clearly, and in proper form. Please DO NOT come back
  for re-evaluation saying, ``What I actually meant was
  \ldots''.
\item Be precise and write clearly. Remember that somebody
  has to read it to evaluate!
\item {\bf IMPORTANT: Some questions have to be answered on the
  sheet provided at the end of the question paper. Write you Name
  and Roll Number, and return it along with the answer book.}

\end{itemize}
\hrule
\end{center}


\newcommand{\pt}{\ensuremath{\pi}}
\newcommand{\fgentry}{{\sf Entry}}
\newcommand{\fgexit}{{\sf Exit}}
\begin{center}
  \Large\bf Notations
\end{center}
\newcommand{\indef}[2]{\ensuremath{{\sf BeforeIN}(#1, #2)}}
\newcommand{\outdef}[2]{\ensuremath{{\sf BeforeOut}(#1, #2)}}
\newcommand{\inuse}[2]{\ensuremath{{\sf AfterIN}(#1, #2)}}
\newcommand{\outuse}[2]{\ensuremath{{\sf AfterOut}(#1, #2)}}
\newcommand{\inacc}[2]{\ensuremath{{\sf AccIN}(#1, #2)}}
\newcommand{\outacc}[2]{\ensuremath{{\sf AccOUT}(#1, #2)}}
\newcommand{\inaccset}[2]{\ensuremath{{\sf AccSetIN}(#1, #2)}}
\newcommand{\outaccset}[2]{\ensuremath{{\sf AccSetOUT}(#1, #2)}}
\newcommand{\IN}[1]{\ensuremath{{\sf IN}(#1)}}
\newcommand{\OUT}[1]{\ensuremath{{\sf OUT}(#1)}}
\newcommand{\Pred}[1]{\ensuremath{{\sf PRED}(#1)}}
\newcommand{\Succ}[1]{\ensuremath{{\sf SUCC}(#1)}}
\newcommand{\cpath}{\ensuremath{\stackrel{+}{\longrightarrow}}}
\newcommand{\idfp}{\ensuremath{\mbox{DF}^+}}

\begin{itemize}
\item CFG stands for control flow graph.
\item \IN{S} denotes the program point before the statement
  $S$. \OUT{S} denotes the program point after
  the statement of $S$.
\item \Pred{S} denotes the set of predecessors, and \Succ{S}
  denotes the set of successors of $S$. %in the CFG.
\item In a CFG, $x \cpath y$ denotes a path from
  node $x$ to node $y$, having one or more edges. Both $x$ and
  $y$ are considered to be a part of the path.
\item \idfp($\varphi$) denotes the Iterated Dominance Frontier of
  the set of CFG nodes $\varphi$.
\end{itemize}
\clearpage

\begin{enumerate}
%%%%%%%%%%%%%%%%%%%%%%%%%%%%%%%%%%%%%%%%%%%%%%%%%%%%%%%%%%%%%%%%%%%%%
\item\label{ques1} $[15+20]$ {\bf Access Range of A Variable.} A
  program point $\pt$ is in the {\em access range} of a variable
  $v$ if {\bf both} the following conditions hold:
\begin{itemize}
\item There is an access (definition or use) of $v$ on some path
  from \fgentry\/ to \pt\ \\[5pt] {\bf AND}
\item There is an access of $v$ on some path from \pt\ to
  \fgexit.
\end{itemize} 

Figure~\ref{fig:accessrange} shows the CFG for an example program having variables A, B and C.
\begin{figure}[h!]
\centering
%\includegraphics*[scale=0.59]{AccessRange.eps}
\begin{center}
\psset{xunit=1mm,yunit=.6mm}
\begin{pspicture}(0,10)(150,-130)
  %\psframe(0,10)(150,-160)
  \putnode{b0}{origin}{75}{5}{\symbox{0}{\fgentry}}
  \putnode{b1}{b0}{0}{-20}{\symbox{1}{$A = \ldots$}}
  \putnode{b2}{b1}{0}{-20}{\rsymbox{2}{$A = \ldots$}}
  \putnode{b3}{b2}{0}{-20}{\rsymbox{3}{$B = \ldots$}}
  \putnode{b4}{b3}{-20}{-20}{\rsymbox{4}{$\ldots = B$}}
  \putnode{b5}{b3}{20}{-20}{\symbox{5}{$\ldots = A$ }}
  \putnode{b6}{b5}{0}{-20}{\symbox{6}{$C = \ldots$}}
  \putnode{b7}{b6}{0}{-20}{\symbox{7}{$\ldots = C$}}
  \putnode{b8}{b7}{0}{-20}{\symbox{8}{\fgexit}}

  \ncline{->}{b0}{b1}
  \ncline{->}{b1}{b2}
  \ncline{->}{b2}{b3}
  \ncline{->}{b3}{b4}
  \ncline{->}{b3}{b5}
  \ncline{->}{b5}{b6}
  \ncline{->}{b6}{b7}
  \ncline{->}{b7}{b8}
  \nccurve[angleA=215,angleB=155,nodesep=0,ncurv=1.2]{->}{b4}{b2}
\end{pspicture}
\end{center}

\caption{An Example Program\label{fig:accessrange}}
\end{figure}

\begin{enumerate}
\item Complete the Table~\ref{table:acc-range} (on the last page) to
show the program points in the access ranges of variables $A$,
$B$ and $C$ for the CFG in Figure~\ref{fig:accessrange}.

\item {\bf Define data flow equations to compute \inacc{S}{v} and \outacc{S}{v}}, where
\begin{itemize}
\item \inacc{S}{v} is true if \IN{S} is in the access range of a variable $v$.  
\item \outacc{S}{v} is true if \OUT{S}
  is in the access range of a variable $v$.
\end{itemize}

You might find it easy to define the equations for the following
auxiliary data flow quantities and then define \inacc{S}{v} and
\outacc{S}{v} in terms of these auxiliary quantities (this is not
compulsory though!):
\begin{itemize}
\item \indef{v}{S} is true if there is an access to variable $v$
  before \IN{S}.  \outdef{v}{S} is true if there is an access to
  the variable $v$ before \OUT{S}.
\item \inuse{v}{S} is true if there is an access to variable $v$
  after \IN{S}. \outuse{v}{S} is true if there is a access to
  variable $v$ after \OUT{S}.
\end{itemize}
\answer{

\[\begin{array}{r@{\ }c@{\ }l}
  \outdef{v}{S} &=& \left\{\renewcommand{\arraystretch}{1.1}
  \begin{array}{@{}l}
    \mbox{true, if $S$ has an access (def or use) of $v$}\\
    \indef{v}{S},\  \mbox{otherwise}
  \end{array}\right. \\ \\ \\
  \indef{v}{S} &=& \left\{\renewcommand{\arraystretch}{1.1}
  \begin{array}{@{}l}
    \mbox{false, if $S$ is \fgentry\ block}\\
    \bigvee\limits_{P \in \Pred{S}}%  \hspace*{-5mm}
    \outdef{v}{P},\ \mbox{otherwise}
  \end{array}\right. \\ \\ \\
  \inuse{v}{S} &=& \left\{\renewcommand{\arraystretch}{1.1}
  \begin{array}{@{}l}
    \mbox{true, if $S$ has an access (def or use) of $v$}\\
    \outuse{v}{S},\  \mbox{otherwise}
  \end{array}\right. \\ \\ \\
  \outuse{v}{S} &=& \left\{\renewcommand{\arraystretch}{1.1}
  \begin{array}{@{}l}
    \mbox{false, if $S$ is \fgexit\ block}\\
    \bigvee\limits_{BS \in \Succ{S}} %\hspace*{-5mm}
    \inuse{v}{BS},\ \mbox{otherwise}
  \end{array}\right.
\end{array}\]

\[
\begin{array}{@{}r@{}c@{}l}
\inacc{v}{S} &=& \indef{v}{S} \bigwedge \inuse{v}{S} \\ \\ \\
\outacc{v}{S} &=& \outdef{v}{S} \bigwedge \outuse{v}{S}
\end{array}\]

}
\end{enumerate}
\answer{\clearpage}
%%%%%%%%%%%%%%%%%%%%%%%%%%%%%%%%%%%%%%%%%%%%%%%%%%%%%%%%%%%%%%%%%%%%%
\newcommand{\vset}{\ensuremath{\varphi}}
\item $[15+10]$  {\bf Access Range of a Set of Variables:} A program point \pt\ is in the access range of a set of variables, \vset, if both the following conditions hold:
  \begin{itemize}
  \item  There is an access to a variable $v \in \vset$ on a path from \fgentry\ to \pt \\[5pt] {\bf AND}
  \item There is an access to a variable $v' \in \vset$ on a path from \pt\ to \fgexit. 
  \end{itemize}
Note that $v$ and $v'$ may or may not be the same variable in \vset.
\begin{enumerate}
\item Complete the Table~\ref{table:acc-range-set} (on the last
  page) to show the program points in the access ranges of given
  sets of variables for the CFG in Figure~\ref{fig:accessrange}.
\item {\bf Define equations to compute \inaccset{S}{\vset} and \outaccset{S}{\vset}}, where
\begin{itemize}
\item \inaccset{S}{\vset} is true if \IN{S} is in the access range of the set of variable $\vset$.  
\item \outaccset{S}{\vset} is true if \OUT{S}
  is in the access range of the set of variables $\vset$.
\end{itemize}
\end{enumerate}
{\bf You can reuse any quantities defined in Question~\ref{ques1}.}
\answer{
\[
\begin{array}{@{}r@{}c@{}l}
\inaccset{\varphi}{S} &=& (\bigvee\limits_{v \in \varphi} \indef{v}{S}) \bigwedge (\bigvee\limits_{v \in \varphi} \inuse{v}{S}) \\
\outaccset{\varphi}{S} &=& (\bigvee\limits_{v \in \varphi} \outdef{v}{S}) \bigwedge (\bigvee\limits_{v \in \varphi} \outuse{v}{S}) 
\end{array}\]
{\color{red}
Note that these {\em simple} definitions {\bf do not work}:
\[\begin{array}{@{}r@{}c@{}l}
\rnode{x0}{}\qquad\inaccset{\varphi}{S}  &=& \bigvee\limits_{v \in \varphi}\inacc{v}{S}\qquad \rnode{x1}{}\\
\rnode{x2}{}\quad\outaccset{\varphi}{S} &=& \bigvee\limits_{v \in \varphi}\outacc{v}{S}\quad\rnode{x3}{}\\
\end{array}\]

\ncline[linecolor=lightgray]{-}{x0}{x3}
\ncline[linecolor=lightgray]{-}{x1}{x2}
For example, \OUT{S5} is in the access range for the set $\{C, A\}$. but it is neither in the access range of $A$, nor that of $C$
}}

\answer{\clearpage}
%%%%%%%%%%%%%%%%%%%%%%%%%%%%%%%%%%%%%%%%%%%%%%%%%%%%%%%%%%%%%%%%%%%%%
\item $[10]$ Prove the following statement:

  For any non-null path $p\!\!\!:\!\! X\!\!\cpath Z$ in a CFG, there exists a node \linebreak
  $X' \in \{X\} \cup \idfp(\{X\})$ on $p$ that dominates
  $Z$. Moreover, unless $X$ dominates every node on p, the node
  $X'$ can be chosen in $\idfp(\{X\})$.

\answer{
  \begin{enumerate}
  \item {\bf $X$ dominates every node in $p$}. Clearly $X$
    dominates $Z$.
  \item {\bf $X$ does not dominate every node in $p$}. Suppose the sequence of nodes in the path $p$ is $n_0 (=X), n_1, n_2, \ldots, n_k (=Z)$. Since $X$ does not dominate all nodes in $p$, some of the nodes in $p$ will be in $\idfp(\{X\})$ {\bf (WHY?)}. Let $n_j$ be the node in $\idfp(\{X\})$ such that it has the highest value of $j$. We claim that $X' = n_j$, i.e., $n_j$ dominates $Z$.
  Suppose $n_j$ does not dominate $Z = (n_k)$. Then, $\exists i, j < i \leq k$ such that $n_j$ does not dominate $n_i$. Choose smallest such $i$. We have, parent of $n_i$ dominated by $n_j$, but $n_i$ is not (strictly) dominated by $n_j$. This gives us:
  \begin{eqnarray*}
    &&n_i \in \mbox{DF}(\{n_j\})\\
       &\Rightarrow& n_i \in \idfp(\{n_j\})\\
       &\Rightarrow& n_i \in \idfp(\{X\})
  \end{eqnarray*}
But this contradicts the fact that $j$ is the largest index such that $n_j \in \idfp(\{X\}).$
  \end{enumerate}
}
\answer{\clearpage}
%%%%%%%%%%%%%%%%%%%%%%%%%%%%%%%%%%%%%%%%%%%%%%%%%%%%%%%%%%%%%%%%%%%%%
\item $[10 + 25]$ Perform {\bf Flow Sensitive May Points-to}
  analysis for the program below. 
  \begin{enumerate}
  \item Give Local Data Flow information for each block
    in terms of May sets. [Do not write the
      generic equation, but instantiate it for the current
      statement.] 
  \item Show result of each iteration of the analysis
    (Maximum $\mathbf{3}$ iterations only).
  \end{enumerate}
  Assume conservative {\em Must} information for Kill.
\begin{center}
\scalebox{0.95}{
  \psset{unit=.8mm}
  \begin{pspicture}(0,10)(65,-70)
    %\psframe(0,10)(65,-70)
    \putnode{b1}{origin}{25}{10}{1\psframebox{b = \&d}}
    \putnode{b2}{b1}{0}{-15}{2\psframebox{c = b}}
    \putnode{b3}{b2}{-15}{-15}{3\psframebox{a = \&b}}
    \putnode{b4}{b3}{0}{-15}{4\psframebox{*a = a}}
    \putnode{b5}{b2}{15}{-22}{5\psframebox{a = \&c}}
    \putnode{b6}{b4}{15}{-15}{6\psframebox{a = *a}}
    \putnode{b7}{b6}{0}{-15}{7\psframebox{*b = c}}
    \ncline{->}{b1}{b2}
    \ncline{->}{b2}{b3}
    \ncline{->}{b2}{b5}
    \ncline{->}{b3}{b4}
    \ncline{->}{b4}{b6}
    \ncline{->}{b5}{b6}
    \ncloop[angleA=-90,angleB=90,loopsize=-30,arm=5,linearc=.2,offset=1]
           {->}{b6}{b5}
    \ncline{->}{b6}{b7}
  \end{pspicture}
}
\end{center}
%
\answer{
\begin{center}
\vspace*{-10mm}\renewcommand{\arraystretch}{1.15}
\hspace*{-25mm}\begin{tabular}{|c||l|l||} \hline
  {\bf \#} & {\bf $May_{GEN}$} & {\bf $May_{KILL}$} \\ \hline \hline
  & & \\
  & $b\rightarrow d$ & $b\rightarrow any$ \\
  1 & \rule{38mm}{0pt}& \rule{38mm}{0pt} \\ \hline
  & & \\
  & $c\rightarrow p$ & $c\rightarrow any$ \\
  2 & $| b\rightarrow p \in MayIN$ &  \\ \hline
  & & \\
  & $a\rightarrow b$ & $a\rightarrow any$ \\
  3 & & \\ \hline
  & $p \rightarrow p'$ & \\
  & $| a\rightarrow p \in MayIN$& \\
  4 & $| a\rightarrow p' \in MayIN$& \\ \hline
  & & \\
  & $a\rightarrow c$ & $a\rightarrow any$ \\
  5 & & \\ \hline
  & $a\rightarrow p$ & $a\rightarrow any$ \\
  & $| a\rightarrow p' \in MayIN$ & \\ 
  6 & $| p'\rightarrow p \in MayIN$ &  \\ \hline
  & $p \rightarrow p'$ & \\
  & $| b\rightarrow p \in MayIN$& \\ 
  7 & $| c\rightarrow p' \in MayIN$& \\ \hline
\multicolumn{3}{c}{    Local Data Flow Information}
\end{tabular}
\end{center}
%\clearpage
\begin{center}
  \renewcommand{\arraystretch}{1.3}
  \hspace*{-25mm}\begin{tabular}{|c||c|c||} \hline
    {\bf \#} & {\bf $May_{IN}$} & {\bf $May_{OUT}$} \\ \hline \hline
    1 & $\phi$ & $b\rightarrow d$\\ \hline
    2 & $b\rightarrow d$ & $b\rightarrow d, c \rightarrow d$ \\ \hline
    3 & $b\rightarrow d, c \rightarrow d$ & $a   \rightarrow b, b\rightarrow d,    c \rightarrow d$ \\ \hline
    4 & $a \rightarrow b, b\rightarrow d,    c \rightarrow d$ & $a \rightarrow b, b\rightarrow bd,    c \rightarrow d$ \\ \hline
    5 & $ b\rightarrow d,    c \rightarrow d$ & $a \rightarrow c, b\rightarrow d,    c \rightarrow d$ \\ \hline
    6 & $a \rightarrow b c, b\rightarrow b d,    c \rightarrow d$ & $a \rightarrow b d, b\rightarrow b d,    c \rightarrow d$ \\ \hline
    7 & $a \rightarrow b d, b\rightarrow b d,    c \rightarrow d$ & $a \rightarrow b d, b\rightarrow b d,    c \rightarrow d, d \rightarrow d$ \\ \hline
    \multicolumn{3}{c}{  Result of Analysis for Iteration No. 1 }
  \end{tabular}
\end{center}

\begin{center}
  \renewcommand{\arraystretch}{1.3}
  \hspace*{-25mm}\begin{tabular}{|c||c|c||c|c|} \hline
    {\bf \#} & {\bf $May_{IN}$} & {\bf $May_{OUT}$} \\ \hline \hline
    1 & $\phi$ & $b\rightarrow d$\\ \hline
    2 & $b\rightarrow d$ & $b\rightarrow d, c \rightarrow d$ \\ \hline
    3 & $b\rightarrow d, c \rightarrow d$ & $a
   \rightarrow b, b\rightarrow d,
    c \rightarrow d$ \\ \hline
    4 & $a \rightarrow b, b\rightarrow d,
    c \rightarrow d$ & $a \rightarrow b, b\rightarrow bd,
    c \rightarrow d$ \\ \hline
    5 & $\mathbf{a \rightarrow b d, b\rightarrow b d,
    c \rightarrow d}$ & $\mathbf{a \rightarrow c, b\rightarrow b d,
    c \rightarrow d}$ \\ \hline 
    6 & $a \rightarrow b c, b\rightarrow b d,
    c \rightarrow d$ & $a \rightarrow b d, b\rightarrow b d,
    c \rightarrow d$ \\ \hline
    7 & $a \rightarrow b d, b\rightarrow b d,
    c \rightarrow d$ & $a \rightarrow b d, b\rightarrow b d,
    c \rightarrow d, d \rightarrow d$ \\ \hline
    \multicolumn{3}{c}{  Result of Analysis for Iteration
      No. 2, Changes shown in bold}
  \end{tabular}
\end{center}

\begin{center}
  \renewcommand{\arraystretch}{1.3}
  \hspace*{-25mm}\begin{tabular}{|c||c|c||c|c|} \hline
    {\bf \#} & {\bf $May_{IN}$} & {\bf $May_{OUT}$} \\ \hline \hline
    1 & $\phi$ & $b\rightarrow d$\\ \hline
    2 & $b\rightarrow d$ & $b\rightarrow d, c \rightarrow d$ \\ \hline
    3 & $b\rightarrow d, c \rightarrow d$ & $a
   \rightarrow b, b\rightarrow d,
    c \rightarrow d$ \\ \hline
    4 & $a \rightarrow b, b\rightarrow d,
    c \rightarrow d$ & $a \rightarrow b, b\rightarrow bd,
    c \rightarrow d$ \\ \hline
    5 & ${a \rightarrow b d, b\rightarrow b d,
    c \rightarrow d}$ & ${a \rightarrow c, b\rightarrow b d,
    c \rightarrow d}$ \\ \hline 
    6 & $a \rightarrow b c, b\rightarrow b d,
    c \rightarrow d$ & $a \rightarrow b d, b\rightarrow b d,
    c \rightarrow d$ \\ \hline
    7 & $a \rightarrow b d, b\rightarrow b d,
    c \rightarrow d$ & $a \rightarrow b d, b\rightarrow b d,
    c \rightarrow d,d \rightarrow d$ \\ \hline
    \multicolumn{3}{c}{  Result of Analysis for Iteration
      No. 3 = No. 2}
  \end{tabular}
\end{center}
}
\answer{\clearpage}
%%%%%%%%%%%%%%%%%%%%%%%%%%%%%%%%%%%%%%%%%%%%%%%%%%%%%%%%%%%%%%%%%%%%%
\item $[10+20]$ Consider a simple programming language with only two
  primitive types: {\bf int} and {\bf bool}. The language
  specification does not require a variable to be declared
  before use. However, the language requires that for a
  program to be valid, every variable must be assigned a {\em
    unique} type during compilation.

  The following table describes the operations permitted by
  the language, the corresponding type restrictions on the
  argument variables and the result of the operation.
  \begin{center}
    \begin{tabular}{|c|c|l|l|}\hline
      {\bf Operation} & {\bf Example} & {\bf Type Restrictions
        on Args} & {\bf Result Type} \\\hline\hline
      == & x == y & x and y have to be of the same type & bool \\
                      & & (int or bool) & \\\hline
      +  & x + y & x and y both are int & int \\\hline
      cint & cint(x) &  cast to int: x can be of any type & int \\\hline
      cbool & cbool(x) &  cast to bool: x can be of any type & bool \\\hline
    \end{tabular}
  \end{center}
  The language has following constants (each constant has a fixed type):
  \begin{center}
    \begin{tabular}{|c|c|c|} \hline
      {\bf Const Class} & {\bf Constants} & {\bf Type} \\ \hline \hline
      int\_constant & $-\infty\ldots -2, -1, 0, 1, 2, \ldots
      \infty$ & int \\ \hline 
      bool\_constant & false, true & bool \\ \hline
    \end{tabular}
  \end{center}
 The language has the following constructs and the restrictions:
 \begin{itemize}
 \item {\em assignment} statements: Both sides of an assignment
   must be of the {\em same} type.
 \item Conditionals ({\em if-else}): The condition of the {\em
     if-else} is a single variable of type bool.
 \end{itemize}
 Here are couple of sample programs, one valid and other invalid.
\begin{verbatim}
   // PROGRAM 1                     // PROGRAM 2
   c = n == e;                      c = n==e;
   e = 5;                           e = false;
   if (c) {                         if (c) {
     d = e + 3;                       d  = c + 3;
   } else {                         } else { 
     d = cint(c) - 5;                 d  = cint(e) - 5;
   }                                }
   // c: bool                      // invalid program
   // n, d, e: int                 // c has to be both bool and int! 
                                   // n, e: bool    d: int 
\end{verbatim}

 \begin{enumerate}
 \item Design a simple (as-simple-as-possible for you) syntax
   (context free grammar) for the language described.
 \item Design a {\bf flow-insensitive} type system to
   infer (or validate) types of variables for programs
   written in the above syntax.
 \end{enumerate}
\answer{Many variations possible.}
\answer{\clearpage}
\item $[10+5+15+5]$ Perform {\em
  Interprocedural Available Expressions} analysis for the
  following program. 
  
\newcommand{\ind}{$\qquad$}
  \renewcommand{\arraystretch}{1.5}
\begin{center}
  \begin{tabular}{|l@{\ind\ind}|l@{\ind\ind}|l|}\hline
    \begin{tabular}[t]{@{}l@{}}
      main() \{ \\
      \ind call p() \\
      \ind c * d \\
      \ind call q() \\
      \}
    \end{tabular}&
    \begin{tabular}[t]{@{}l@{}}
      p() \{ \\
      \ind a * b \\
      \ind if(\ldots) then \\
      \ind\ind call q() \\
      \ind else \\
      \ind\ind call q() \\
      \ind c = 5 \\
      \}
    \end{tabular}&
    \begin{tabular}[t]{@{}l@{}}
      q() \{ \\
      \ind if(\ldots) then \\
      \ind\ind a = 5 \\
      \ind else \\
      \ind\ind b = 6 \\
      \}
    \end{tabular}\\\hline
  \end{tabular}
\end{center}
Show the following steps to arrive at the answer:
\begin{enumerate}
\item Draw the inter-procedural flow graph (any one of the two representations discussed in class). Clearly label the nodes, and distinguish the $E^0$ (intra-procedural) and $E^1$ (inter-procedural) edges.
\item For each procedure $p$, show the constraints for $\phi(r_p, n_p)$ for each node $n_p$ in the CFG of the procedure. Recall that $r_p$ is the entry node for procedure $p$. Assume the existence of flow functions $f_0, f_1, id$ etc for the underlying data flow framework.
\item Give the fixed-point solution for $\phi(r_p, n_p)$.
\item List the program points where each of the expressions
  {\tt a * b} and {\tt c * d} is available.
\end{enumerate}

\answer{
%\clearpage
\noindent{\bf (a) Inter-Procedural FG:} 
\begin{center}
  \begin{tabular}{l@{\ind\ind}l@{\ind\ind}l}
    \begin{tabular}[t]{@{}l@{}}
      main() \{ \\
      \rnode{n00}{ $r_{m}$ }\\
      \rnode{n01}{ $m_1.$ call p()%
      } \\
      \rnode{n02}{ $m_2.$ c * d\rnode{n21}{} } \\
      \rnode{n03}{ $m_3.$ call q()%
      } \\
      \rnode{n04}{ $e_{m}$\rnode{n23}{} }\\
      \}
    \end{tabular}&
    \begin{tabular}[t]{@{}l@{}}
      p() \{ \\
      \rnode{n05}{ $r_p$ }\\
      \rnode{n06}{ $p_1.$ a * b }\\
      \rnode{n07}{ $p_2.$ call q()}
      \ind \rnode{n08}{  $p_3.$  call q()}\\
      \rnode{n09}{ $p_4.$ c = 5\rnode{n27}{}\rnode{n28}{} }\\
      \rnode{n10}{ $e_p$ }\\
      \}
    \end{tabular}&
    \begin{tabular}[t]{@{}l@{}}
      q() \{ \\
      \rnode{n11}{ $r_q$ }\\
      \rnode{n13}{ $q_1.$ a = 5 } \ind \rnode{n15}{  $q_2.$ b = 6 }\\
      \rnode{n16}{ $e_q$ }\\
      \}
    \end{tabular}
  \end{tabular}

  \ncline{->}{n00}{n01}
  \ncline{->}{n02}{n03}

  \ncline{->}{n05}{n06}
  \ncline{->}{n06}{n07}
  \ncline{->}{n06}{n08}
  \ncline{->}{n09}{n10}

  \ncline{->}{n11}{n13}
  \ncline{->}{n11}{n15}
  \ncline{->}{n13}{n16}
  \ncline{->}{n15}{n16}

  \nccurve[linestyle=dashed,angleB=90]{->}{n01}{n05}
  \nccurve[linestyle=dashed,angleB=75]{->}{n03}{n11}
  \nccurve[linestyle=dashed,angleB=90]{->}{n07}{n11}
  \nccurve[linestyle=dashed,angleB=135]{->}{n08}{n11}
  \nccurve[linestyle=dashed,angleB=-135,angleA=0,offsetA=.3]{<-}{n21}{n10}
  \nccurve[linestyle=dashed,angleB=-450,offsetA=.2]{<-}{n23}{n16}
  \nccurve[linestyle=dashed,angleB=-90,offsetA=.3]{<-}{n27}{n16}
  \nccurve[linestyle=dashed,angleB=-135,offsetA=.4]{<-}{n28}{n16}
  Solid lines are $E^0$ edges, dashed lines are $E^1$ edges.
\end{center}
\newcommand{\pair}[1]{\ensuremath{\langle #1 \rangle}}

Since we are tracking {\bf two} expressions, all data flow values will be pairs. The first component of the pair is for $a*b$ and the second for $c*d$. composition of function pairs will be pairwise composition, i.e.:
\[\pair{f_1, g_1} \circ \pair{f_2, g_2} \equiv \pair{f_1 \circ g_1, f_2 \circ g_2}\]
\begin{minipage}{.60\textwidth}
\noindent{\bf (b) Constraints:} 
\begin{eqnarray*}
  \phi(r_m, m_1) & = & \pair{id_L, id_L} \\
  \phi(r_m, m_2) & = & \phi(r_p, e_p)\circ\phi(r_m, m_1) \\
  \phi(r_m, m_3) & = & \pair{id_L, f_1}\circ\phi(r_m, m_2) \\
  \phi(r_m, e_m) & = & \phi(r_q, e_q)\circ\phi(r_m, m_3)\\
  \\
  \phi(r_p, p_1) & = & \pair{id_L, id_L} \\
  \phi(r_p, p_2) & = & \pair{f_1, id_L}\circ\phi(r_p, p_1)\\
  \phi(r_p, p_3) & = & \pair{f_1, id_L}\circ\phi(r_p, p_1)\\
  \phi(r_p, p_4) & = & \phi(r_q, e_q)\circ\phi(r_p, p_2)\bigwedge\phi(r_q, e_q)\circ\phi(r_p, p_3)\\
  \phi(r_p, e_p) & = & \pair{id_L,f_0}\circ\phi(r_p, p_4)\\
  \\
  \phi(r_q, q_1) & = & \pair{id_L, id_L} \\
  \phi(r_q, q_2) & = & \pair{id_L, id_L} \\
  \phi(r_q, e_q) & = & \pair{f_0, id_L}\circ\phi(r_q, q_1) \bigwedge \pair{f_0, id_L}\circ\phi(r_q, q_2)\\
\end{eqnarray*}
\end{minipage}
\begin{minipage}[t]{.05\textwidth}
  \ 
\end{minipage}
\begin{minipage}{.25\textwidth}
\noindent{\bf (c) Solutions:} 
\begin{eqnarray*}
  \phi(r_m, m_1) & = & \pair{id, id} \\
  \phi(r_m, m_2) & = & \pair{f_0, f_0}\\
  \phi(r_m, m_3) & = & \pair{f_0, f_1}\\
  \phi(r_m, e_m) & = & \pair{f_0, f_1}\\
  \\
  \phi(r_p, p_1) & = & \pair{id, id} \\
  \phi(r_p, p_2) & = & \pair{f_1, id}\\
  \phi(r_p, p_3) & = & \pair{f_1, id}\\
  \phi(r_p, p_4) & = & \pair{f_0, id}\\
  \phi(r_p, e_p) & = & \pair{f_0,f_0}\\
  \\
  \phi(r_q, q_1) & = & \pair{id, id} \\
  \phi(r_q, q_2) & = & \pair{id, id} \\
  \phi(r_q, e_q) & = & \pair{f_0, id}\\
\end{eqnarray*}
\end{minipage}

\noindent{\bf (d) Availability:} 
\begin{itemize}
\item {\bf $a*b$:} \OUT{p_1}, \IN{p_2}, \IN{p_3}
\item {\bf $c*d$:} \OUT{m_2}, \IN{m_3}, \OUT{m_3}
\end{itemize}
}
\end{enumerate}
\hrule
%\[ T~H~E \qquad E~N~D \]
%{\centering(Detach the next page and submit with your answer book.)}
\clearpage
\thispagestyle{empty}\mbox{}
\clearpage
\lhead{\fancyplain{}{{\em Name:}}}
\vspace*{-15mm}
\begin{table}[h!]
  \caption{Access Ranges for Variables.
    {\bf T} denotes true, {\bf F} denotes false.
    \label{table:acc-range}}
\centering
\renewcommand{\arraystretch}{1.35}
\begin{tabular}{|r|c|c|c|}
  \hline 
  Program Point & \multicolumn{3}{c|}{Is \pt\ In Access Range Of} \\ 
  \pt& Variable A? & Variable B& Variable C\\
  \hline\hline
  \IN{S1}  & F&F&F  \\\hline
  \OUT{S1} & T&F&F  \\\hline
  \IN{S2}  & \answer{T}&\answer{T}&\answer{F}  \\\hline
  \OUT{S2} & \answer{T}&\answer{T}&\answer{F}  \\\hline
  \IN{S3}  & \answer{T}&\answer{T}&\answer{F}  \\\hline
  \OUT{S3} & \answer{T}&\answer{T}&\answer{F}  \\\hline
  \IN{S4}  & \answer{T}&\answer{T}&\answer{F}  \\\hline
  \OUT{S4} & \answer{T}&\answer{T}&\answer{F}  \\\hline
  \IN{S5}  & \answer{T}&\answer{F}&\answer{F}  \\\hline
  \OUT{S5} & \answer{F}&\answer{F}&\answer{F}  \\\hline
  \IN{S6}  & \answer{F}&\answer{F}&\answer{F}  \\\hline
  \OUT{S6} & \answer{F}&\answer{F}&\answer{T}  \\\hline
  \hline
\end{tabular}
\end{table}
\begin{table}[h!]
\centering
\caption{Access Ranges for Sets of Variables. {\bf
      T} denotes true, {\bf F} denotes false.
    \label{table:acc-range-set}}
\renewcommand{\arraystretch}{1.35}
\begin{tabular}{|r|c|c|c|}
  \hline 
  Program Point & \multicolumn{3}{c|}{Is \pt\ In Access Range Of} \\ 
  \pt& Set \{A, B\}? & Set \{B, C\} & Set \{C, A\} \\
  \hline\hline
  \IN{S1}  & F&F&F  \\\hline
  \OUT{S1} & T&F&T  \\\hline
  \IN{S2}  & \answer{T}&\answer{T}&\answer{T}  \\\hline
  \OUT{S2} & \answer{T}&\answer{T}&\answer{T}  \\\hline
  \IN{S3}  & \answer{T}&\answer{T}&\answer{T}  \\\hline
  \OUT{S3} & \answer{T}&\answer{T}&\answer{T}  \\\hline
  \IN{S4}  & \answer{T}&\answer{T}&\answer{T}  \\\hline
  \OUT{S4} & \answer{T}&\answer{T}&\answer{T}  \\\hline
  \IN{S5}  & \answer{T}&\answer{T}&\answer{T}  \\\hline
  \OUT{S5} & \answer{F}&\answer{T}&\answer{T}  \\\hline
  \IN{S6}  & \answer{F}&\answer{T}&\answer{T}  \\\hline
  \OUT{S6} & \answer{F}&\answer{T}&\answer{T}  \\\hline 
  \hline
\end{tabular}
\end{table}
\noindent{\bf Write your roll number at the top of this page and submit along with the answer-sheet.}
\end{document}

