\documentclass[12pt]{article}
\usepackage{graphicx}
\usepackage{graphics}
\usepackage{anysize}
\usepackage{fancyhdr}
\usepackage{pstricks}
\usepackage{pst-node}
\usepackage{pst-rel-points}

\setlength{\headheight}{15pt}
\pagestyle{fancyplain}
\lhead{\fancyplain{}{{\em Page \#\/}\thepage}}
\chead{Compiler Optimizations}
\rhead{\fancyplain{}{\textit{Roll No:}\rule{30mm}{0pt}}}
\lfoot{}
\cfoot{}
\rfoot{}

\newcommand{\tbd}[1]{{\red #1}}
\newcommand{\answer}[1]{{{\blue #1}}}
%\renewcommand{\answer}[1]{\mbox{}}

%% \textwidth 7in
%% \addtolength{\oddsidemargin}{-1in}
%% \textheight 9.5in
%% \addtolength{\topmargin}{-1in}
%% \setlength{\parindent}{0pt}
%% \setlength{\parskip}{0.5cm}
%% \topskip 0.0in
\thispagestyle{empty}

\begin{document}

{\Large\bf NAME: \hspace*{3in} ROLL NO:}

\begin{center}
Compiler Optimizations \\
End of Course Examination, 2012\\
Max Time: 1.5 Hours  \hfill Max Marks: {80} \\

{\bf NOTE: }
\begin{itemize}

\item There is no separate answer script. {\bf Write down the
  answers in the space provided on the question paper.} There
  is enough space for rough work too, so do not ask for extra
  sheets. \\

\item Presenting your answers properly is your
  responsibility. You lose credit if you can not present your
  ideas clearly, and in proper form. Please DO NOT come back
  for re-evaluation saying, ``What I actually meant was
  \ldots''.

\item Be precise and write clearly. Remember that somebody
  has to read it to evaluate!

\item {\bf ONE} A4 size {\em cheat-sheet} is allowed. Sharing
  of cheat-sheet or any other exam material is {\bf not
    allowed}.
\end{itemize}



\vspace*{4mm}

{\Large
\begin{tabular}{|c|c|c|}\hline
Question No. & Max Marks & Marks Obtained \\\hline
1 & 20 & \\\hline
2 & 20 & \\\hline
3 & 40 & \\\hline
Total & \bf{80} & \\\hline
\end{tabular}
}
\end{center}

\vspace*{4mm} I pledge my honour as a gentleman/lady that
during the examination I have neither given assistance nor
received assistance.

\vspace*{1cm}
Signature


%%%%%%%%%%%%%%%%%%%%%%%%%%%%%%%%%%%%%%%%%%%%%%%%%%%%
\clearpage
\begin{enumerate}
\item Perform available expressions analysis, i.e. compute
  gen, kill, in and out for each block for the given flow
  graph. \hfill[20]
\begin{center}
\psset{unit=1mm}
\begin{pspicture}(0,0)(150,-100)
  \psframe(0,0)(150,-100)
  \putnode{be}{origin}{75}{-5}{\psframebox{ENTRY}}
  \putnode{b1}{be}{0}{-15}{\psframebox{1: u + v}}

  \putnode{b2}{b1}{-20}{-15}{\psframebox{2: w = u}}
  \putnode{b3}{b2}{0}{-15}{\psframebox{3: \ldots}}
  \putnode{b4}{b3}{0}{-15}{\psframebox{4: w + u}}

  \putnode{b5}{b1}{20}{-15}{\psframebox{5: a * b}}

  \putnode{b6}{b5}{-15}{-15}{\psframebox{6: a = c}}
  \putnode{b7}{b5}{15}{-15}{\psframebox{7: \ldots }}

  \putnode{b8}{b6}{15}{-15}{\psframebox{8: a * b}}

  \putnode{b9}{b4}{25}{-15}{\psframebox{9: u + v}}

  \putnode{bx}{b9}{0}{-15}{\psframebox{EXIT}}

  \ncline{->}{be}{b1}

  \ncline{->}{b1}{b2}
  \ncline{->}{b1}{b5}

  \ncline{->}{b2}{b3}

  \ncline{->}{b3}{b4}
  \ncline{->}{b3}{b9}
%  \nccurve[angleA=-90,angleB=90,nodesep=-.2,ncurv=0.1]{->}{b3}{b9}

  \nccurve[angleA=-135,angleB=135,nodesep=-.2,ncurv=1.8]{->}{b4}{b3}

  \ncline{->}{b5}{b6}
  \ncline{->}{b5}{b7}

  \ncline{->}{b6}{b8}
  \ncline{->}{b7}{b8}

  \ncline{->}{b8}{b9}

  \ncline{->}{b9}{bx}

\end{pspicture}
\end{center}

\begin{center}
\renewcommand{\arraystretch}{1.8}
  \begin{tabular}{|c|c|c|}
    \hline
    {\bf Block} & {\bf Gen}\rule{5cm}{0pt} & {\bf
      Kill}\rule{5cm}{0pt} \\ \hline \hline
    1 & & \\ \hline
    2 & & \\ \hline
    3 & & \\ \hline
    4 & & \\ \hline
    5 & & \\ \hline
    6 & & \\ \hline
    7 & & \\ \hline
    8 & & \\ \hline
    9 & & \\ \hline
  \end{tabular}
\end{center}
\clearpage
{\bf PASS: }
\begin{center}
\renewcommand{\arraystretch}{1.8}
  \begin{tabular}{|c|c|c|}
    \hline
    {\bf Block} & {\bf IN }\rule{5cm}{0pt} & {\bf
      OUT }\rule{5cm}{0pt} \\ \hline \hline
    Entry & & \\ \hline
    1 & & \\ \hline
    2 & & \\ \hline
    3 & & \\ \hline
    4 & & \\ \hline
    5 & & \\ \hline
    6 & & \\ \hline
    7 & & \\ \hline
    8 & & \\ \hline
    EXIT & & \\ \hline
  \end{tabular}
\end{center}
{\bf PASS: }
\begin{center}
\renewcommand{\arraystretch}{1.8}
  \begin{tabular}{|c|c|c|}
    \hline
    {\bf Block} & {\bf IN }\rule{5cm}{0pt} & {\bf
      OUT }\rule{5cm}{0pt} \\ \hline \hline
    Entry & & \\ \hline
    1 & & \\ \hline
    2 & & \\ \hline
    3 & & \\ \hline
    4 & & \\ \hline
    5 & & \\ \hline
    6 & & \\ \hline
    7 & & \\ \hline
    8 & & \\ \hline
    EXIT & & \\ \hline
  \end{tabular}
\end{center}
\clearpage
{\bf PASS: }
\begin{center}
\renewcommand{\arraystretch}{1.8}
  \begin{tabular}{|c|c|c|}
    \hline
    {\bf Block} & {\bf IN }\rule{5cm}{0pt} & {\bf
      OUT }\rule{5cm}{0pt} \\ \hline \hline
    Entry & & \\ \hline
    1 & & \\ \hline
    2 & & \\ \hline
    3 & & \\ \hline
    4 & & \\ \hline
    5 & & \\ \hline
    6 & & \\ \hline
    7 & & \\ \hline
    8 & & \\ \hline
    EXIT & & \\ \hline
  \end{tabular}
\end{center}
{\bf PASS: }
\begin{center}
\renewcommand{\arraystretch}{1.8}
  \begin{tabular}{|c|c|c|}
    \hline
    {\bf Block} & {\bf IN }\rule{5cm}{0pt} & {\bf
      OUT }\rule{5cm}{0pt} \\ \hline \hline
    Entry & & \\ \hline
    1 & & \\ \hline
    2 & & \\ \hline
    3 & & \\ \hline
    4 & & \\ \hline
    5 & & \\ \hline
    6 & & \\ \hline
    7 & & \\ \hline
    8 & & \\ \hline
    EXIT & & \\ \hline
  \end{tabular}
\end{center}
\clearpage
\mbox{}
\clearpage
%%%%%%%%%%%%%%%%%%%%%%%%%%%%%%%%%%%%%%%%%%%%%%%%%%%%
\item Perform reaching definition analysis, i.e. compute
  gen, kill, in and out for each block for the given flow
  graph.\hfill[20]
\begin{center}
\psset{unit=1mm}
\begin{pspicture}(0,0)(150,-100)
  \psframe(0,0)(150,-100)
  \putnode{be}{origin}{75}{-5}{\psframebox{ENTRY}}
  \putnode{b1}{be}{0}{-15}{\psframebox{1: u = v}}

  \putnode{b2}{b1}{-20}{-15}{\psframebox{2: w = u + v}}
  \putnode{b3}{b2}{0}{-15}{\psframebox{3: u = 7}}
  \putnode{b4}{b3}{0}{-15}{\psframebox{4: w = 10}}

  \putnode{b5}{b1}{20}{-15}{\psframebox{5: v = u}}

  \putnode{b6}{b5}{-15}{-15}{\psframebox{6: u = u+ w}}
  \putnode{b7}{b5}{15}{-15}{\psframebox{7: v = w + v}}

  \putnode{b8}{b6}{15}{-15}{\psframebox{8: \ldots}}

  \putnode{b9}{b4}{25}{-15}{\psframebox{9: \ldots}}

  \putnode{bx}{b9}{0}{-15}{\psframebox{EXIT}}

  \ncline{->}{be}{b1}

  \ncline{->}{b1}{b2}
  \ncline{->}{b1}{b5}

  \ncline{->}{b2}{b3}

  \ncline{->}{b3}{b4}
  \ncline{->}{b3}{b9}
%  \nccurve[angleA=-90,angleB=90,nodesep=-.2,ncurv=0.1]{->}{b3}{b9}

  \nccurve[angleA=-135,angleB=135,nodesep=-.2,ncurv=1.8]{->}{b4}{b3}

  \ncline{->}{b5}{b6}
  \ncline{->}{b5}{b7}

  \ncline{->}{b6}{b8}
  \ncline{->}{b7}{b8}

  \ncline{->}{b8}{b9}

  \ncline{->}{b9}{bx}

\end{pspicture}
\end{center}

\begin{center}
\renewcommand{\arraystretch}{1.8}
  \begin{tabular}{|c|c|c|}
    \hline
    {\bf Block} & {\bf Gen}\rule{5cm}{0pt} & {\bf
      Kill}\rule{5cm}{0pt} \\ \hline \hline
    1 & & \\ \hline
    2 & & \\ \hline
    3 & & \\ \hline
    4 & & \\ \hline
    5 & & \\ \hline
    6 & & \\ \hline
    7 & & \\ \hline
    8 & & \\ \hline
    9 & & \\ \hline
  \end{tabular}
\end{center}
\clearpage
{\bf PASS: }
\begin{center}
\renewcommand{\arraystretch}{1.8}
  \begin{tabular}{|c|c|c|}
    \hline
    {\bf Block} & {\bf IN }\rule{5cm}{0pt} & {\bf
      OUT }\rule{5cm}{0pt} \\ \hline \hline
    Entry & & \\ \hline
    1 & & \\ \hline
    2 & & \\ \hline
    3 & & \\ \hline
    4 & & \\ \hline
    5 & & \\ \hline
    6 & & \\ \hline
    7 & & \\ \hline
    8 & & \\ \hline
    EXIT & & \\ \hline
  \end{tabular}
\end{center}
{\bf PASS: }
\begin{center}
\renewcommand{\arraystretch}{1.8}
  \begin{tabular}{|c|c|c|}
    \hline
    {\bf Block} & {\bf IN }\rule{5cm}{0pt} & {\bf
      OUT }\rule{5cm}{0pt} \\ \hline \hline
    Entry & & \\ \hline
    1 & & \\ \hline
    2 & & \\ \hline
    3 & & \\ \hline
    4 & & \\ \hline
    5 & & \\ \hline
    6 & & \\ \hline
    7 & & \\ \hline
    8 & & \\ \hline
    EXIT & & \\ \hline
  \end{tabular}
\end{center}
\clearpage
{\bf PASS: }
\begin{center}
\renewcommand{\arraystretch}{1.8}
  \begin{tabular}{|c|c|c|}
    \hline
    {\bf Block} & {\bf IN }\rule{5cm}{0pt} & {\bf
      OUT }\rule{5cm}{0pt} \\ \hline \hline
    Entry & & \\ \hline
    1 & & \\ \hline
    2 & & \\ \hline
    3 & & \\ \hline
    4 & & \\ \hline
    5 & & \\ \hline
    6 & & \\ \hline
    7 & & \\ \hline
    8 & & \\ \hline
    EXIT & & \\ \hline
  \end{tabular}
\end{center}
{\bf PASS: }
\begin{center}
\renewcommand{\arraystretch}{1.8}
  \begin{tabular}{|c|c|c|}
    \hline
    {\bf Block} & {\bf IN }\rule{5cm}{0pt} & {\bf
      OUT }\rule{5cm}{0pt} \\ \hline \hline
    Entry & & \\ \hline
    1 & & \\ \hline
    2 & & \\ \hline
    3 & & \\ \hline
    4 & & \\ \hline
    5 & & \\ \hline
    6 & & \\ \hline
    7 & & \\ \hline
    8 & & \\ \hline
    EXIT & & \\ \hline
  \end{tabular}
\end{center}
\clearpage
\mbox{}
\clearpage
\clearpage

%%%%%%%%%%%%%%%%%%%%%%%%%%%%%%%%%%%%%%%%%%%%%%%%%%%%
\item {\bf Sign Analysis} :

  Consider a simple programming language with only integer
  variables and two operations:  {\bf multiplication
    ($\times$)} and {\bf addition ($+$)}, where the operations
  have their natural meaning. 

  The language has {\em Positive Constants} (including Zero) and
  {\em Negative Constants}, i.e.:
  \begin{center}
    \begin{tabular}{|c|c|c|} \hline
      {\bf Constant} & {\bf Sign} \\ \hline \hline
      $0, 1, 2, 3, \ldots$ & Positive \\ \hline
      $-1, -2, -3, \ldots$ & Negative \\ \hline
    \end{tabular}
  \end{center}

  The language obviously contains basic constructs like
  {\em assignment} statements, simple branches and while loops
  (all the cases as discussed in class).
  
  We want to compute at each program point the {\bf POSITIVE}
  set of variables, i.e. variables that have values
  {\bf definitely $\geq 0$}. Note that we do not want to compute
  the values of variables.

  The basic rules, in English, are:
  \begin{enumerate}
  \item All Positive constants are always POSITIVE.
  \item  For $x = y$ (y is a constants or a variable), the
    variable ``x'' is POSITIVE at OUT if ``y'' is POSITIVE at
    IN of the statement.
  \item For $x = y + z$ (each of y and z is a  constant or a
    variable), ``x'' is POSITIVE at OUT if {\bf both} ``y'' and ``z''
    are POSITIVE at IN.
  \item For $x = y \times z$ (each of y and z is a  constant or a
    variable), ``x'' is POSITIVE at OUT if {\bf both} ``y'' and ``z''
    are POSITIVE at IN.
  \item Any variable other than ``x'' is unaffected by the
    assignment $x = \ldots$.
  \item Initially, all variables are considered  not  POSITIVE.
  \end{enumerate}

For example
\begin{verbatim}
   // STATEMENT    // PROPERTIES  
   x = 5;      // x is POSITIVE at OUT
   y = -1 * x; // x is POSITIVE, y is not POSITIVE at OUT
   z = y * y;  // x POSITIVE, y not, z POSITIVE

   if (something) {
       w = y; // w not POSITIVE
   } else {
       w = x; // w POSITIVE
   }
   // at this point w is NOT POSITIVE because it is 
   // positive along one path, but not other.

\end{verbatim}

We want to design an {\bf intraprocedural flow-insensitive} data flow
analysis  to infer signs of variables for programs
written in the above language. Answer the following questions.
\begin{enumerate}
\item Define the gen and kill constraints for each type of
  basic statement (x = y, x = y + z, x = y $\times$
  z). \hfill[12]
  
\item Describe the gen and kill constraints for each type of
  structure in the program (sequence of statements,
  if-branch, while loop). \hfill[12]
  
\item Is your Analysis {\em Forward} or {\em Backward}?
  Justify your answer. Also describe the {\em BoundaryInfo},
  the initialization information at the boundary (ENTRY or EXIT)
  of the flow  graph. \hfill[2+3] 
  
\item What is the optimistic value (top) and the pessimistic
  value (bot) for this analysis?\linebreak\mbox{ }  \hfill[5]
  
\item Consider the rule for $x = y \times z$. Suppose we
  change the rule as follows:\\
{\em
  For $x = y \times z$ (each of y and z is a
  constant or a variable), ``x'' is POSITIVE at OUT if ``y''
  and ``z'' are either {\bf both} POSITIVE or {\bf both} NOT
  POSITIVE.
}\\
Is this rule correct? If yes, prove it. If no, give an
 example program where we get incorrect results. \hfill[6]
\end{enumerate}
\clearpage
\mbox{}
\clearpage
\mbox{}
\clearpage
\mbox{}
\clearpage
\end{enumerate}
\end{document}

