\mode<presentation>
{
  %\usetheme{umbc4}
  %\setbeamercovered{dynamic}
}
\usepackage[english]{babel}
% or whatever

\usepackage[latin1]{inputenc}
% or whatever

\usepackage{times}
\usepackage[T1]{fontenc}
% Or whatever. Note that the encoding and the font should match. If T1
% does not look nice, try deleting the line with the fontenc.
%--------------------------------------------------------------------------------------------
%--- Other packages
\usepackage{pstricks}
\usepackage{pst-node}
\usepackage{pst-rel-points}

\newcommand{\change}[1]{{\red #1}}
%%%%%% Declarations Defined in custom-foils
%\setfootline{\insertshortauthor, \insertshortinstitute \hfill   \insertshorttitle \hfill \insertframenumber/\inserttotalframenumber} 
\newcommand{\mypart}[1]{%
  \frame[plain]{\mbox{}\vfill \psshadowbox{\huge #1} \vfill\mbox{}}
}
\author[karkare]{Amey Karkare \\ \url{karkare@cse.iitk.ac.in}}
\date[]{\scalebox{0.3}{\includegraphics{iitklogo.epsi}}}%
\institute[CSE, IITK]{\url{http://www.cse.iitk.ac.in/~karkare/cs738}\\Department of CSE, IIT Kanpur}
\title[CS738]{CS738: Advanced Compiler Optimizations}
\subtitle[]{}


\subtitle[]{\ \\{\LARGE SSAPRE: SSA based Partial Redundancy
    Elimination }}

\begin{document}

\frame{\titlepage}
\frame{
  \frametitle{PRE without SSA}
  \begin{itemize}[<+->]
  \item Based on well known DF analyses
    \begin{itemize}[<+->]
    \item Availability
    \item Anticipability
    \item Partial Availability
    \item Partial Anticipability
    \end{itemize}
  \item Identifies paritially redundant computations, make
    them totally redundant by inserting new compitations
  \item Remove totally redundant computations (CSE)
  \end{itemize}
}

\frame{
  \frametitle{PRE without SSA}
  \begin{itemize}[<+->]
  \item Iterative data flow analysis
  \item Operates on control flow graph
  \item Computes global and local versions of data flow
    information 
  \end{itemize}
}

\frame{
  \frametitle{SSAPRE}
  \begin{itemize}[<+->]
  \item Information flow along SSA edges
  \item No distinction between global and local information
  \end{itemize}
}

\frame{
  \frametitle{SSAPRE: Challenge}
  \begin{itemize}[<+->]
  \item SSA form defined for variables
  \item How to identify potentially redundant expressions
    \begin{itemize}[<+->]
    \item Expressions having different variable versions as
      operands
      \uncover<4->{
      \begin{center}
        \psset{unit=1mm}
        \begin{pspicture}(0,0)(40,-40)
          %\psframe(0,0)(40,-40)
          \putnode{b1}{origin}{10}{-10}{\psframebox{$a_1 + b_1$}}
          \putnode{b2}{b1}{20}{0}{\psframebox{$a_2 =
              \cdots$}}
          \putnode{b3}{b1}{10}{-10}{\psframebox{$a_3 = \phi(a_1, a_2)$}}
          \putnode{b4}{b3}{0}{-10}{\psframebox{$a_3 + b_1$}}
          \ncline{->}{b1}{b3}
          \ncline{->}{b2}{b3}
          \ncline{->}{b3}{b4}
        \end{pspicture}
      \end{center}}
    \item Here $ a_1 + b_1$ is same as $a_3 + b_1$. Lexically
      different, but computationally identical
    \end{itemize}
  \end{itemize}
}
\frame{
  \frametitle{SSAPRE: Key Idea}
  \begin{itemize}[<+->]
  \item Redundancy Class Variables (RCVs) 
    \begin{itemize}[<+->]
    \item variable (say $h$) to represent computation of
      an expression (say $E$)
    \end{itemize}
  \item Computation of expression could represent either a
    {\em def} or a {\em use} 
    \begin{itemize}[<+->]
    \item definition of $E \Rightarrow$ store into $h$
    \item use of $E \Rightarrow$ load from $h$
    \end{itemize}
  \item PRE on SSA form of RCVs ($h$) to remove redundancies 
  \item Final program will be in SSA form
  \end{itemize}
}

\frame{
  \frametitle{SSAPRE: Preparations}
  \begin{itemize}[<+->]
  \item Split all the {\em critical edges} in the flow graph
    \begin{itemize}[<+->]
    \item Edge from a node with more than one successor to a
      node with more than one predecessor
    \item WHY is this important?
    \end{itemize}
  \item Single pass to identify identical expressions
    \begin{itemize}[<+->]
    \item Ignoring the version number of the operands
    \item $a_3 + b_1$ and $a_1 + b_1$ are identical
    \end{itemize}
  \end{itemize}
}

\frame{
  \frametitle{SSAPRE Steps}
  \begin{itemize}[<+->]
  \item Six step algorithm
    \begin{enumerate}[<+->]
    \item $\Phi$-insertion
    \item Renaming
    \item Down-safety computation
    \item WillBeAvail computation
    \item Finalization
    \item Code Motion
    \end{enumerate}
  \end{itemize}
}

\frame{
  \frametitle{Running Example}
  \begin{center}
    \psset{unit=1mm}
    \begin{pspicture}(0,0)(60,-70)
      %\psframe(0,0)(60,-70)
      \putnode{b1}{origin}{20}{0}{\psframebox{$a_1 = \cdots$}}
      \putnode{b2}{b1}{0}{-15}{\psframebox{$a_2 = \phi(a_4, a_1)$}}
      \putnode{b3}{b2}{20}{-15}{\psframebox{
          \begin{tabular}{@{}r@{}c@{}l@{}}
            &&$a_2 + b_1$\\ 
            $a_3$&=&$\cdots$
          \end{tabular}
      }}
      \putnode{b4}{b3}{-20}{-15}{\psframebox{
          \begin{tabular}{@{}r@{}c@{}l@{}}
            $a_4$&=&$\phi(a_2, a_3)$\\ 
            && $a_4 + b_1$
          \end{tabular}
      }}
      \putnode{b5}{b4}{0}{-15}{\psframebox{EXIT}}
      \ncline{->}{b1}{b2}
      \ncline{->}{b2}{b3}
      \ncline{->}{b2}{b4}
      \ncline{->}{b3}{b4}
      \ncline{->}{b4}{b5}
      \nccurve[angleA=-135, angleB=135,ncurv=1.5]{->}{b4}{b2}
    \end{pspicture}
  \end{center}
}

\frame{
  \frametitle{$\Phi$-insertion}
  \begin{itemize}[<+->]
  \item $\Phi$ for an expression $E$ is required where two
    potentially different values of an expression merge
  \item At iterated dominance frontiers of occurances of $E$
  \item At each block having a $\phi$ for some argument of
    $E$ 
    \begin{itemize}[<+->]
    \item Potential change in the expression's value
    \end{itemize}
  \end{itemize}
}

\frame{
  \frametitle{$\Phi$-insertion}
  \begin{center}
    \psset{unit=1mm}
    \begin{pspicture}(0,0)(60,-70)
      %\psframe(0,0)(60,-70)
      \putnode{b1}{origin}{20}{0}{\psframebox{$a_1 = \cdots$}}
      \putnode{b2}{b1}{0}{-15}{\psframebox{
          \begin{tabular}{@{}r@{}c@{}l@{}}
            &&$\Phi(\cdots)$\\
            $a_2$ &=& $\phi(a_4, a_1)$
          \end{tabular}
      }}
      \putnode{b3}{b2}{20}{-15}{\psframebox{
          \begin{tabular}{@{}r@{}c@{}l@{}}
            &&$a_2 + b_1$\\ 
            $a_3$&=&$\cdots$
          \end{tabular}
      }}
      \putnode{b4}{b3}{-20}{-20}{\psframebox{
          \begin{tabular}{@{}r@{}c@{}l@{}}
            &&$\Phi(\cdots)$\\
            $a_4$&=&$\phi(a_2, a_3)$\\ 
            && $a_4 + b_1$
          \end{tabular}
      }}
      \putnode{b5}{b4}{0}{-15}{\psframebox{EXIT}}
      \ncline{->}{b1}{b2}
      \ncline{->}{b2}{b3}
      \ncline{->}{b2}{b4}
      \ncline{->}{b3}{b4}
      \ncline{->}{b4}{b5}
      \nccurve[angleA=-135, angleB=135,ncurv=1.5]{->}{b4}{b2}
    \end{pspicture}
  \end{center}
}

\frame{
  \frametitle{Rename}
  \begin{itemize}[<+->]
  \item Similar to SSA variable renaming
  \item Stack of every expression is maintained
  \item Three kinds of occurrences of $E$
    \begin{itemize}[<+->]
    \item Real occurrences (present in original program)
    \item Results of $\Phi$ operators inserted
    \item Operands of inserted $\Phi$
    \end{itemize}
  \item After renaming
    \begin{itemize}[<+->]
    \item Identical SSA instances of $h$ represent identical
      values of $E$
    \item A control flow path with two different instances of
      $h$ has to cross either an assignment to an operand of
      $E$ or a $\Phi$ of $h$
    \end{itemize}
  \end{itemize}
}

\frame{
  \frametitle{Rename Algorithm}
  \begin{itemize}[<+->]
  \item Runs with variable renaming
  \item When an $E$ is encountered
    \begin{itemize}[<+->]
    \item if $E$ is result of $\Phi$, assign a new version to
      $h$ and push it on $E$ stack
    \item if $E$ is the real occurrence
      \begin{itemize}[<+->]
      \item for each operand, compare the version of operand
        with the top of the rename stack for operand
      \item If all match, $h$ gets same version as the top of
        $E$ stack
      \item If any mismatch, assign a new version to $h$ and
        push it on $E$ stack
      \end{itemize}
    \item if $E$ is operand of $\Phi$, in the corresponding
      predecessor block
      \begin{itemize}[<+->]
      \item for each operand of $E$, compare the version of operand
        with the top of the rename stack for operand
      \item If all match, $h$ gets same version as the top of
        $E$ stack
      \item If any mismatch, replace $E$ by $\bot$ in the operand
        push it on $E$ stack (WHY?)
      \end{itemize}
    \end{itemize}
  \end{itemize}
}

\frame{
  \frametitle{Rename}
  \begin{center}
    \psset{unit=1mm}
    \begin{pspicture}(0,0)(60,-70)
      %\psframe(0,0)(60,-70)
      \putnode{b1}{origin}{20}{0}{\psframebox{$a_1 = \cdots$}}
      \putnode{b2}{b1}{0}{-15}{\psframebox{
          \begin{tabular}{@{}r@{}c@{}l@{}}
            $h_1$&=&$\Phi(h_2, \bot)$\\
            $a_2$ &=& $\phi(a_4, a_1)$
          \end{tabular}
      }}
      \putnode{b3}{b2}{20}{-15}{\psframebox{
          \begin{tabular}{@{}r@{}c@{}l@{}}
            &&$a_2 + b_1 : [h_1]$\\ 
            $a_3$&=&$\cdots$
          \end{tabular}
      }}
      \putnode{b4}{b3}{-20}{-20}{\psframebox{
          \begin{tabular}{@{}r@{}c@{}l@{}}
            $h_2$&=&$\Phi(h_1, \bot)$\\
            $a_4$&=&$\phi(a_2, a_3)$\\ 
            && $a_4 + b_1 : [h_2]$
          \end{tabular}
      }}
      \putnode{b5}{b4}{0}{-15}{\psframebox{EXIT}}
      \ncline{->}{b1}{b2}
      \ncline{->}{b2}{b3}
      \ncline{->}{b2}{b4}
      \ncline{->}{b3}{b4}
      \ncline{->}{b4}{b5}
      \nccurve[angleA=-135, angleB=135,ncurv=1.5]{->}{b4}{b2}
    \end{pspicture}
  \end{center}
}

\frame{
  \frametitle{Down-safety}
  \begin{itemize}[<+->]
  \item Down-safety is same as very-busy (anticipability)
    property of expressions
    \begin{itemize}[<+->]
    \item Do not want to introduce new computation of $E$
    \end{itemize}
  \item We only need to compute down-safety for inserted
    $\Phi$-operators 
  \item A $\Phi$ computation is {\bf NOT} down-safe if
    \item there is a path to EXIT from $\Phi$ along which
      the result of $\Phi$ is 
    \begin{itemize}[<+->]
    \item either not used
    \item used only as an operand of another $\Phi$ that
      itself is {\bf NOT} down-safe
    \end{itemize}
  \item \mbox{\em HasRealUse}: Real occurrence of an
    expression
  \end{itemize}
}

\frame{
  \frametitle{Down-safety (ds = $\cdots$)}
  \begin{center}
    \psset{unit=1mm}
    \begin{pspicture}(0,0)(60,-70)
      %\psframe(0,0)(60,-70)
      \putnode{b1}{origin}{20}{0}{\psframebox{$a_1 = \cdots$}}
      \putnode{b2}{b1}{0}{-15}{\psframebox{
          \begin{tabular}{@{}r@{}c@{}l@{}}
            $h_1$&=&$\Phi(h_2, \bot) : [ds=1]$\\
            $a_2$ &=& $\phi(a_4, a_1)$
          \end{tabular}
      }}
      \putnode{b3}{b2}{20}{-15}{\psframebox{
          \begin{tabular}{@{}r@{}c@{}l@{}}
            &&$a_2 + b_1 : [h_1]$\\ 
            $a_3$&=&$\cdots$
          \end{tabular}
      }}
      \putnode{b4}{b3}{-20}{-20}{\psframebox{
          \begin{tabular}{@{}r@{}c@{}l@{}}
            $h_2$&=&$\Phi(h_1, \bot) : [ds=1]$\\
            $a_4$&=&$\phi(a_2, a_3)$\\ 
            && $a_4 + b_1 : [h_2]$
          \end{tabular}
      }}
      \putnode{b5}{b4}{0}{-15}{\psframebox{EXIT}}
      \ncline{->}{b1}{b2}
      \ncline{->}{b2}{b3}
      \ncline{->}{b2}{b4}
      \ncline{->}{b3}{b4}
      \ncline{->}{b4}{b5}
      \nccurve[angleA=-135, angleB=135,ncurv=1.5]{->}{b4}{b2}
    \end{pspicture}
  \end{center}
}

\frame{
  \frametitle{WillBeAvail}
  \begin{itemize}[<+->]
  \item The set of $\Phi$s where the expression must be
    available in any computationally optimal placement
  \item Computation of {\em two forward} properties:
    \begin{itemize}[<+->]
    \item {\em CanBeAvail\/}: $\Phi$s for which $E$ is either
      available or anticipable or both
    \item {\em Later\/}: $\Phi$s beyond which insertion can not
      be postponed without introducing new redundancy
    \end{itemize}
    \uncover<4-> {
    $$\mbox{\em WillBeAvail} = \mbox{\em CanBeAvail} \wedge
    \neg\mbox{\em Later}$$ 
}
  \end{itemize}
}

\frame{
  \frametitle{CanBeAvail}
  \begin{itemize}[<+->]

  \item Initialized to {\em true} for all $\Phi$s
  \item Boundary $\Phi$s: 
    \begin{itemize}[<+->]

    \item Not Down-safe, and
    \item At least one argument is $\bot$
    \end{itemize}
  \item Set {\em false} for boundary $\Phi$s
  \item Propagate {\em false} value along the chain of
    def-use to other $\Phi$s 
    \begin{itemize}[<+->]

    \item exclude edges along which  {\em HasRealUse} is {\em true} 
    \end{itemize}
  \end{itemize}
}

\frame{
  \frametitle{Later}
  \begin{itemize}[<+->]

  \item Determines latest (final) insertion points
  \item Initialize Later to {\em true} wherever CanBeAvail is
    {\em true}, otherwise false
  \item Assign {\em false} for $\Phi$s with at least one
    operand with HasRealUse flag {\em true}
  \item Propagate {\em false} value forward to other $\Phi$s
  \item Later $\Rightarrow$ $\Phi$s that are CanBeAvail, but
    do not reach any real occurrence of $E$
  \end{itemize}
}

\frame{
  \frametitle{Insertion Points}
  \begin{itemize}[<+->]
  \item Insertions are done for $\Phi$ operands
  \item Along the corresponding predecessor edges
  \item Insertion done along $i^{th}$ predecessor of $\Phi$
    if {\em Insert} is {\em true}, i.e.
    \begin{itemize}[<+->]
    \item {\em WillBeAvail}($\Phi$) == {\em true}; AND
    \item $Arg_i \mbox{ is } \bot$; OR 
      \begin{itemize}[<+->]
      \item (HasRealUse($Arg_i$) == \mbox{\em false}), AND
      \item $Arg_i$ is defined by $\Phi'$ with {\em
        WillBeAvail}($\Phi'$) == {\em false}
      \end{itemize}
    \end{itemize}
  \end{itemize}
}

\frame{
  \frametitle{Finalize}
  \begin{itemize}[<+->]
  \item Transforms the program with RCVs into a valid SSA
    form 
  \item For every real occurrence of $E$, decide whether it is
    a {\em def} or a {\em use}
  \item For every $\Phi$ with {\em WillBeAvail} being {\em true},
    insert $E$ along incoming edges with {\em Insert} being
    {\em true} 
  \item For each $\Phi$ for $E$
    \begin{itemize}[<+->]
    \item If {\em WillBeAvail} is {\em true}, it is replaced by SSA
      temporary with appropriate version ($h_x$)
    \item If {\em WillBeAvail} is {\em false}, it is not part of
      SSA form, and is removed
    \end{itemize}
  \end{itemize}
}

\frame{
  \frametitle{Finalize: AvailDef}
  \begin{itemize}[<+->]

  \item AvailDef: Table to mark def of expression
    occurrences
  \item Computed for each class (say $h_x$) of $E$
  \item Preorder traversal of dominator tree 
  \end{itemize}
}

\frame{
  \frametitle{AvailDef Computation}
  \begin{itemize}[<+->]

  \item Initialize: AvailDef[$x$] = $\bot \ \forall x$ (all
    classes of all expressions) 
  \item During course of traversal, process occurrence $x$ of
    $E$
    \begin{itemize}[<+->]

    \item $\Phi$ occurrence: 
      \begin{itemize}[<+->]

      \item If WillBeAvail is {\em false}, ignore. 
      \item Otherwise AvailDef[$x$] = this $\Phi$ (we must
        be visiting $x$ for first time)  \pause -- WHY?
      \end{itemize}
    \item Real occurrence: 
      \begin{itemize}[<+->]
      \item If AvailDef[$x$] is $\bot$, mark this
        occurrence as def 
      \item Else, if AvailDef[$x$] does not dominate this
        occurrence, mark this occurrence as def
      \item Else, mark this occurrence as use of AvailDef[$x$]
      \end{itemize}
    \item $\Phi$ operand (processed in predecessor block $P$)
      \begin{itemize}[<+->]
      \item If WillBeAvail of $\Phi$ is false, ignore.
      \item Else, if {\em Insert} is true for the operand,
        insert computation of $E$ in block $P$, set it as a
        def, mark this occurrence as use of inserted.
      \item Else ({\em Insert} is false), mark this occurrence
        as use of AvailDef[$x$]
      \end{itemize}
    \end{itemize}
  \end{itemize}
}
\frame{
  \frametitle{Finalize}
  \begin{center}
    \begin{tabular}{ccc}
    \scalebox{0.8}{\psset{unit=1mm}
    \begin{pspicture}(0,0)(60,-70)
      %\psframe(0,0)(60,-70)
      \putnode{b1}{origin}{20}{0}{\psframebox{$a_1 = \cdots$}}
      \putnode{b2}{b1}{0}{-15}{\psframebox{
          \begin{tabular}{@{}r@{}c@{}l@{}}
            $h_1$&=&$\Phi(h_2, \bot)$\\
            $a_2$ &=& $\phi(a_4, a_1)$
          \end{tabular}
      }}
      \putnode{b3}{b2}{20}{-15}{\psframebox{
          \begin{tabular}{@{}r@{}c@{}l@{}}
            &&$a_2 + b_1 : [h_1]$\\ 
            $a_3$&=&$\cdots$
          \end{tabular}
      }}
      \putnode{b4}{b3}{-20}{-20}{\psframebox{
          \begin{tabular}{@{}r@{}c@{}l@{}}
            $h_2$&=&$\Phi(h_1, \bot)$\\
            $a_4$&=&$\phi(a_2, a_3)$\\ 
            && $a_4 + b_1 : [h_2]$
          \end{tabular}
      }}
      \putnode{b5}{b4}{0}{-15}{\psframebox{EXIT}}
      \ncline{->}{b1}{b2}
      \ncline{->}{b2}{b3}
      \ncline{->}{b2}{b4}
      \ncline{->}{b3}{b4}
      \ncline{->}{b4}{b5}
      \nccurve[angleA=-135, angleB=135,ncurv=1.5]{->}{b4}{b2}
    \end{pspicture}}
      & 
    \scalebox{0.8}{\psset{unit=1mm}
    \begin{pspicture}(0,0)(5,-70)
      \putnode{arr}{origin}{0}{-35}{\Huge $\Rightarrow$}
    \end{pspicture}
      }&
    \scalebox{0.8}{\psset{unit=1mm}
    \begin{pspicture}(0,0)(60,-70)
      %\psframe(0,0)(60,-70)
      \putnode{b1}{origin}{20}{0}{\psframebox{
          \begin{tabular}{@{}r@{}c@{}l@{}}
            $a_1$ &=& $\cdots$ \\
            $h_3$ &=& $\cdots$
          \end{tabular}
      }}
      \putnode{b2}{b1}{0}{-15}{\psframebox{
          \begin{tabular}{@{}r@{}c@{}l@{}}
            $h_1$&=&$\Phi(h_2, h_3)$\\
            $a_2$ &=& $\phi(a_4, a_1)$
          \end{tabular}
      }}
      \putnode{b3}{b2}{20}{-18}{\psframebox{
          \begin{tabular}{@{}r@{}c@{}l@{}}
            &&$a_2 + b_1 : [h_1]$\\ 
            $a_3$&=&$\cdots$ \\ 
            $h_4$&=& $\cdots$
          \end{tabular}
      }}
      \putnode{b4}{b3}{-20}{-20}{\psframebox{
          \begin{tabular}{@{}r@{}c@{}l@{}}
            $h_2$&=&$\Phi(h_1, h_4)$\\
            $a_4$&=&$\phi(a_2, a_3)$\\ 
            && $a_4 + b_1 : [h_2]$
          \end{tabular}
      }}
      \putnode{b5}{b4}{0}{-15}{\psframebox{EXIT}}
      \ncline{->}{b1}{b2}
      \ncline{->}{b2}{b3}
      \ncline{->}{b2}{b4}
      \ncline{->}{b3}{b4}
      \ncline{->}{b4}{b5}
      \nccurve[angleA=-135, angleB=135,ncurv=1.2]{->}{b4}{b2}
    \end{pspicture}}
    \end{tabular}
  \end{center}
}

\frame{
  \frametitle{Code Motion}
  \begin{itemize}[<+->]
  \item  For real {\em def} occurance of $E$, compute $E$ in
    a new version of temporary $t$
  \item  For real {\em use} occurance of $E$, replace $E$ by
    current version of $t$
  \item For inserted occurrence of $E$, compute $E$ in
    a new version of temporary $t$
  \item For a $\Phi$ occurrence, insert appropriate $\phi$ for
    $t$ 
  \end{itemize}
}

\frame{
  \frametitle{Code Motion}
  \begin{center}
    \begin{tabular}{ccc}
    \scalebox{0.8}{\psset{unit=1mm}
    \begin{pspicture}(0,0)(60,-70)
      %\psframe(0,0)(60,-70)
      \putnode{b1}{origin}{20}{0}{\psframebox{
          \begin{tabular}{@{}r@{}c@{}l@{}}
            $a_1$ &=& $\cdots$ \\
            $h_3$ &=& $\cdots$
          \end{tabular}
      }}
      \putnode{b2}{b1}{0}{-15}{\psframebox{
          \begin{tabular}{@{}r@{}c@{}l@{}}
            $h_1$&=&$\Phi(h_2, h_3)$\\
            $a_2$ &=& $\phi(a_4, a_1)$
          \end{tabular}
      }}
      \putnode{b3}{b2}{20}{-18}{\psframebox{
          \begin{tabular}{@{}r@{}c@{}l@{}}
            &&$a_2 + b_1 : [h_1]$\\ 
            $a_3$&=&$\cdots$ \\ 
            $h_4$&=& $\cdots$
          \end{tabular}
      }}
      \putnode{b4}{b3}{-20}{-20}{\psframebox{
          \begin{tabular}{@{}r@{}c@{}l@{}}
            $h_2$&=&$\Phi(h_1, h_4)$\\
            $a_4$&=&$\phi(a_2, a_3)$\\ 
            && $a_4 + b_1 : [h_2]$
          \end{tabular}
      }}
      \putnode{b5}{b4}{0}{-15}{\psframebox{EXIT}}
      \ncline{->}{b1}{b2}
      \ncline{->}{b2}{b3}
      \ncline{->}{b2}{b4}
      \ncline{->}{b3}{b4}
      \ncline{->}{b4}{b5}
      \nccurve[angleA=-135, angleB=135,ncurv=1.2]{->}{b4}{b2}
    \end{pspicture}}
    &
    \scalebox{0.8}{\psset{unit=1mm}
    \begin{pspicture}(0,0)(5,-70)
      \putnode{arr}{origin}{0}{-35}{\Huge $\Rightarrow$}
    \end{pspicture}
    }
    &
    \scalebox{0.8}{
    \psset{unit=1mm}
    \begin{pspicture}(0,0)(60,-70)
      %\psframe(0,0)(60,-70)
      \putnode{b1}{origin}{20}{0}{\psframebox{
          \begin{tabular}{@{}r@{}c@{}l@{}}
            $a_1$ &=& $\cdots$ \\
            $t_1$ &=& $a_1 + b_1$
          \end{tabular}
      }}
      \putnode{b2}{b1}{0}{-18}{\psframebox{
          \begin{tabular}{@{}r@{}c@{}l@{}}
            $t_2$&=&$\phi(t_4, t_1)$\\
            $a_2$ &=& $\phi(a_4, a_1)$
          \end{tabular}
      }}
      \putnode{b3}{b2}{20}{-18}{\psframebox{
          \begin{tabular}{@{}r@{}c@{}l@{}}
            &&$t_2$\\ 
            $a_3$&=&$\cdots$ \\ 
            $t_3$&=& $a_3 + b_1$
          \end{tabular}
      }}
      \putnode{b4}{b3}{-20}{-20}{\psframebox{
          \begin{tabular}{@{}r@{}c@{}l@{}}
            $t_4$&=&$\phi(t_2, t_3)$\\
            $a_4$&=&$\phi(a_2, a_3)$\\ 
            && $t_4$
          \end{tabular}
      }}
      \putnode{b5}{b4}{0}{-15}{\psframebox{EXIT}}
      \ncline{->}{b1}{b2}
      \ncline{->}{b2}{b3}
      \ncline{->}{b2}{b4}
      \ncline{->}{b3}{b4}
      \ncline{->}{b4}{b5}
      \nccurve[angleA=-135, angleB=135,ncurv=1.2]{->}{b4}{b2}
    \end{pspicture}}
    \end{tabular}
  \end{center}
}


\end{document}

