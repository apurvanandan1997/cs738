\mode<presentation>
{
  \usetheme{umbc4}
  %\setbeamercovered{dynamic}
}
\usepackage[english]{babel}
% or whatever

\usepackage[latin1]{inputenc}
% or whatever

\usepackage{times}
\usepackage[T1]{fontenc}
% Or whatever. Note that the encoding and the font should match. If T1
% does not look nice, try deleting the line with the fontenc.
%--------------------------------------------------------------------------------------------
%--- Other packages
\usepackage{pstricks}
\usepackage{pst-node}
\usepackage{pst-rel-points}
\usepackage{xspace}

%%%%%% Declarations Defined in custom-foils
%\setfootline{\insertshortauthor, \insertshortinstitute \hfill   \insertshorttitle \hfill \insertframenumber/\inserttotalframenumber} 
\newcommand{\mypart}[1]{%
  \frame[plain]{\mbox{}\vfill \psshadowbox{\huge #1} \vfill\mbox{}}
}
\author[karkare]{Amey Karkare \\ \url{karkare@cse.iitk.ac.in}}
\date[]{\scalebox{0.3}{\includegraphics{iitklogo.epsi}}}%
\institute[CSE, IITK]{\url{http://www.cse.iitk.ac.in/~karkare/cs738}\\Department of CSE, IIT Kanpur}
\title[CS738]{CS738: Advanced Compiler Optimizations}
\subtitle[]{}


\subtitle[]{\ \\{\LARGE SSA Continued}\\}

\newcommand{\bbent}{{\it Entry}\xspace}
\newcommand{\bbext}{{\it Exit}\xspace}

\newcommand{\IN}{{\sf IN}\xspace}
\newcommand{\OUT}{{\sf OUT}\xspace}

\newcommand{\Pred}{{\sf PRED}\xspace}
\newcommand{\Succ}{{\sf SUCC}\xspace}

\newcommand{\Kill}{{\sf KILL}\xspace}
\newcommand{\Gen}{{\sf GEN}\xspace}

\newcommand{\cmt}[1]{{}}
\newcommand{\meet}{\ensuremath{\bigwedge}\xspace}
\newcommand{\join}{\ensuremath{\bigvee}\xspace}
\newcommand{\glb}{{\sf glb}\xspace}
\newcommand{\lub}{{\sf lub}\xspace}

%% Framed minipage %%%%%%%%%%%%%%%%%%%%%%%%%%%%%%%%%%%%
\newsavebox\TestBox
\newenvironment{frmminipage}[2][]
{\begin{lrbox}{\TestBox}\begin{minipage}[#1]{#2}}
{\end{minipage}\end{lrbox}\fbox{\usebox{\TestBox}}}


\newcommand{\mylabbox}[2]{{\small #1}\psframebox{#2}\phantom{\small #1}}

\newcommand{\tossa}{%
  \ensuremath{\stackrel{SSA\quad\ }{\scalebox{3}{$\Rightarrow$}}}}

\newcommand{\jstfy}{\mbox{\phantom{XX}}}

\newcommand{\dom}{\operatorname{dom}}
\newcommand{\sdom}{\operatorname{sdom}}
\newcommand{\idom}{\operatorname{idom}}

\newcommand{\DOM}{\operatorname{DOM}}
\newcommand{\DF}{\operatorname{DF}}

\begin{document}

\frame{\titlepage}

\frame{
  \frametitle{Agenda}
  \begin{itemize}
  \item Properties of SSA
  \item SSA to Executable
  \item SSA for Optimizations
  \end{itemize}
}

\frame{
  \frametitle{Complexity of Construction}
  \begin{itemize}[<+->]
  \item $R$ = $\operatorname{max}(N, E, A, M)$
  \item $N$: nodes, $E$: edges in flow graph
  \item $A$: number of assignments
  \item $M$: number of uses of variables
  \item Computation of DF: $O(R^2)$
  \item Computation of SSA: $O(R^3)$
  \item In practice, worst case is rare.
  \item Practical complexity: $O(R)$
  \end{itemize}
}

\frame{
  \frametitle{Linear Time Algorithm for $\phi$-functions}
  \begin{itemize}[<+->]
  \item By Sreedhar and Gao, in POPL'95
  \item Uses a new data structure called DJ-graph
  \item Linear time is achieved by careful ordering of nodes in the DJ-graph
  \item DF for a node is computed only once an reused later if required.
  \end{itemize}
}

\frame{
  \frametitle{Variants of SSA Form: Simple Example}
  \begin{center}
    \psset{unit=1mm}
    \begin{pspicture}(0,0)(60,67)
      %\psframe(0,0)(60,67)
      \putnode{n0}{origin}{10}{56}{\psframebox{
          \begin{tabular}{@{}c@{}}
            x = \ldots\\
            \ldots = x\\
            y = \ldots\\
            z = \ldots
          \end{tabular}
      }}
      \putnode{n1}{n0}{40}{0}{\psframebox{
          \begin{tabular}{@{}c@{}}
            x = \ldots\\
            \ldots = x\\
            y = \ldots\\
            z = \ldots
          \end{tabular}
      }}
      \putnode{n2}{n0}{0}{-20}{\psframebox{
          \begin{tabular}{@{}c@{}}
            \ldots = y
          \end{tabular}
      }}
      \putnode{n3}{n1}{0}{-20}{\psframebox{
          \begin{tabular}{@{}c@{}}
            \ldots = y
          \end{tabular}
      }}
      \putnode{n4}{n2}{20}{-25}{\psframebox{
          \begin{tabular}{@{}c@{}}
            \\
            \\
            \\
            \ldots = z
          \end{tabular}
      }}
      \ncline{->}{n0}{n2}
      \ncline{->}{n1}{n3}
      \ncline{->}{n2}{n4}
      \ncline{->}{n3}{n4}
      \putnode{lbl}{n4}{0}{-13}{Original Program}
    \end{pspicture}
  \end{center}
}

\frame{
  \frametitle{Variants of SSA Form: Simple Example}
  \begin{center}
    \psset{unit=1mm}
    \begin{pspicture}(0,0)(60,67)
      %\psframe(0,0)(60,67)
      \putnode{n0}{origin}{10}{56}{\psframebox{
          \begin{tabular}{@{}c@{}}
            x1 = \ldots\\
            \ldots = x1\\
            y1 = \ldots\\
            z1 = \ldots
          \end{tabular}
      }}
      \putnode{n1}{n0}{40}{0}{\psframebox{
          \begin{tabular}{@{}c@{}}
            x2 = \ldots\\
            \ldots = x2\\
            y2 = \ldots\\
            z2 = \ldots
          \end{tabular}
      }}
      \putnode{n2}{n0}{0}{-20}{\psframebox{
          \begin{tabular}{@{}c@{}}
            \ldots = y1
          \end{tabular}
      }}
      \putnode{n3}{n1}{0}{-20}{\psframebox{
          \begin{tabular}{@{}c@{}}
            \ldots = y2
          \end{tabular}
      }}
      \putnode{n4}{n2}{20}{-25}{\psframebox{
          \begin{tabular}{@{}l@{}}
            x3 = $\phi$(x1, x2)\\
            y3 = $\phi$(y1, y2)\\
            z3 = $\phi$(z1, z2)\\
            \ldots = z3
          \end{tabular}
      }}
      \ncline{->}{n0}{n2}
      \ncline{->}{n1}{n3}
      \ncline{->}{n2}{n4}
      \ncline{->}{n3}{n4}
      \putnode{lbl}{n4}{0}{-13}{Minimal SSA form}
    \end{pspicture}
  \end{center}
}

\frame{
  \frametitle{Variants of SSA Form}
  \begin{itemize}[<+->]
  \item Minimal SSA still contains extraneous $\phi$-functions 
    \begin{itemize}
    \item Inserts some $\phi$-functions where they are dead
    \item Would like to avoid inserting them
    \end{itemize}
  \item Pruned SSA
  \item Semi-Pruned SSA
  \end{itemize}
}

\frame{
  \frametitle{Pruned SSA}
  \begin{itemize}[<+->]
  \item Only insert $\phi$-functions where their value is live
  \item Inserts fewer $\phi$-functions 
  \item Costs more to do
  \item Requires global Live variable analysis
  \end{itemize}
}

\frame{
  \frametitle{Variants of SSA Form: Pruned SSA Example}
  \begin{center}
    \psset{unit=1mm}
    \begin{pspicture}(0,0)(60,67)
      %\psframe(0,0)(60,67)
      \putnode{n0}{origin}{10}{56}{\psframebox{
          \begin{tabular}{@{}c@{}}
            x1 = \ldots\\
            \ldots = x1\\
            y1 = \ldots\\
            z1 = \ldots
          \end{tabular}
      }}
      \putnode{n1}{n0}{40}{0}{\psframebox{
          \begin{tabular}{@{}c@{}}
            x2 = \ldots\\
            \ldots = x2\\
            y2 = \ldots\\
            z2 = \ldots
          \end{tabular}
      }}
      \putnode{n2}{n0}{0}{-20}{\psframebox{
          \begin{tabular}{@{}c@{}}
            \ldots = y1
          \end{tabular}
      }}
      \putnode{n3}{n1}{0}{-20}{\psframebox{
          \begin{tabular}{@{}c@{}}
            \ldots = y2
          \end{tabular}
      }}
      \putnode{n4}{n2}{20}{-25}{\psframebox{
          \begin{tabular}{@{}l@{}}
            \\
            \\
            z3 = $\phi$(z1, z2)\\
            \ldots = z3
          \end{tabular}
      }}
      \ncline{->}{n0}{n2}
      \ncline{->}{n1}{n3}
      \ncline{->}{n2}{n4}
      \ncline{->}{n3}{n4}
      %\putnode{lbl}{n4}{0}{-10}{Pruned SSA form}
    \end{pspicture}
  \end{center}
}

\frame{
  \frametitle{semi-pruned SSA Form}
  \begin{itemize}[<+->]
  \item Discard names used in only one block
  \item Total number of $\phi$-functions between minimal and pruned SSA
  \item Needs only local Live information
  \item Non-locals can be computed without iteration or elimination
  \end{itemize}
}

\frame{
  \frametitle{Variants of SSA Form: Semi-pruned SSA Example}
  \begin{center}
    \psset{unit=1mm}
    \begin{pspicture}(0,0)(60,67)
      %\psframe(0,0)(60,67)
      \putnode{n0}{origin}{10}{56}{\psframebox{
          \begin{tabular}{@{}c@{}}
            x1 = \ldots\\
            \ldots = x1\\
            y1 = \ldots\\
            z1 = \ldots
          \end{tabular}
      }}
      \putnode{n1}{n0}{40}{0}{\psframebox{
          \begin{tabular}{@{}c@{}}
            x2 = \ldots\\
            \ldots = x2\\
            y2 = \ldots\\
            z2 = \ldots
          \end{tabular}
      }}
      \putnode{n2}{n0}{0}{-20}{\psframebox{
          \begin{tabular}{@{}c@{}}
            \ldots = y1
          \end{tabular}
      }}
      \putnode{n3}{n1}{0}{-20}{\psframebox{
          \begin{tabular}{@{}c@{}}
            \ldots = y2
          \end{tabular}
      }}
      \putnode{n4}{n2}{20}{-25}{\psframebox{
          \begin{tabular}{@{}l@{}}
            \\
            y3 = $\phi$(y1, y2)\\
            z3 = $\phi$(z1, z2)\\
            \ldots = z3
          \end{tabular}
      }}
      \ncline{->}{n0}{n2}
      \ncline{->}{n1}{n3}
      \ncline{->}{n2}{n4}
      \ncline{->}{n3}{n4}
      %\putnode{lbl}{n4}{0}{-10}{Semi-pruned SSA form}
    \end{pspicture}
  \end{center}
}

\frame{
  \frametitle{Computing Non-locals}
  \tt
  \begin{tabbing}
    for\=each block B \{\\ \pause
    \>defined = \{\}\\ \pause
    \>for\=each instruction $v = x \operatorname{op} y$ \{\\ \pause
    \>\>if $x$\=\ not in defined \\ \pause
    \>\>\>non-locals = non-locals $\cup\ \{x\}$\\ \pause
    \>\>if $y$ not in defined\\ \pause 
    \>\>\>non-locals = non-locals $\cup\ \{y\}$\\ \pause
    \>\>defined = defined $\cup\ \{v\}$ \\
    \>\}\\
    \}
  \end{tabbing}
}

\frame{
  \frametitle{SSA to Executable}
  \begin{itemize}[<+->]
  \item At some point, we need executable code
    \begin{itemize}
    \item Need to fix up the $\phi$-function
    \end{itemize}
  \item Basic idea
    \begin{itemize}
    \item Insert copies in predecessors to mimick $\phi$-function 
    \item Simple algorithm
      \begin{itemize}
      \item Works in most cases, but \alert{not always}
      \end{itemize}
    \item Adds lots of copies
      \begin{itemize}
      \item Many of them will be optimized by later passes
      \end{itemize}
    \end{itemize}
  \end{itemize}
}

\frame{
  \frametitle{$\phi$-removal: Example}
  \begin{center}
    \psset{unit=1mm}
    \begin{pspicture}(0,0)(60,72)
      %\psframe(0,0)(60,72)
      \putnode{n0}{origin}{10}{72}{}
      \putnode{n2}{n0}{0}{0}{\psframebox{
          \begin{tabular}{@{}c@{}}
            x0 = \ldots \\
            \\
          \end{tabular}
      }}
      \putnode{n3}{n2}{40}{0}{\psframebox{
          \begin{tabular}{@{}c@{}}
            x1 = \ldots \\
            \\
          \end{tabular}
      }}
      \putnode{n4}{n2}{20}{-18}{\psframebox{
          \begin{tabular}{@{}l@{}}
            x2 = $\phi$(x0, x1)\\
            \ldots = x2
          \end{tabular}
      }}
      \ncline{->}{n2}{n4}
      \ncline{->}{n3}{n4}
      \onslide<2->{
        \putnode{lbl}{origin}{38}{40}{{\Huge$\Downarrow$} $\phi$-removal}
      \putnode{n0}{origin}{10}{26}{}
      \putnode{n2}{n0}{0}{0}{\psframebox{
          \begin{tabular}{@{}c@{}}
            x0 = \ldots \\
            x2 = x0 \\
          \end{tabular}
      }}
      \putnode{n3}{n2}{40}{0}{\psframebox{
          \begin{tabular}{@{}c@{}}
            x1 = \ldots \\
            x2 = x1\\
          \end{tabular}
      }}
      \putnode{n4}{n2}{20}{-18}{\psframebox{
          \begin{tabular}{@{}l@{}}
            \\ 
            \ldots = x2
          \end{tabular}
      }}
      \ncline{->}{n2}{n4}
      \ncline{->}{n3}{n4}}
    \end{pspicture}
  \end{center}
}

\frame{
  \frametitle{Lost Copy Problem}
  \begin{center}
  \begin{tabular}{ccc}
    \psset{unit=1mm}
    \begin{pspicture}(0,0)(33,50)
      %\psframe(0,0)(33,50)
      \putnode{n0}{origin}{15}{50}{\psframebox{
          \begin{tabular}{@{}l@{}}
            x = 1 \\
          \end{tabular}
      }}
      \putnode{n2}{n0}{0}{-20}{\psframebox{
          \begin{tabular}{@{}l@{}}
            y = x \\
            x = x + 1\\
          \end{tabular}
      }}
      \putnode{n3}{n2}{0}{-20}{\psframebox{
          \begin{tabular}{@{}c@{}}
            print y
          \end{tabular}
      }}
      \ncline{->}{n0}{n2}
      \ncline{->}{n2}{n3}
      \nccurve[ncurv=3,angleA=-45,angleB=45]{->}{n2}{n2}
    \end{pspicture} &
    \onslide<2->{\psset{unit=1mm}
    \begin{pspicture}(0,0)(33,50)
      %\psframe(0,0)(33,50)
      \putnode{n0}{origin}{15}{50}{\psframebox{
          \begin{tabular}{@{}l@{}}
            x1 = 1 \\
          \end{tabular}
      }}
      \putnode{n2}{n0}{0}{-20}{\psframebox{
          \begin{tabular}{@{}l@{}}
            x2 = $\phi$(x1, x3)\\
            x3 = x2 + 1\\
          \end{tabular}
      }}
      \putnode{n3}{n2}{0}{-20}{\psframebox{
          \begin{tabular}{@{}c@{}}
            print x2
          \end{tabular}
      }}
      \ncline{->}{n0}{n2}
      \ncline{->}{n2}{n3}
      \nccurve[ncurv=3,angleA=-45,angleB=45]{->}{n2}{n2}
    \end{pspicture}} &
    \onslide<3->{\psset{unit=1mm}
    \begin{pspicture}(0,0)(33,50)
      %\psframe(0,0)(33,50)
      \putnode{n0}{origin}{15}{50}{\psframebox{
          \begin{tabular}{@{}l@{}}
            x1 = 1 \\
            x2 = x1
          \end{tabular}
      }}
      \putnode{n2}{n0}{0}{-20}{\psframebox{
          \begin{tabular}{@{}l@{}}
            x3 = x2 + 1\\
            x2 = x3 \\
          \end{tabular}
      }}
      \putnode{n3}{n2}{0}{-20}{\psframebox{
          \begin{tabular}{@{}c@{}}
            print x2
          \end{tabular}
      }}
      \ncline{->}{n0}{n2}
      \ncline{->}{n2}{n3}
      \nccurve[ncurv=3,angleA=-45,angleB=45]{->}{n2}{n2}
    \end{pspicture}} \\
    Program &
    \onslide<2->{SSA from with} &
    \onslide<3->{After $\phi$-removel} \\
    & \onslide<2->{copy propagation} & \\
  \end{tabular}
  \end{center}
}

\frame{
  \frametitle{Lost Copy Problem: Solutions}
  \begin{center}
  \begin{tabular}{ccc}
    \psset{unit=1mm}
    \begin{pspicture}(0,0)(33,50)
      %\psframe(0,0)(33,50)
      \putnode{n0}{origin}{15}{50}{\psframebox{
          \begin{tabular}{@{}l@{}}
            x1 = 1 \\
            x2 = x1
          \end{tabular}
      }}
      \putnode{n2}{n0}{0}{-20}{\psframebox{
          \begin{tabular}{@{}l@{}}
            x3 = x2 + 1 \\
            t = x2 \\
            x2 = x3 \\
          \end{tabular}
      }}
      \putnode{n3}{n2}{0}{-20}{\psframebox{
          \begin{tabular}{@{}c@{}}
            print t
          \end{tabular}
      }}
      \ncline{->}{n0}{n2}
      \ncline{->}{n2}{n3}
      \nccurve[ncurv=3,angleA=-45,angleB=45]{->}{n2}{n2}
    \end{pspicture} & \rule{10mm}{0pt} &
    \onslide<2->{\psset{unit=1mm}
    \begin{pspicture}(0,0)(53,50)
      %\psframe(0,0)(53,50)
      \putnode{n0}{origin}{15}{50}{\psframebox{
          \begin{tabular}{@{}l@{}}
            x1 = 1 \\
            x2 = x1
          \end{tabular}
      }}
      \putnode{n2}{n0}{0}{-20}{\psframebox{
          \begin{tabular}{@{}l@{}}
            x3 = x2 + 1
          \end{tabular}
      }}
      \putnode{n3}{n2}{0}{-20}{\psframebox{
          \begin{tabular}{@{}c@{}}
            print x2
          \end{tabular}
      }}
      \putnode{n4}{n2}{25}{0}{\psframebox{
          \begin{tabular}{@{}c@{}}
            x2 = x3
          \end{tabular}
      }}
      \ncline{->}{n0}{n2}
      \ncline{->}{n2}{n3}
      \nccurve[angleA=-45,angleB=-135]{->}{n2}{n4}
      \nccurve[angleA=135,angleB=45]{->}{n4}{n2}
    \end{pspicture}} \\
    1. Use of Temporary &&
    \onslide<2->{2. Critical Edge Split} \\
  \end{tabular}
  \end{center}
}

\frame{
  \frametitle{Swap Problem}
  \begin{center}
  \begin{tabular}{ccc}
    \psset{unit=1mm}
    \begin{pspicture}(0,0)(33,50)
      %\psframe(0,0)(33,50)
      \putnode{n0}{origin}{15}{50}{\psframebox{
          \begin{tabular}{@{}l@{}}
            a = 1 \\
            b = 2
          \end{tabular}
      }}
      \putnode{n2}{n0}{0}{-20}{\psframebox{
          \begin{tabular}{@{}l@{}}
            x = a \\
            a = b \\
            b = x \\
          \end{tabular}
      }}
      \putnode{n3}{n2}{0}{-20}{\psframebox{
          \begin{tabular}{@{}c@{}}
            print b
          \end{tabular}
      }}
      \ncline{->}{n0}{n2}
      \ncline{->}{n2}{n3}
      \nccurve[ncurv=3,angleA=-45,angleB=45]{->}{n2}{n2}
    \end{pspicture} &
    \onslide<2->{\psset{unit=1mm}
    \begin{pspicture}(0,0)(33,50)
      %\psframe(0,0)(53,50)
      \putnode{n0}{origin}{15}{50}{\psframebox{
          \begin{tabular}{@{}l@{}}
            a1 = 1 \\
            b1 = 2
          \end{tabular}
      }}
      \putnode{n2}{n0}{0}{-20}{\psframebox{
          \begin{tabular}{@{}l@{}}
            a2 = $\phi$(a1, b2) \\
            b2 = $\phi$(b1, a2)
          \end{tabular}
      }}
      \putnode{n3}{n2}{0}{-20}{\psframebox{
          \begin{tabular}{@{}c@{}}
            print a2
          \end{tabular}
      }}
      \ncline{->}{n0}{n2}
      \ncline{->}{n2}{n3}
      \nccurve[ncurv=4,angleA=-45,angleB=45]{->}{n2}{n2}
    \end{pspicture}} &
    \onslide<3->{\psset{unit=1mm}
    \begin{pspicture}(0,0)(33,50)
      %\psframe(0,0)(53,50)
      \putnode{n0}{origin}{15}{50}{\psframebox{
          \begin{tabular}{@{}l@{}}
            a1 = 1 \\
            b1 = 2 \\
            a2 = a1 \\
            b2 = b1
          \end{tabular}
      }}
      \putnode{n2}{n0}{0}{-20}{\psframebox{
          \begin{tabular}{@{}l@{}}
            a2 = b2 \\
            b2 = a2
          \end{tabular}
      }}
      \putnode{n3}{n2}{0}{-20}{\psframebox{
          \begin{tabular}{@{}c@{}}
            print a2
          \end{tabular}
      }}
      \ncline{->}{n0}{n2}
      \ncline{->}{n2}{n3}
      \nccurve[ncurv=3,angleA=-45,angleB=45]{->}{n2}{n2}
    \end{pspicture}} \\
    Program &
    \onslide<2->{SSA form with} &
    \onslide<3->{After $\phi$-removel} \\
    & \onslide<2->{copy propagation} &
  \end{tabular}
  \end{center}
}
\frame{
  \frametitle{Swap Problem: Solution}
  \begin{itemize}[<+->]
  \item Fix requires compiler to detect and break dependency from
    output of one $\phi$-function to input of another
    $\phi$-function.
  \item May require temporary if cyclic dependency exists.
  \end{itemize}
}
\frame{
  \frametitle{SSA Form for Optimizations}
  \begin{itemize}[<+->]
  \item SSA form can improve and/or speed up many analyses and optimizations
    \begin{itemize}
    \item (Conditional) Constant propagation
    \item Dead code elimination
    \item Value numbering
    \item PRE
    \item Loop Invariant Code Motion
    \item Strength Reduction
    \end{itemize}
  \end{itemize}
}
\end{document}
