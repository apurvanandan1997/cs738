\mode<presentation>
{
  %\usetheme{umbc4}
  %\setbeamercovered{dynamic}
}
\usepackage[english]{babel}
% or whatever

\usepackage[latin1]{inputenc}
% or whatever

\usepackage{times}
\usepackage[T1]{fontenc}
\usepackage{colortbl}
% Or whatever. Note that the encoding and the font should match. If T1
% does not look nice, try deleting the line with the fontenc.
%--------------------------------------------------------------------------------------------
%--- Other packages
\usepackage{pstricks}
\usepackage{pst-node}
\usepackage{pst-rel-points}

\newcommand{\myexample}{
  \begin{center}
    \psset{unit=1mm}
    \begin{pspicture}(0,0)(50,-78)
      %\psframe(0,0)(50,-78)
      \putnode{entry}{origin}{25}{5}{\psframebox{ENTRY}}
      
      \putnode{b1}{entry}{0}{-11}{\psframebox{a = 2}}
      \putnode{b2}{b1}{0}{-13}{\psframebox{b = 3}}
      \putnode{b3}{b2}{0}{-13}{\psframebox{a < b}}
      \putnode{b4}{b3}{-13}{-13}{\psframebox{c1 = 4}}
      \putnode{b5}{b3}{10}{-13}{\psframebox{c2 = 5}}
      \putnode{b6}{b3}{0}{-26}{\psframebox{c3 = $\phi$(c1, c2)}}

      \putnode{exit}{entry}{0}{-74}{\psframebox{EXIT}}

      \ncline{->}{entry}{b1}
      \ncline{->}{b1}{b2}
      \ncline{->}{b2}{b3}
      \ncline{->}{b3}{b4} \Bput[0.1]{true}
      \ncline{->}{b3}{b5} \Aput[0.1]{false}
      \ncline{->}{b4}{b6}
      \ncline{->}{b5}{b6}
      \ncline{->}{b6}{exit}

      \nccurve[linestyle=dotted,dotsep=0.3]{->}{b1}{b3}
      \nccurve[linestyle=dotted,dotsep=0.3,angle=180]{->}{b2}{b3}
      \nccurve[linestyle=dotted,dotsep=0.3,angle=180]{->}{b4}{b6}
      \nccurve[linestyle=dotted,dotsep=0.3]{->}{b5}{b6}
    \end{pspicture}
  \end{center}
}

%%%%%% Declarations Defined in custom-foils
%\setfootline{\insertshortauthor, \insertshortinstitute \hfill   \insertshorttitle \hfill \insertframenumber/\inserttotalframenumber} 
\newcommand{\mypart}[1]{%
  \frame[plain]{\mbox{}\vfill \psshadowbox{\huge #1} \vfill\mbox{}}
}
\author[karkare]{Amey Karkare \\ \url{karkare@cse.iitk.ac.in}}
\date[]{\scalebox{0.3}{\includegraphics{iitklogo.epsi}}}%
\institute[CSE, IITK]{\url{http://www.cse.iitk.ac.in/~karkare/cs738}\\Department of CSE, IIT Kanpur}
\title[CS738]{CS738: Advanced Compiler Optimizations}
\subtitle[]{}


\subtitle[]{\ \\{\LARGE Welcome \& Introduction}\\}

\begin{document}

\frame{\titlepage}
\frame{
  \frametitle{About the Course}
  \begin{itemize}[<+->]
  \item \alert{Program Analysis}
  \item Analysis of a Program, by a Program, for a
    Program\footnote{\scriptsize ``Democracy is the government of the
      people, by the people, for the people'' - Abraham Lincoln}
    \begin{itemize}
    \item Of a Program -- User Program
    \item By a Program -- Analyzer (Compiler, Runtime)
    \item For a Program -- Optimizer, Verifier
    \end{itemize}
  \item Transforming user program based on the results of the
    analysis
  \end{itemize}
}

\frame{
  \frametitle{Expectations from You}
  \begin{itemize}[<+->]
  \item Basic Compiler Knowledge
  \item Write Code
  \item Willingness to understand and modify large code bases
  \item Read and present state-of-the-art research papers
  \end{itemize}
}

\frame{
  \frametitle{Your Expectations}
  \alert{\Huge ?}
  \alert{\Huge Share through the Google Form}

}

\frame{
  \frametitle{Quick Quizzes (QQs)}
  \begin{itemize}[<+->]
  \item There will be small quizzes (10-15 min duration) during the class.
  %\item These can be announced or un-announced (surprise quizzes).
  \item Always keep a pen and some loose papers handy.
  \end{itemize}
}

\frame{
  \newcommand{\compilerphase}{%
    \psset{unit=1mm}
    \begin{pspicture}(6.5,8)
      \psframe(-1,-5)(8,8)
    \end{pspicture}
  }
  \frametitle{QQ \#1 (Ungraded)}
  \begin{itemize}
  \item What are the various phases of a typical compiler? (5 minutes)

    \[\mbox{\tt file.c}
    \rightarrow \compilerphase
    \rightarrow \compilerphase
    \rightarrow \cdots
    \rightarrow \compilerphase
    \rightarrow \mbox{\tt file.asm}\]
    
  \end{itemize}
}

\frame{
  \frametitle{Assignments}
  \begin{itemize}[<+->]
  \item Short assignments to apply the lecture material.
  \item Assignments will have some written and some programming
    tasks.
  \item 4--5 Assignments for the semester 
  \end{itemize}
}

\frame{
  \frametitle{Using Program Analysis}
  \begin{itemize}[<+->]
  \item Compiler Code Optimizations
  \item Why are optimizations important?
  \item Why not write optimized code to begin with?
  \item Where do optimizations fit in the compiler flow?
  \end{itemize}
}

\frame{
  \frametitle{Code Optimization}
  \begin{itemize}[<+->]
  \item Machine Independent
    \begin{itemize}
    \item Remove redundancy introduced by the Programmer
    \item Remove redundancy not required by later phases of compiler
    \item Take advantage of algebraic properties of operators
    \end{itemize}
    \item Machine dependent
      \begin{itemize}
      \item Take advantage of the properties of target machine
      \end{itemize}
    \item \alert{Optimization must preserve the semantics of the
      original program!}
  \end{itemize}
}

\mypart{
  \begin{tabular}{c}
    Machine Independent \\ Optimizations
  \end{tabular}
}

\frame{
  \frametitle{Motivational Example}
  \tt\vspace*{-4mm}%
  \only<2>{\vspace*{-4mm}\psframe[fillstyle=solid, fillcolor=yellow](1,-6.6)(9.5,-2.7)} 
  \begin{tabbing}
void quicksort(int m, int n) \\
/* recursively sort a[m] through a[n] */ \\
\{\ \ \ \ \= \\
  \> int i, j; \\
  \> int v, x; \\
  \>   if(n <= m) return; \\
  \> i = m-1; j = n; v = a[n]; \\
  \> whil\=e (1) \{\\
    \> \>do i = i+1; while (a[i] < v);\\
    \> \>do j = j-1; while (a[j] > v);\\
    \> \>if (i > j) break;\\
    \> \>x = a[i]; a[i] = a[j]; a[j] = x;\\
  \>\}\\
  \>x = a[i]; a[i] = a[n]; a[n] = x;\\
  \>quicksort(m,j); quicksort(i+1,n);\\
\}   
  \end{tabbing}
}

\frame{\tt
  \begin{tabular}{@{}ll@{}}
    \begin{tabular}{@{}r@{\ }p{4.4cm}}
      ( 1) & i = m-1 \\
      ( 2) & j = n \\
      ( 3) & t1 = 4*n \\
      ( 4) & v = a[t1]\\
      ( 5) & i = i+1\\
      ( 6) & t2 = 4*i\\
      ( 7) & t3 = a[t2]\\
      ( 8) & if t3 < v goto\,(5)\\
      ( 9) & j = j-1\\
      (10) & t4 = 4*j\\
      (11) & t5 = a[t4]\\
      (12) & if t5 > v goto\,(9)\\
      (13) & if i >= j goto\,(23)     
    \end{tabular}
    &
    \begin{tabular}{@{}rp{3.1cm}}
      (14) & t6 = 4*i\\
      (15) & x = a[t6] \\
      (16) & t7 = 4*i\\
      (17) & t8 = 4*j\\
      (18) & t9 = a[t8]\\
      (19) & a[t7] = t9\\
      (20) & t10 = 4*j\\
      (21) & a[t10] = x\\
      (22) & goto\,(5) \\
      (23) & t11 = 4*i\\
      (24) & x = a[t11]\\
      (25) & t12 = 4*i\\
      (26) & t13 = 4*n\\
      (27) & t14 = a[t13]\\
      (28) & a[t12] = t14\\
      (29) & t15 = 4*n\\
      (30) & a[t15] = x
    \end{tabular}
  \end{tabular}
}

\frame{\tt
  \begin{tabular}{@{}ll@{}}
    \begin{tabular}{@{}r@{\ }p{4.4cm}}
      \rowcolor{cyan}
      ( 1) & i = m-1 \\
      ( 2) & j = n \\
      ( 3) & t1 = 4*n \\
      ( 4) & v = a[t1]\\
      \rowcolor{green}
      ( 5) & i = i+1\\
      ( 6) & t2 = 4*i\\
      ( 7) & t3 = a[t2]\\
      ( 8) & if t3 < v goto\,(5)\\
      \rowcolor{cyan}
      ( 9) & j = j-1\\
      (10) & t4 = 4*j\\
      (11) & t5 = a[t4]\\
      (12) & if t5 > v goto\,(9)\\
      \rowcolor{green}
      (13) & if i >= j goto\,(23)     
    \end{tabular}
    &
    \begin{tabular}{@{}rp{3.1cm}}
      \rowcolor{cyan}
      (14) & t6 = 4*i\\
      (15) & x = a[t6] \\
      (16) & t7 = 4*i\\
      (17) & t8 = 4*j\\
      (18) & t9 = a[t8]\\
      (19) & a[t7] = t9\\
      (20) & t10 = 4*j\\
      (21) & a[t10] = x\\
      (22) & goto\,(5) \\
      \rowcolor{green}
      (23) & t11 = 4*i\\
      (24) & x = a[t11]\\
      (25) & t12 = 4*i\\
      (26) & t13 = 4*n\\
      (27) & t14 = a[t13]\\
      (28) & a[t12] = t14\\
      (29) & t15 = 4*n\\
      (30) & a[t15] = x
    \end{tabular}
  \end{tabular}
}

\newcommand{\blklab}[1]{%
  \psframebox[fillstyle=solid,fillcolor=green,framesep=0.3]
             {\scriptsize #1}}

\newcommand{\blk}[1]{%
  \psframebox[fillstyle=solid,fillcolor=pink,framesep=0]
             {#1}}

\newcommand{\bca}{\scriptsize\tt
  \begin{tabular}[t]{@{\ }p{1.55cm}@{}}
    i = m-1 \\
    j = n \\
    t1 = 4*n \\
    v = a[t1]
  \end{tabular}
}
\newcommand{\bcb}{\scriptsize\tt
  \begin{tabular}[t]{@{\ }p{2.93cm}@{}}
    i = i+1\\
    t2 = 4*i\\
    t3 = a[t2]\\
    if t3 < v goto\,B2 
  \end{tabular}
}
\newcommand{\bcc}{\scriptsize\tt
  \begin{tabular}[t]{@{\ }p{2.95cm}@{}}
    j = j-1\\
    t4 = 4*j\\
    t5 = a[t4]\\
    if t5 > v goto\,B3\\
  \end{tabular}
}

\newcommand{\bcd}{\scriptsize\tt
  \begin{tabular}[t]{@{\ }p{3.15cm}@{}}
    if i >= j goto\,B6     
  \end{tabular}
}

\newcommand{\bce}{\scriptsize\tt
  \begin{tabular}[t]{@{\ }p{2.95cm}@{}}
    t6 = 4*i\\
    x = a[t6] \\
    t7 = 4*i\\
    t8 = 4*j\\
    t9 = a[t8]\\
    a[t7] = t9\\
    t10 = 4*j\\
    a[t10] = x\\
    goto\,B2 \\
  \end{tabular}
}
\newcommand{\bcf}{\scriptsize\tt
  \begin{tabular}[t]{@{\ }p{2.95cm}@{}}
    t11 = 4*i\\
    x = a[t11]\\
    t12 = 4*i\\
    t13 = 4*n\\
    t14 = a[t13]\\
    a[t12] = t14\\
    t15 = 4*n\\
    a[t15] = x
  \end{tabular}
}
\frame{
  \frametitle{Common Subexpression Elimination}
  \psset{unit=1mm}
  \begin{pspicture}(0,-20)(110,50)
    %\psframe(0,-20)(110,50)
    \putnode{bc1}{origin}{10}{38}{\blklab{B1}\blk{\bca}}
    \putnode{bc2}{bc1}{35}{0}{\blklab{B2}\blk{\bcb}}
    \putnode{bc3}{bc2}{45}{0}{\blklab{B3}\blk{\bcc}}
    \putnode{bc4}{bc3}{-35}{-16}{\blklab{B4}\blk{\bcd}}
    \putnode{bc5}{bc1}{20}{-43}{\blklab{B5}\blk{\bce}}
    \putnode{bc6}{bc5}{43}{2}{\blklab{B6}\blk{\bcf}}

    \ncline[nodesepB=-5]{->}{bc1}{bc2}
    \ncline[nodesepB=-5]{->}{bc2}{bc3}
    \ncline[nodesepA=-.3,angleA=-90]{->}{bc3}{bc4}
    \ncline[nodesepB=-0.1]{->}{bc4}{bc5}
    \ncline[nodesepB=-0.1]{->}{bc4}{bc6}

    \ncangles[angleA=0,angleB=90,armA=.15cm,offset=1]{->}{bc2}{bc2}
    \ncangles[angleA=0,angleB=90,armA=.15cm,offset=1]{->}{bc3}{bc3}
    \ncangle[angleA=90,angleB=180,armB=.15cm,offsetB=-2,
      offsetA=2,nodesepB=-5]{->}{bc5}{bc2}
  \end{pspicture}
}

\renewcommand{\bca}{\scriptsize\tt
  \begin{tabular}[t]{@{\ }p{1.55cm}@{}}
    i = m-1 \\
    j = n \\
    \cellcolor{blue!20} t1 = 4*n \\
    v = a[t1]
  \end{tabular}
}
\renewcommand{\bcb}{\scriptsize\tt
  \begin{tabular}[t]{@{\ }p{2.93cm}@{}}
    i = i+1\\
    \cellcolor{red!80} t2 = 4*i\\
    t3 = a[t2]\\
    if t3 < v goto\,B2 
  \end{tabular}
}
\renewcommand{\bcc}{\scriptsize\tt
  \begin{tabular}[t]{@{\ }p{2.95cm}@{}}
    j = j-1\\
    \cellcolor{green!50!black!30} t4 = 4*j\\
    t5 = a[t4]\\
    if t5 > v goto\,B3\\
  \end{tabular}
}

\renewcommand{\bcd}{\scriptsize\tt
  \begin{tabular}[t]{@{\ }p{3.15cm}@{}}
    if i >= j goto\,B6     
  \end{tabular}
}

\renewcommand{\bce}{\scriptsize\tt
  \begin{tabular}[t]{@{\ }p{2.95cm}@{}}
    \cellcolor{red!80} t6 = 4*i\\
    x = a[t6] \\
    \cellcolor{red!80} t7 = 4*i\\
    \cellcolor{green!50!black!30} t8 = 4*j\\
    t9 = a[t8]\\
    a[t7] = t9\\
    \cellcolor{green!50!black!30} t10 = 4*j\\
    a[t10] = x\\
    goto\,B2 \\
  \end{tabular}
}
\renewcommand{\bcf}{\scriptsize\tt
  \begin{tabular}[t]{@{\ }p{2.95cm}@{}}
    \cellcolor{red!80} t11 = 4*i\\
    x = a[t11]\\
    \cellcolor{red!80} t12 = 4*i\\
    \cellcolor{blue!20}t13 = 4*n\\
    t14 = a[t13]\\
    a[t12] = t14\\
    \cellcolor{blue!20} t15 = 4*n\\
    a[t15] = x
  \end{tabular}
}


\frame{
  \frametitle{Common Subexpression Elimination}
  \psset{unit=1mm}
  \begin{pspicture}(0,-20)(110,50)
    %\psframe(0,-20)(110,50)
    \putnode{bc1}{origin}{10}{38}{\blklab{B1}\blk{\bca}}
    \putnode{bc2}{bc1}{35}{0}{\blklab{B2}\blk{\bcb}}
    \putnode{bc3}{bc2}{45}{0}{\blklab{B3}\blk{\bcc}}
    \putnode{bc4}{bc3}{-35}{-16}{\blklab{B4}\blk{\bcd}}
    \putnode{bc5}{bc1}{20}{-43}{\blklab{B5}\blk{\bce}}
    \putnode{bc6}{bc5}{43}{2}{\blklab{B6}\blk{\bcf}}

    \ncline[nodesepB=-5]{->}{bc1}{bc2}
    \ncline[nodesepB=-5]{->}{bc2}{bc3}
    \ncline[nodesepA=-.3,angleA=-90]{->}{bc3}{bc4}
    \ncline[nodesepB=-0.1]{->}{bc4}{bc5}
    \ncline[nodesepB=-0.1]{->}{bc4}{bc6}

    \ncangles[angleA=0,angleB=90,armA=.15cm,offset=1]{->}{bc2}{bc2}
    \ncangles[angleA=0,angleB=90,armA=.15cm,offset=1]{->}{bc3}{bc3}
    \ncangle[angleA=90,angleB=180,armB=.15cm,offsetB=-2,
      offsetA=2,nodesepB=-5]
            {->}{bc5}{bc2}
  \end{pspicture}
}


\renewcommand{\bca}{\scriptsize\tt
  \begin{tabular}[t]{@{\ }p{1.55cm}@{}}
    i = m-1 \\
    j = n \\
    \cellcolor{blue!20} t1 = 4*n \\
    v = a[t1]
  \end{tabular}
}
\renewcommand{\bcb}{\scriptsize\tt
  \begin{tabular}[t]{@{\ }p{2.93cm}@{}}
    i = i+1\\
    \cellcolor{red!80} t2 = 4*i\\
    t3 = a[t2]\\
    if t3 < v goto\,B2 
  \end{tabular}
}
\renewcommand{\bcc}{\scriptsize\tt
  \begin{tabular}[t]{@{\ }p{2.95cm}@{}}
    j = j-1\\
    \cellcolor{green!50!black!30} t4 = 4*j\\
    t5 = a[t4]\\
    if t5 > v goto\,B3\\
  \end{tabular}
}

\renewcommand{\bcd}{\scriptsize\tt
  \begin{tabular}[t]{@{\ }p{3.15cm}@{}}
    if i >= j goto\,B6     
  \end{tabular}
}

\renewcommand{\bce}{\scriptsize\tt
  \begin{tabular}[t]{@{\ }p{2.95cm}@{}}
    \\
    x = a[t2] \\
    \\
    \\
    t9 = a[t4]\\
    a[t2] = t9\\
    \\
    a[t4] = x\\
    goto\,B2 \\
  \end{tabular}
}
\renewcommand{\bcf}{\scriptsize\tt
  \begin{tabular}[t]{@{\ }p{2.95cm}@{}}
    \\
    x = a[t2]\\
    \\
    \\
    t14 = a[t1]\\
    a[t2] = t14\\
    \\
    a[t1] = x
  \end{tabular}
}

\frame{
  \frametitle{Common Subexpression Elimination}
  \psset{unit=1mm}
  \begin{pspicture}(0,-20)(110,50)
    %\psframe(0,-20)(110,50)
    \putnode{bc1}{origin}{10}{38}{\blklab{B1}\blk{\bca}}
    \putnode{bc2}{bc1}{35}{0}{\blklab{B2}\blk{\bcb}}
    \putnode{bc3}{bc2}{45}{0}{\blklab{B3}\blk{\bcc}}
    \putnode{bc4}{bc3}{-35}{-16}{\blklab{B4}\blk{\bcd}}
    \putnode{bc5}{bc1}{20}{-43}{\blklab{B5}\blk{\bce}}
    \putnode{bc6}{bc5}{43}{2}{\blklab{B6}\blk{\bcf}}

    \ncline[nodesepB=-5]{->}{bc1}{bc2}
    \ncline[nodesepB=-5]{->}{bc2}{bc3}
    \ncline[nodesepA=-.3,angleA=-90]{->}{bc3}{bc4}
    \ncline[nodesepB=-0.1]{->}{bc4}{bc5}
    \ncline[nodesepB=-0.1]{->}{bc4}{bc6}

    \ncangles[angleA=0,angleB=90,armA=.15cm,offset=1]{->}{bc2}{bc2}
    \ncangles[angleA=0,angleB=90,armA=.15cm,offset=1]{->}{bc3}{bc3}
    \ncangle[angleA=90,angleB=180,armB=.15cm,offsetB=-2,
      offsetA=2,nodesepB=-5]
            {->}{bc5}{bc2}
  \end{pspicture}
}


\renewcommand{\bca}{\scriptsize\tt
  \begin{tabular}[t]{@{\ }p{1.55cm}@{}}
    i = m-1 \\
    j = n \\
    t1 = 4*n \\
    v = a[t1]
  \end{tabular}
}
\renewcommand{\bcb}{\scriptsize\tt
  \begin{tabular}[t]{@{\ }p{2.93cm}@{}}
    i = i+1\\
    t2 = 4*i\\
    \cellcolor{blue!20} t3 = a[t2]\\
    if t3 < v goto\,B2 
  \end{tabular}
}
\renewcommand{\bcc}{\scriptsize\tt
  \begin{tabular}[t]{@{\ }p{2.95cm}@{}}
    j = j-1\\
    t4 = 4*j\\
    \cellcolor{green!50!black!30} t5 = a[t4]\\
    if t5 > v goto\,B3\\
  \end{tabular}
}

\renewcommand{\bcd}{\scriptsize\tt
  \begin{tabular}[t]{@{\ }p{3.15cm}@{}}
    if i >= j goto\,B6     
  \end{tabular}
}

\renewcommand{\bce}{\scriptsize\tt
  \begin{tabular}[t]{@{\ }p{2.95cm}@{}}
    \\
    \cellcolor{blue!20} x = a[t2] \\
    \\
    \\
    \cellcolor{green!50!black!30} t9 = a[t4]\\
    a[t2] = t9\\
    \\
    a[t4] = x\\
    goto\,B2 \\
  \end{tabular}
}
\renewcommand{\bcf}{\scriptsize\tt
  \begin{tabular}[t]{@{\ }p{2.95cm}@{}}
    \\
    \cellcolor{blue!20} x = a[t2]\\
    \\
    \\
    t14 = a[t1]\\
    a[t2] = t14\\
    \\
    a[t1] = x
  \end{tabular}
}

\frame{
  \frametitle{Common Subexpression Elimination}
  \psset{unit=1mm}
  \begin{pspicture}(0,-20)(110,50)
    %\psframe(0,-20)(110,50)
    \putnode{bc1}{origin}{10}{38}{\blklab{B1}\blk{\bca}}
    \putnode{bc2}{bc1}{35}{0}{\blklab{B2}\blk{\bcb}}
    \putnode{bc3}{bc2}{45}{0}{\blklab{B3}\blk{\bcc}}
    \putnode{bc4}{bc3}{-35}{-16}{\blklab{B4}\blk{\bcd}}
    \putnode{bc5}{bc1}{20}{-43}{\blklab{B5}\blk{\bce}}
    \putnode{bc6}{bc5}{43}{2}{\blklab{B6}\blk{\bcf}}

    \ncline[nodesepB=-5]{->}{bc1}{bc2}
    \ncline[nodesepB=-5]{->}{bc2}{bc3}
    \ncline[nodesepA=-.3,angleA=-90]{->}{bc3}{bc4}
    \ncline[nodesepB=-0.1]{->}{bc4}{bc5}
    \ncline[nodesepB=-0.1]{->}{bc4}{bc6}

    \ncangles[angleA=0,angleB=90,armA=.15cm,offset=1]{->}{bc2}{bc2}
    \ncangles[angleA=0,angleB=90,armA=.15cm,offset=1]{->}{bc3}{bc3}
    \ncangle[angleA=90,angleB=180,armB=.15cm,offsetB=-2,
      offsetA=2,nodesepB=-5]
            {->}{bc5}{bc2}
  \end{pspicture}
}

\renewcommand{\bca}{\scriptsize\tt
  \begin{tabular}[t]{@{\ }p{1.55cm}@{}}
    i = m-1 \\
    j = n \\
    t1 = 4*n \\
    v = a[t1]
  \end{tabular}
}
\renewcommand{\bcb}{\scriptsize\tt
  \begin{tabular}[t]{@{\ }p{2.93cm}@{}}
    i = i+1\\
    t2 = 4*i\\
    \cellcolor{blue!20} t3 = a[t2]\\
    if t3 < v goto\,B2 
  \end{tabular}
}
\renewcommand{\bcc}{\scriptsize\tt
  \begin{tabular}[t]{@{\ }p{2.95cm}@{}}
    j = j-1\\
    t4 = 4*j\\
    \cellcolor{green!50!black!30} t5 = a[t4]\\
    if t5 > v goto\,B3\\
  \end{tabular}
}

\renewcommand{\bcd}{\scriptsize\tt
  \begin{tabular}[t]{@{\ }p{3.15cm}@{}}
    if i >= j goto\,B6     
  \end{tabular}
}

\renewcommand{\bce}{\scriptsize\tt
  \begin{tabular}[t]{@{\ }p{2.95cm}@{}}
    \\
    x = t3 \\
    \\
    \\
    \\
    a[t2] = t5\\
    \\
    a[t4] = x\\
    goto\,B2 \\
  \end{tabular}
}
\renewcommand{\bcf}{\scriptsize\tt
  \begin{tabular}[t]{@{\ }p{2.95cm}@{}}
    \\
    x = t3\\
    \\
    \\
    t14 = a[t1]\\
    a[t2] = t14\\
    \\
    a[t1] = x
  \end{tabular}
}

\frame{
  \frametitle{Common Subexpression Elimination}
  \psset{unit=1mm}
  \begin{pspicture}(0,-20)(110,50)
    %\psframe(0,-20)(110,50)
    \putnode{bc1}{origin}{10}{38}{\blklab{B1}\blk{\bca}}
    \putnode{bc2}{bc1}{35}{0}{\blklab{B2}\blk{\bcb}}
    \putnode{bc3}{bc2}{45}{0}{\blklab{B3}\blk{\bcc}}
    \putnode{bc4}{bc3}{-35}{-16}{\blklab{B4}\blk{\bcd}}
    \putnode{bc5}{bc1}{20}{-43}{\blklab{B5}\blk{\bce}}
    \putnode{bc6}{bc5}{43}{2}{\blklab{B6}\blk{\bcf}}

    \ncline[nodesepB=-5]{->}{bc1}{bc2}
    \ncline[nodesepB=-5]{->}{bc2}{bc3}
    \ncline[nodesepA=-.3,angleA=-90]{->}{bc3}{bc4}
    \ncline[nodesepB=-0.1]{->}{bc4}{bc5}
    \ncline[nodesepB=-0.1]{->}{bc4}{bc6}

    \ncangles[angleA=0,angleB=90,armA=.15cm,offset=1]{->}{bc2}{bc2}
    \ncangles[angleA=0,angleB=90,armA=.15cm,offset=1]{->}{bc3}{bc3}
    \ncangle[angleA=90,angleB=180,armB=.15cm,offsetB=-2,
      offsetA=2,nodesepB=-5]
            {->}{bc5}{bc2}
  \end{pspicture}
}


\renewcommand{\bca}{\scriptsize\tt
  \begin{tabular}[t]{@{\ }p{1.55cm}@{}}
    i = m-1 \\
    j = n \\
    t1 = 4*n \\
    \cellcolor{red} v = a[t1]
  \end{tabular}
}
\renewcommand{\bcb}{\scriptsize\tt
  \begin{tabular}[t]{@{\ }p{2.93cm}@{}}
    i = i+1\\
    t2 = 4*i\\
    t3 = a[t2]\\
    if t3 < v goto\,B2 
  \end{tabular}
}
\renewcommand{\bcc}{\scriptsize\tt
  \begin{tabular}[t]{@{\ }p{2.95cm}@{}}
    j = j-1\\
    t4 = 4*j\\
    t5 = a[t4]\\
    if t5 > v goto\,B3\\
  \end{tabular}
}

\renewcommand{\bcd}{\scriptsize\tt
  \begin{tabular}[t]{@{\ }p{3.15cm}@{}}
    if i >= j goto\,B6     
  \end{tabular}
}

\renewcommand{\bce}{\scriptsize\tt
  \begin{tabular}[t]{@{\ }p{2.95cm}@{}}
    \\
    x = t3 \\
    \\
    \\
    \\
    a[t2] = t5\\
    \\
    a[t4] = x\\
    goto\,B2 \\
  \end{tabular}
}
\renewcommand{\bcf}{\scriptsize\tt
  \begin{tabular}[t]{@{\ }p{2.95cm}@{}}
    \\
    x = t3\\
    \\
    \\
    \cellcolor{red} t14 = a[t1]\\
    a[t2] = t14\\
    \\
    a[t1] = x
  \end{tabular}
}

\frame{
  \frametitle{Common Subexpression Elimination}
  \psset{unit=1mm}
  \begin{pspicture}(0,-20)(110,50)
    %\psframe(0,-20)(110,50)
    \putnode{bc1}{origin}{10}{38}{\blklab{B1}\blk{\bca}}
    \putnode{bc2}{bc1}{35}{0}{\blklab{B2}\blk{\bcb}}
    \putnode{bc3}{bc2}{45}{0}{\blklab{B3}\blk{\bcc}}
    \putnode{bc4}{bc3}{-35}{-16}{\blklab{B4}\blk{\bcd}}
    \putnode{bc5}{bc1}{20}{-43}{\blklab{B5}\blk{\bce}}
    \putnode{bc6}{bc5}{43}{2}{\blklab{B6}\blk{\bcf}}

    \ncline[nodesepB=-5]{->}{bc1}{bc2}
    \ncline[nodesepB=-5]{->}{bc2}{bc3}
    \ncline[nodesepA=-.3,angleA=-90]{->}{bc3}{bc4}
    \ncline[nodesepB=-0.1]{->}{bc4}{bc5}
    \ncline[nodesepB=-0.1]{->}{bc4}{bc6}

    \ncangles[angleA=0,angleB=90,armA=.15cm,offset=1]{->}{bc2}{bc2}
    \ncangles[angleA=0,angleB=90,armA=.15cm,offset=1]{->}{bc3}{bc3}
    \ncangle[angleA=90,angleB=180,armB=.15cm,offsetB=-2,
      offsetA=2,nodesepB=-5]
            {->}{bc5}{bc2}
     \putnode{a2}{origin}{55}{55}{\psovalbox[fillstyle=solid,fillcolor=blue!20]{\alert{Did we miss one expression?}}}
            \pause
     \putnode{al}{origin}{55}{-17}{\psovalbox[fillstyle=solid,fillcolor=blue!20]{\begin{tabular}{c}Elimination not safe as a[] is
       modified on path \\B1$\rightarrow$B2$\rightarrow$B3$\rightarrow$B4$\rightarrow${\red B5}$\rightarrow$B2$\rightarrow$B3$\rightarrow$B4$\rightarrow$B6
         \end{tabular}
       }}
  \end{pspicture}
}


\renewcommand{\bca}{\scriptsize\tt
  \begin{tabular}[t]{@{\ }p{1.55cm}@{}}
    i = m-1 \\
    j = n \\
    t1 = 4*n \\
    v = a[t1]
  \end{tabular}
}
\renewcommand{\bcb}{\scriptsize\tt
  \begin{tabular}[t]{@{\ }p{2.93cm}@{}}
    i = i+1\\
    t2 = 4*i\\
    t3 = a[t2]\\
    if t3 < v goto\,B2 
  \end{tabular}
}
\renewcommand{\bcc}{\scriptsize\tt
  \begin{tabular}[t]{@{\ }p{2.95cm}@{}}
    j = j-1\\
    t4 = 4*j\\
    t5 = a[t4]\\
    if t5 > v goto\,B3\\
  \end{tabular}
}

\renewcommand{\bcd}{\scriptsize\tt
  \begin{tabular}[t]{@{\ }p{3.15cm}@{}}
    if i >= j goto\,B6     
  \end{tabular}
}

\renewcommand{\bce}{\scriptsize\tt
  \begin{tabular}[t]{@{\ }p{2.95cm}@{}}
    \\
    \cellcolor{red} x = t3 \\
    \\
    \\
    \\
    a[t2] = t5\\
    \\
    \cellcolor{red} a[t4] = x\\
    goto\,B2 \\
  \end{tabular}
}
\renewcommand{\bcf}{\scriptsize\tt
  \begin{tabular}[t]{@{\ }p{2.95cm}@{}}
    \\
    \cellcolor{red} x = t3\\
    \\
    \\
    t14 = a[t1]\\
    a[t2] = t14\\
    \\
    \cellcolor{red} a[t1] = x
  \end{tabular}
}

\frame{
  \frametitle{Copy Propagation}
  \psset{unit=1mm}
  \begin{pspicture}(0,-20)(110,50)
    %\psframe(0,-20)(110,50)
    \putnode{bc1}{origin}{10}{38}{\blklab{B1}\blk{\bca}}
    \putnode{bc2}{bc1}{35}{0}{\blklab{B2}\blk{\bcb}}
    \putnode{bc3}{bc2}{45}{0}{\blklab{B3}\blk{\bcc}}
    \putnode{bc4}{bc3}{-35}{-16}{\blklab{B4}\blk{\bcd}}
    \putnode{bc5}{bc1}{20}{-43}{\blklab{B5}\blk{\bce}}
    \putnode{bc6}{bc5}{43}{2}{\blklab{B6}\blk{\bcf}}

    \ncline[nodesepB=-5]{->}{bc1}{bc2}
    \ncline[nodesepB=-5]{->}{bc2}{bc3}
    \ncline[nodesepA=-.3,angleA=-90]{->}{bc3}{bc4}
    \ncline[nodesepB=-0.1]{->}{bc4}{bc5}
    \ncline[nodesepB=-0.1]{->}{bc4}{bc6}

    \ncangles[angleA=0,angleB=90,armA=.15cm,offset=1]{->}{bc2}{bc2}
    \ncangles[angleA=0,angleB=90,armA=.15cm,offset=1]{->}{bc3}{bc3}
    \ncangle[angleA=90,angleB=180,armB=.15cm,offsetB=-2,
      offsetA=2,nodesepB=-5]
            {->}{bc5}{bc2}
  \end{pspicture}
}


\renewcommand{\bca}{\scriptsize\tt
  \begin{tabular}[t]{@{\ }p{1.55cm}@{}}
    i = m-1 \\
    j = n \\
    t1 = 4*n \\
    v = a[t1]
  \end{tabular}
}
\renewcommand{\bcb}{\scriptsize\tt
  \begin{tabular}[t]{@{\ }p{2.93cm}@{}}
    i = i+1\\
    t2 = 4*i\\
    t3 = a[t2]\\
    if t3 < v goto\,B2 
  \end{tabular}
}
\renewcommand{\bcc}{\scriptsize\tt
  \begin{tabular}[t]{@{\ }p{2.95cm}@{}}
    j = j-1\\
    t4 = 4*j\\
    t5 = a[t4]\\
    if t5 > v goto\,B3\\
  \end{tabular}
}

\renewcommand{\bcd}{\scriptsize\tt
  \begin{tabular}[t]{@{\ }p{3.15cm}@{}}
    if i >= j goto\,B6     
  \end{tabular}
}

\renewcommand{\bce}{\scriptsize\tt
  \begin{tabular}[t]{@{\ }p{2.95cm}@{}}
    \\
    \cellcolor{red} x = t3 \\
    \\
    \\
    \\
    a[t2] = t5\\
    \\
    \cellcolor{red} a[t4] = t3\\
    goto\,B2 \\
  \end{tabular}
}
\renewcommand{\bcf}{\scriptsize\tt
  \begin{tabular}[t]{@{\ }p{2.95cm}@{}}
    \\
    \cellcolor{red} x = t3\\
    \\
    \\
    t14 = a[t1]\\
    a[t2] = t14\\
    \\
    \cellcolor{red} a[t1] = t3
  \end{tabular}
}

\frame{
  \frametitle{Copy Propagation}
  \psset{unit=1mm}
  \begin{pspicture}(0,-20)(110,50)
    %\psframe(0,-20)(110,50)
    \putnode{bc1}{origin}{10}{38}{\blklab{B1}\blk{\bca}}
    \putnode{bc2}{bc1}{35}{0}{\blklab{B2}\blk{\bcb}}
    \putnode{bc3}{bc2}{45}{0}{\blklab{B3}\blk{\bcc}}
    \putnode{bc4}{bc3}{-35}{-16}{\blklab{B4}\blk{\bcd}}
    \putnode{bc5}{bc1}{20}{-43}{\blklab{B5}\blk{\bce}}
    \putnode{bc6}{bc5}{43}{2}{\blklab{B6}\blk{\bcf}}

    \ncline[nodesepB=-5]{->}{bc1}{bc2}
    \ncline[nodesepB=-5]{->}{bc2}{bc3}
    \ncline[nodesepA=-.3,angleA=-90]{->}{bc3}{bc4}
    \ncline[nodesepB=-0.1]{->}{bc4}{bc5}
    \ncline[nodesepB=-0.1]{->}{bc4}{bc6}

    \ncangles[angleA=0,angleB=90,armA=.15cm,offset=1]{->}{bc2}{bc2}
    \ncangles[angleA=0,angleB=90,armA=.15cm,offset=1]{->}{bc3}{bc3}
    \ncangle[angleA=90,angleB=180,armB=.15cm,offsetB=-2,
      offsetA=2,nodesepB=-5]
            {->}{bc5}{bc2}
            \pause
            \putnode{al}{origin}{55}{-17}{\psovalbox[fillstyle=solid,fillcolor=blue!20]{\begin{tabular}{c}Created Dead Assignments \\
                  Apply Dead Code Elimination
         \end{tabular}
       }}
  \end{pspicture}
}


\renewcommand{\bca}{\scriptsize\tt
  \begin{tabular}[t]{@{\ }p{1.55cm}@{}}
    i = m-1 \\
    j = n \\
    t1 = 4*n \\
    v = a[t1]
  \end{tabular}
}
\renewcommand{\bcb}{\scriptsize\tt
  \begin{tabular}[t]{@{\ }p{2.93cm}@{}}
    i = i+1\\
    t2 = 4*i\\
    t3 = a[t2]\\
    if t3 < v goto\,B2 
  \end{tabular}
}
\renewcommand{\bcc}{\scriptsize\tt
  \begin{tabular}[t]{@{\ }p{2.95cm}@{}}
    j = j-1\\
    t4 = 4*j\\
    t5 = a[t4]\\
    if t5 > v goto\,B3\\
  \end{tabular}
}

\renewcommand{\bcd}{\scriptsize\tt
  \begin{tabular}[t]{@{\ }p{3.15cm}@{}}
    if i >= j goto\,B6     
  \end{tabular}
}

\renewcommand{\bce}{\scriptsize\tt
  \begin{tabular}[t]{@{\ }p{2.95cm}@{}}
    \\
    \\
    \\
    \\
    \\
    a[t2] = t5\\
    \\
    a[t4] = t3\\
    goto\,B2 \\
  \end{tabular}
}
\renewcommand{\bcf}{\scriptsize\tt
  \begin{tabular}[t]{@{\ }p{2.95cm}@{}}
    \\
    \\
    \\
    \\
    t14 = a[t1]\\
    a[t2] = t14\\
    \\
    a[t1] = t3
  \end{tabular}
}

\frame{
  \frametitle{Copy Propagation}
  \psset{unit=1mm}
  \begin{pspicture}(0,-20)(110,50)
    %\psframe(0,-20)(110,50)
    \putnode{bc1}{origin}{10}{38}{\blklab{B1}\blk{\bca}}
    \putnode{bc2}{bc1}{35}{0}{\blklab{B2}\blk{\bcb}}
    \putnode{bc3}{bc2}{45}{0}{\blklab{B3}\blk{\bcc}}
    \putnode{bc4}{bc3}{-35}{-16}{\blklab{B4}\blk{\bcd}}
    \putnode{bc5}{bc1}{20}{-43}{\blklab{B5}\blk{\bce}}
    \putnode{bc6}{bc5}{43}{2}{\blklab{B6}\blk{\bcf}}

    \ncline[nodesepB=-5]{->}{bc1}{bc2}
    \ncline[nodesepB=-5]{->}{bc2}{bc3}
    \ncline[nodesepA=-.3,angleA=-90]{->}{bc3}{bc4}
    \ncline[nodesepB=-0.1]{->}{bc4}{bc5}
    \ncline[nodesepB=-0.1]{->}{bc4}{bc6}

    \ncangles[angleA=0,angleB=90,armA=.15cm,offset=1]{->}{bc2}{bc2}
    \ncangles[angleA=0,angleB=90,armA=.15cm,offset=1]{->}{bc3}{bc3}
    \ncangle[angleA=90,angleB=180,armB=.15cm,offsetB=-2,
      offsetA=2,nodesepB=-5]
            {->}{bc5}{bc2}
  \end{pspicture}
}


\renewcommand{\bca}{\scriptsize\tt
  \begin{tabular}[t]{@{\ }p{1.55cm}@{}}
    i = m-1 \\
    j = n \\
    t1 = 4*n \\
    v = a[t1]
  \end{tabular}
}
\renewcommand{\bcb}{\scriptsize\tt
  \begin{tabular}[t]{@{\ }p{2.93cm}@{}}
    \cellcolor{red!80}i = i+1\\
    \cellcolor{red!80}t2 = 4*i\\
    t3 = a[t2]\\
    if t3 < v goto\,B2 
  \end{tabular}
}
\renewcommand{\bcc}{\scriptsize\tt
  \begin{tabular}[t]{@{\ }p{2.95cm}@{}}
    \cellcolor{blue!30}j = j-1\\
    \cellcolor{blue!30}t4 = 4*j\\
    t5 = a[t4]\\
    if t5 > v goto\,B3\\
  \end{tabular}
}

\renewcommand{\bcd}{\scriptsize\tt
  \begin{tabular}[t]{@{\ }p{3.15cm}@{}}
    if i >= j goto\,B6     
  \end{tabular}
}

\renewcommand{\bce}{\scriptsize\tt
  \begin{tabular}[t]{@{\ }p{2.95cm}@{}}
    \\
    \\
    \\
    \\
    \\
    a[t2] = t5\\
    \\
    a[t4] = t3\\
    goto\,B2 \\
  \end{tabular}
}
\renewcommand{\bcf}{\scriptsize\tt
  \begin{tabular}[t]{@{\ }p{2.95cm}@{}}
    \\
    \\
    \\
    \\
    t14 = a[t1]\\
    a[t2] = t14\\
    \\
    a[t1] = t3
  \end{tabular}
}

\frame{
  \frametitle{Strength Reduction}
  \psset{unit=1mm}
  \begin{pspicture}(0,-20)(110,50)
    %\psframe(0,-20)(110,50)
    \putnode{bc1}{origin}{10}{38}{\blklab{B1}\blk{\bca}}
    \putnode{bc2}{bc1}{35}{0}{\blklab{B2}\blk{\bcb}}
    \putnode{bc3}{bc2}{45}{0}{\blklab{B3}\blk{\bcc}}
    \putnode{bc4}{bc3}{-35}{-16}{\blklab{B4}\blk{\bcd}}
    \putnode{bc5}{bc1}{20}{-43}{\blklab{B5}\blk{\bce}}
    \putnode{bc6}{bc5}{43}{2}{\blklab{B6}\blk{\bcf}}

    \ncline[nodesepB=-5]{->}{bc1}{bc2}
    \ncline[nodesepB=-5]{->}{bc2}{bc3}
    \ncline[nodesepA=-.3,angleA=-90]{->}{bc3}{bc4}
    \ncline[nodesepB=-0.1]{->}{bc4}{bc5}
    \ncline[nodesepB=-0.1]{->}{bc4}{bc6}

    \ncangles[angleA=0,angleB=90,armA=.15cm,offset=1]{->}{bc2}{bc2}
    \ncangles[angleA=0,angleB=90,armA=.15cm,offset=1]{->}{bc3}{bc3}
    \ncangle[angleA=90,angleB=180,armB=.15cm,offsetB=-2,
      offsetA=2,nodesepB=-5]
            {->}{bc5}{bc2}
  \end{pspicture}
}


\renewcommand{\bca}{\scriptsize\tt
  \begin{tabular}[t]{@{\ }p{1.55cm}@{}}
    i = m-1 \\
    j = n \\
    t1 = 4*n \\
    v = a[t1] \\
    \cellcolor{red!80} t2 = 4*i \\
    \cellcolor{blue!30} t4 = 4*j
  \end{tabular}
}
\renewcommand{\bcb}{\scriptsize\tt
  \begin{tabular}[t]{@{\ }p{2.93cm}@{}}
    \cellcolor{red!80}i = i+1\\
    \cellcolor{red!80}t2 = t2+4\\
    t3 = a[t2]\\
    if t3 < v goto\,B2 
  \end{tabular}
}
\renewcommand{\bcc}{\scriptsize\tt
  \begin{tabular}[t]{@{\ }p{2.95cm}@{}}
    \cellcolor{blue!30}j = j-1\\
    \cellcolor{blue!30}t4 = t4-4\\
    t5 = a[t4]\\
    if t5 > v goto\,B3\\
  \end{tabular}
}

\renewcommand{\bcd}{\scriptsize\tt
  \begin{tabular}[t]{@{\ }p{3.15cm}@{}}
    if i >= j goto\,B6     
  \end{tabular}
}

\renewcommand{\bce}{\scriptsize\tt
  \begin{tabular}[t]{@{\ }p{2.95cm}@{}}
    \\
    \\
    \\
    \\
    \\
    a[t2] = t5\\
    \\
    a[t4] = t3\\
    goto\,B2 \\
  \end{tabular}
}
\renewcommand{\bcf}{\scriptsize\tt
  \begin{tabular}[t]{@{\ }p{2.95cm}@{}}
    \\
    \\
    \\
    \\
    t14 = a[t1]\\
    a[t2] = t14\\
    \\
    a[t1] = t3
  \end{tabular}
}

\frame{
  \frametitle{Strength Reduction}
  \psset{unit=1mm}
  \begin{pspicture}(0,-20)(110,50)
    %\psframe(0,-20)(110,50)
    \putnode{bc1}{origin}{10}{38}{\blklab{B1}\blk{\bca}}
    \putnode{bc2}{bc1}{35}{0}{\blklab{B2}\blk{\bcb}}
    \putnode{bc3}{bc2}{45}{0}{\blklab{B3}\blk{\bcc}}
    \putnode{bc4}{bc3}{-35}{-16}{\blklab{B4}\blk{\bcd}}
    \putnode{bc5}{bc1}{20}{-43}{\blklab{B5}\blk{\bce}}
    \putnode{bc6}{bc5}{43}{2}{\blklab{B6}\blk{\bcf}}

    \ncline[nodesepB=-5]{->}{bc1}{bc2}
    \ncline[nodesepB=-5]{->}{bc2}{bc3}
    \ncline[nodesepA=-.3,angleA=-90]{->}{bc3}{bc4}
    \ncline[nodesepB=-0.1]{->}{bc4}{bc5}
    \ncline[nodesepB=-0.1]{->}{bc4}{bc6}

    \ncangles[angleA=0,angleB=90,armA=.15cm,offset=1]{->}{bc2}{bc2}
    \ncangles[angleA=0,angleB=90,armA=.15cm,offset=1]{->}{bc3}{bc3}
    \ncangle[angleA=90,angleB=180,armB=.15cm,offsetB=-2,
      offsetA=2,nodesepB=-5]
            {->}{bc5}{bc2}
  \end{pspicture}
}


\renewcommand{\bca}{\scriptsize\tt
  \begin{tabular}[t]{@{\ }p{1.55cm}@{}}
    i = m-1 \\
    j = n \\
    t1 = 4*n \\
    v = a[t1] \\
    t2 = 4*i \\
    t4 = 4*j
  \end{tabular}
}
\renewcommand{\bcb}{\scriptsize\tt
  \begin{tabular}[t]{@{\ }p{2.93cm}@{}}
    \cellcolor{red!80}i = i+1\\
    \cellcolor{red!80}t2 = t2+4\\
    t3 = a[t2]\\
    if t3 < v goto\,B2 
  \end{tabular}
}
\renewcommand{\bcc}{\scriptsize\tt
  \begin{tabular}[t]{@{\ }p{2.95cm}@{}}
    \cellcolor{red!80}j = j-1\\
    \cellcolor{red!80}t4 = t4-4\\
    t5 = a[t4]\\
    if t5 > v goto\,B3\\
  \end{tabular}
}

\renewcommand{\bcd}{\scriptsize\tt
  \begin{tabular}[t]{@{\ }p{3.15cm}@{}}
    \cellcolor{red!80} if i >= j goto\,B6     
  \end{tabular}
}

\renewcommand{\bce}{\scriptsize\tt
  \begin{tabular}[t]{@{\ }p{2.95cm}@{}}
    \\
    \\
    \\
    \\
    \\
    a[t2] = t5\\
    \\
    a[t4] = t3\\
    goto\,B2 \\
  \end{tabular}
}
\renewcommand{\bcf}{\scriptsize\tt
  \begin{tabular}[t]{@{\ }p{2.95cm}@{}}
    \\
    \\
    \\
    \\
    t14 = a[t1]\\
    a[t2] = t14\\
    \\
    a[t1] = t3
  \end{tabular}
}

\frame{
  \frametitle{Induction Variable Elimination}
  \psset{unit=1mm}
  \begin{pspicture}(0,-20)(110,50)
    %\psframe(0,-20)(110,50)
    \putnode{bc1}{origin}{10}{38}{\blklab{B1}\blk{\bca}}
    \putnode{bc2}{bc1}{35}{0}{\blklab{B2}\blk{\bcb}}
    \putnode{bc3}{bc2}{45}{0}{\blklab{B3}\blk{\bcc}}
    \putnode{bc4}{bc3}{-35}{-16}{\blklab{B4}\blk{\bcd}}
    \putnode{bc5}{bc1}{20}{-43}{\blklab{B5}\blk{\bce}}
    \putnode{bc6}{bc5}{43}{2}{\blklab{B6}\blk{\bcf}}

    \ncline[nodesepB=-5]{->}{bc1}{bc2}
    \ncline[nodesepB=-5]{->}{bc2}{bc3}
    \ncline[nodesepA=-.3,angleA=-90]{->}{bc3}{bc4}
    \ncline[nodesepB=-0.1]{->}{bc4}{bc5}
    \ncline[nodesepB=-0.1]{->}{bc4}{bc6}

    \ncangles[angleA=0,angleB=90,armA=.15cm,offset=1]{->}{bc2}{bc2}
    \ncangles[angleA=0,angleB=90,armA=.15cm,offset=1]{->}{bc3}{bc3}
    \ncangle[angleA=90,angleB=180,armB=.15cm,offsetB=-2,
      offsetA=2,nodesepB=-5]
            {->}{bc5}{bc2}
  \end{pspicture}
}


\renewcommand{\bca}{\scriptsize\tt
  \begin{tabular}[t]{@{\ }p{1.55cm}@{}}
    i = m-1 \\
    j = n \\
    t1 = 4*n \\
    v = a[t1] \\
    t2 = 4*i \\
    t4 = 4*j
  \end{tabular}
}
\renewcommand{\bcb}{\scriptsize\tt
  \begin{tabular}[t]{@{\ }p{2.93cm}@{}}
    \cellcolor{red!80}i = i+1\\
    \cellcolor{red!80}t2 = t2+4\\
    t3 = a[t2]\\
    if t3 < v goto\,B2 
  \end{tabular}
}
\renewcommand{\bcc}{\scriptsize\tt
  \begin{tabular}[t]{@{\ }p{2.95cm}@{}}
    \cellcolor{red!80}j = j-1\\
    \cellcolor{red!80}t4 = t4-4\\
    t5 = a[t4]\\
    if t5 > v goto\,B3\\
  \end{tabular}
}

\renewcommand{\bcd}{\scriptsize\tt
  \begin{tabular}[t]{@{\ }p{3.15cm}@{}}
    \cellcolor{red!80} if t2 >= t4 goto\,B6     
  \end{tabular}
}

\renewcommand{\bce}{\scriptsize\tt
  \begin{tabular}[t]{@{\ }p{2.95cm}@{}}
    \\
    \\
    \\
    \\
    \\
    a[t2] = t5\\
    \\
    a[t4] = t3\\
    goto\,B2 \\
  \end{tabular}
}
\renewcommand{\bcf}{\scriptsize\tt
  \begin{tabular}[t]{@{\ }p{2.95cm}@{}}
    \\
    \\
    \\
    \\
    t14 = a[t1]\\
    a[t2] = t14\\
    \\
    a[t1] = t3
  \end{tabular}
}

\frame{
  \frametitle{Induction Variable Elimination}
  \psset{unit=1mm}
  \begin{pspicture}(0,-20)(110,50)
    %\psframe(0,-20)(110,50)
    \putnode{bc1}{origin}{10}{38}{\blklab{B1}\blk{\bca}}
    \putnode{bc2}{bc1}{35}{0}{\blklab{B2}\blk{\bcb}}
    \putnode{bc3}{bc2}{45}{0}{\blklab{B3}\blk{\bcc}}
    \putnode{bc4}{bc3}{-35}{-16}{\blklab{B4}\blk{\bcd}}
    \putnode{bc5}{bc1}{20}{-43}{\blklab{B5}\blk{\bce}}
    \putnode{bc6}{bc5}{43}{2}{\blklab{B6}\blk{\bcf}}

    \ncline[nodesepB=-5]{->}{bc1}{bc2}
    \ncline[nodesepB=-5]{->}{bc2}{bc3}
    \ncline[nodesepA=-.3,angleA=-90]{->}{bc3}{bc4}
    \ncline[nodesepB=-0.1]{->}{bc4}{bc5}
    \ncline[nodesepB=-0.1]{->}{bc4}{bc6}

    \ncangles[angleA=0,angleB=90,armA=.15cm,offset=1]{->}{bc2}{bc2}
    \ncangles[angleA=0,angleB=90,armA=.15cm,offset=1]{->}{bc3}{bc3}
    \ncangle[angleA=90,angleB=180,armB=.15cm,offsetB=-2,
      offsetA=2,nodesepB=-5]
            {->}{bc5}{bc2}
  \end{pspicture}
}


\renewcommand{\bca}{\scriptsize\tt
  \begin{tabular}[t]{@{\ }p{1.55cm}@{}}
    i = m-1 \\
    j = n \\
    t1 = 4*n \\
    v = a[t1] \\
    t2 = 4*i \\
    t4 = 4*j
  \end{tabular}
}
\renewcommand{\bcb}{\scriptsize\tt
  \begin{tabular}[t]{@{\ }p{2.93cm}@{}}
    \\
    t2 = t2+4\\
    t3 = a[t2]\\
    if t3 < v goto\,B2 
  \end{tabular}
}
\renewcommand{\bcc}{\scriptsize\tt
  \begin{tabular}[t]{@{\ }p{2.95cm}@{}}
    \\
    t4 = t4-4\\
    t5 = a[t4]\\
    if t5 > v goto\,B3\\
  \end{tabular}
}

\renewcommand{\bcd}{\scriptsize\tt
  \begin{tabular}[t]{@{\ }p{3.15cm}@{}}
    if t2 >= t4 goto\,B6     
  \end{tabular}
}

\renewcommand{\bce}{\scriptsize\tt
  \begin{tabular}[t]{@{\ }p{2.95cm}@{}}
    \\
    \\
    \\
    \\
    \\
    a[t2] = t5\\
    \\
    a[t4] = t3\\
    goto\,B2 \\
  \end{tabular}
}
\renewcommand{\bcf}{\scriptsize\tt
  \begin{tabular}[t]{@{\ }p{2.95cm}@{}}
    \\
    \\
    \\
    \\
    t14 = a[t1]\\
    a[t2] = t14\\
    \\
    a[t1] = t3
  \end{tabular}
}

\frame{
  \frametitle{Dead Code Elimination (Again!)}
  \psset{unit=1mm}
  \begin{pspicture}(0,-20)(110,50)
    %\psframe(0,-20)(110,50)
    \putnode{bc1}{origin}{10}{38}{\blklab{B1}\blk{\bca}}
    \putnode{bc2}{bc1}{35}{0}{\blklab{B2}\blk{\bcb}}
    \putnode{bc3}{bc2}{45}{0}{\blklab{B3}\blk{\bcc}}
    \putnode{bc4}{bc3}{-35}{-16}{\blklab{B4}\blk{\bcd}}
    \putnode{bc5}{bc1}{20}{-43}{\blklab{B5}\blk{\bce}}
    \putnode{bc6}{bc5}{43}{2}{\blklab{B6}\blk{\bcf}}

    \ncline[nodesepB=-5]{->}{bc1}{bc2}
    \ncline[nodesepB=-5]{->}{bc2}{bc3}
    \ncline[nodesepA=-.3,angleA=-90]{->}{bc3}{bc4}
    \ncline[nodesepB=-0.1]{->}{bc4}{bc5}
    \ncline[nodesepB=-0.1]{->}{bc4}{bc6}

    \ncangles[angleA=0,angleB=90,armA=.15cm,offset=1]{->}{bc2}{bc2}
    \ncangles[angleA=0,angleB=90,armA=.15cm,offset=1]{->}{bc3}{bc3}
    \ncangle[angleA=90,angleB=180,armB=.15cm,offsetB=-2,
      offsetA=2,nodesepB=-5]
            {->}{bc5}{bc2}
  \end{pspicture}
}


\frame{
  \frametitle{Benefits}
  \begin{minipage}[b]{.3\textwidth}
    \begin{center}\scriptsize
      \begin{tabular}[b]{@{}l|@{\ }c@{\ }|@{\ }c@{}} \hline
        {\bf B\#} & {\bf \# Stmts} & {\bf \# Stmts} \\
        & {\bf before} & {\bf after} \\
        & {\bf Opts} & {\bf Opts} \\ \hline \hline
        B1 & 4 & {\red 6} \\ \hline
        B2 & 4 & {\green 3} \\\hline
        B3 & 4 & {\green 3} \\\hline
        B4 & 1 & {1} \\\hline
        B5 & 9 & {\green 3} \\\hline
        B6 & 8 & {\green 3} \\\hline
        \multicolumn{3}{c}{}\\
      \end{tabular}
    \end{center}
  \end{minipage}
  \begin{minipage}[b]{.68\textwidth}
    \begin{itemize}[<+->]
    \item {\bf Assumptions:}
    \begin{itemize}
    \item Unit cost for each stmt
    \item Outer loop: 10 iterations
    \item Inner loops: 100 iterations each
    \end{itemize}
  \item {\bf Cost of Execution:}
    \begin{itemize}%\scriptsize
    \item {\bf Original Program:}\\
      {\scriptsize 1*4  + 100*4 + 100*4 + 10*1 + 10*9 + 1*8 = 912}
    \item {\bf Optimized Program:}\\
      {\scriptsize 1*6  + 100*3 + 100*3 + 10*1 + 10*3 + 1*3 = 649}
    \end{itemize}
  \end{itemize}
  \end{minipage}
}

\mypart{
  \begin{tabular}{c}
    Machine Dependent \\ Optimizations
  \end{tabular}
}

\frame{
  \frametitle{Peephole Optimizations}
  \begin{itemize}[<+->]
  \item Target code often contains redundant instructions and
    suboptimal constructs

  \item Examine a short sequence of target instruction (peephole)
    and replace by a shorter or faster sequence

  \item Peephole is a small moving window on the target systems
  \end{itemize}
}

\frame{
  \frametitle{Peephole Optimizations: Examples}
  \begin{itemize}[<+->]
  \item \alert{Redundant loads and stores}
  \item Consider the code sequence
    \begin{center}\tt
      \begin{tabular}{l}
        move $R_0$, $a$\\
        move $a$, $R_0$
      \end{tabular}
    \end{center}
  \item Is instruction 2 redundant? Can we always remove it?
    \begin{itemize}
    \item \alert{YES, if it does not have label}
    \end{itemize}
  \end{itemize}
}

\frame{
  \frametitle{Peephole Optimizations: Unreachable code}
  \begin{itemize}[<+->]
  \item Consider the following code
    {\tt\begin{tabbing}
      int debug = 0;\\
      if (\=debug) \{\\
        \>print debugging info\\
      \}
    \end{tabbing}}
  \item This may be translated as
    {\tt\begin{tabbing}
      \ \ \ \ \=int debug = 0;\\
      \>if (debug == 1) goto L1\\
      \>goto L2\\
      L1: \>print debugging info\\
      L2: \\
    \end{tabbing}}
  \end{itemize}
}
\frame{
  \frametitle{Peephole Optimizations: Unreachable code}
  \begin{itemize}[<+->]
  \item Eliminate Jumps
    {\tt\begin{tabbing}
      \ \ \ \ \=int debug = 0;\\
      \>if (debug != 1) goto L2\\
      \>print debugging info\\
      L2: \\
    \end{tabbing}}
  \item Constant propagation
    {\tt\begin{tabbing}
      \ \ \ \ \=int debug = 0;\\
      \>if (0 != 1) goto L2\\
      \>print debugging info\\
      L2: \\
    \end{tabbing}}
  \end{itemize}
}


\frame{
  \frametitle{Peephole Optimizations: Unreachable code}
  \begin{itemize}[<+->]
  \item Constant folding and simplification:  Since {\tt if} condition is always true, the code becomes:
    {\tt\begin{tabbing}
      \ \ \ \ \=goto L2\\
      \>print debugging info\\
      L2: \\
    \end{tabbing}}
  \item The print statement is now unreachable. Therefore, the code becomes
    {\tt\begin{tabbing}
      L2: \\
    \end{tabbing}}
  \end{itemize}
}

\frame{
  \frametitle{Peephole Optimizations: Jump Optimizations}
  \begin{itemize}
  \item Replace jump-over-jumps
    \begin{center}
      \begin{tabular}{|l|} \hline
        \\\ \ \ \ \ \ \ goto L1 \\
        \ \ \ \ \ \ \ \vdots \\
        L1: goto L2 \\ \\ \hline
      \end{tabular}
      \pause can be replaced by
      \begin{tabular}{|l|} \hline
        \\\ \ \ \ \ \ goto L2 \\
        \ \ \ \ \ \ \ \vdots \\
        L1: goto L2 \\ \\ \hline
      \end{tabular}
    \end{center}
  \end{itemize}
}

\frame{
  \frametitle{Peephole Optimizations: Simplify Algebraic Expressions}
  \begin{itemize}
  \item Remove 
    \begin{itemize}
    \item[] x = x + 0;
    \item[] x = x * 1;
    \end{itemize}
  \end{itemize}
}

\frame{
  \frametitle{Peephole Optimizations: Strength Reduction}
  \begin{itemize}[<+->]
  \item Replace $X^\wedge 2$ by $X*X$
  \item Replace multiplication by left shift
  \item Replace division by right shift
  \end{itemize}
}

\frame{
  \frametitle{Peephole Optimizations: Use of Faster Instructions}
  \begin{itemize}
  \item Replace 
    \begin{itemize}
    \item[] Add \#1, R
    \end{itemize}
    by
    \begin{itemize}
    \item[] Inc R
    \end{itemize}
  \end{itemize}
}

\mypart{Course Logistics}
\frame{
  \frametitle{Evaluation}
  \begin{itemize}
  \item Assignments
  \item Course project
  \item Mid semester exam (? for online offering)
  \item End semester exam (? for online offering)
  \item Quizzes/Class participation
  \item \alert{Refer to course webpage for details.}
  \end{itemize}
}
\end{document}
