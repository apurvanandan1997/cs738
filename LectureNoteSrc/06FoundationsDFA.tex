\mode<presentation>
{
  \usetheme{umbc4}
  %\setbeamercovered{dynamic}
}
\usepackage[english]{babel}
% or whatever

\usepackage[latin1]{inputenc}
% or whatever

\usepackage{times}
\usepackage[T1]{fontenc}
% Or whatever. Note that the encoding and the font should match. If T1
% does not look nice, try deleting the line with the fontenc.
%--------------------------------------------------------------------------------------------
%--- Other packages
\usepackage{pstricks}
\usepackage{pst-node}
\usepackage{pst-rel-points}
\usepackage{xspace}

%%%%%% Declarations Defined in custom-foils
%\setfootline{\insertshortauthor, \insertshortinstitute \hfill   \insertshorttitle \hfill \insertframenumber/\inserttotalframenumber} 
\newcommand{\mypart}[1]{%
  \frame[plain]{\mbox{}\vfill \psshadowbox{\huge #1} \vfill\mbox{}}
}
\author[karkare]{Amey Karkare \\ \url{karkare@cse.iitk.ac.in}}
\date[]{\scalebox{0.3}{\includegraphics{iitklogo.epsi}}}%
\institute[CSE, IITK]{\url{http://www.cse.iitk.ac.in/~karkare/cs738}\\Department of CSE, IIT Kanpur}
\title[CS738]{CS738: Advanced Compiler Optimizations}
\subtitle[]{}


\subtitle[]{\ \\{\LARGE Foundations of Data Flow Analysis}\\}

\newcommand{\bbent}{{\it Entry}\xspace}
\newcommand{\bbext}{{\it Exit}\xspace}

\newcommand{\IN}{{\sf IN}\xspace}
\newcommand{\OUT}{{\sf OUT}\xspace}

\newcommand{\Pred}{{\sf PRED}\xspace}
\newcommand{\Succ}{{\sf SUCC}\xspace}

\newcommand{\Kill}{{\sf KILL}\xspace}
\newcommand{\Gen}{{\sf GEN}\xspace}

\newcommand{\cmt}[1]{{}}
\newcommand{\meet}{\ensuremath{\bigwedge}\xspace}
\newcommand{\join}{\ensuremath{\bigvee}\xspace}
\newcommand{\glb}{{\sf glb}\xspace}
\newcommand{\lub}{{\sf lub}\xspace}

%% Framed minipage %%%%%%%%%%%%%%%%%%%%%%%%%%%%%%%%%%%%
\newsavebox\TestBox
\newenvironment{frmminipage}[2][]
{\begin{lrbox}{\TestBox}\begin{minipage}[#1]{#2}}
{\end{minipage}\end{lrbox}\fbox{\usebox{\TestBox}}}


\newcommand{\mylabbox}[2]{{\small #1}\psframebox{#2}\phantom{\small #1}}

\newcommand{\tossa}{%
  \ensuremath{\stackrel{SSA\quad\ }{\scalebox{3}{$\Rightarrow$}}}}

\newcommand{\jstfy}{\mbox{\phantom{XX}}}

\newcommand{\dom}{\operatorname{dom}}
\newcommand{\sdom}{\operatorname{sdom}}
\newcommand{\idom}{\operatorname{idom}}

\newcommand{\DOM}{\operatorname{DOM}}
\newcommand{\DF}{\operatorname{DF}}

\begin{document}

\frame{\titlepage}
\frame{
  \frametitle{Agenda}
  \begin{itemize}
  \item Poset, Lattice, and Data Flow Frameworks: Review 
  \item Connecting Tarski Lemma with Data Flow Analysis
  \item Soutions of Data Flow Analysis constraints
  \end{itemize}
}

\frame{
  \frametitle{Knaster-Tarski Fixed Point Theorem}
  \begin{itemize}
  \item Let $f$ be a monotonic function on a complete lattice $(S,\meet,\join)$. Define
    \begin{itemize}
    \item ${\sf red}(f) = \{v \mid v \in S, f(v) \leq v\}$, pre fix-points
    \item ${\sf ext}(f) = \{v \mid v \in S, f(v) \geq v\}$, post fix-points
    \item ${\sf fix}(f) = \{v \mid v \in S, f(v) = v\}$, fix-points
    \end{itemize}
    Then,
    \begin{itemize}
    \item $\meet {\sf red}(f) \in {\sf fix}(f)$. Further, $\meet {\sf red}(f) = \meet {\sf fix}(f)$
    \item $\join {\sf ext}(f) \in {\sf fix}(f)$. Further, $\join {\sf ext}(f) = \meet {\sf fix}(f)$
    \item ${\sf fix}(f)$ is a complete lattice 
    \end{itemize}
  \end{itemize}
}

\frame{
  \frametitle{Application of Fixed Point Theorem}
  \begin{itemize}
  \item $f: S\rightarrow S$ is a {\bf monotonic} function
  \item $(S, \meet)$ is a {\bf finite height} semilattice
  \item $\top$ is the top element of $(S, \meet)$
  \item Notation: $f^0(x) = x, f^{i+1}(x) = f(f^i(x)), \forall i \geq 0$
  \item The greatest fixed point of $f$ is
    \[ f^k(\top), \mbox{ where } f^{k+1}(\top) = f^k(\top) \]
  \end{itemize}
}

\frame{
  \frametitle{Fixed Point Algorithm}
  \texttt{// monotonic function f on a meet semilattice \\
    x := $\top$;\\
    while (x $\neq$ f(x)) x := f(x);\\
    return x;\\
  }
}

\frame{
  \frametitle{Resemblance to Iterative Algorithm (Forward)}
  \texttt{
    \begin{tabbing}
      $\OUT[\bbent] = {\tt Info}_\bbent$;\\ \pause
      for (other blocks $B$) $\OUT[B] = \top$;\\ \pause
      whil\=e (changes to any $\OUT$) \{\\ \pause
        \>for \=(each block $B$) \{\\ \pause
        \>\>$\IN(B) = \meet_{P\in\Pred(B)}\OUT(P)$;\\ \pause
        \>\>$\OUT(B) = f_B(\IN(B))$;\\
        \>\}\\
      \}
    \end{tabbing}
  }
}

\frame{
  \frametitle{Iterative Algorithm}
  \begin{itemize}[<+->]
  \item $f_B(X) = X - \Kill(B) \cup \Gen(B)$
  \item Backward:
    \begin{itemize}
    \item Swap \IN and \OUT everywhere
    \item Replace \bbent by \bbext
    \item Replace predecessors by successors
    \end{itemize}
  \item In other words: just ``invert'' the flow graph!!
  \end{itemize}
}

\frame{
  \frametitle{Acknowledgement}
  Rest of the slides based on the material at  \url{http://infolab.stanford.edu/~ullman/dragon/w06/} \url{w06.html}
}


\frame{
  \frametitle{Solutions}
  \begin{itemize}[<+->]
  \item {\bf IDEAL solution} = meet over all executable paths
    from entry to a point (ignore unrealizable paths)
  \item {\bf MOP} = meet over all paths from entry to a given point, of the transfer function along that path applied to ${\sf Info}_\bbent$.
  \item {\bf MFP} {\em (maximal fixedpoint)} = result of iterative algorithm.
  \end{itemize}
}

\frame{
  \frametitle{Maximum Fixedpoint}
  \begin{itemize}[<+->]
  \item {\bf Fixedpoint} = solution to the equations used in the iteration:
    \[\IN(B) = \meet_{P\in\Pred(B)}\OUT(P)\]
    \[\OUT(B) = f_B(\IN(B))\]
  \item {\bf Maximum Fixedpoint} = any other solution is $\leq$
    the result if the iterative algorithm (MFP)
  \item $\leq$: carries less information.
  \end{itemize}
}

\frame{
  \frametitle{MOP and IDEAL}
  \begin{itemize}[<+->]
  \item All solutions are really meets of the result of starting
    with ${\sf Info}_\bbent$ and following some set of paths to the point
    in question.
  \item If we don't include {\bf at least the IDEAL paths}, we
    have an error.
  \item But try not to include too many more.
  \item Less ``ignorance,'' but we ``know too much.''
  \end{itemize}
}

\frame{
  \frametitle{MOP Versus IDEAL}
  \begin{itemize}[<+->]
  \item Any solution that is $\leq$ IDEAL accounts for all
    executable paths (and maybe more paths)
    \begin{itemize}
    \item and is therefore conservative (safe)
    \item even if not accurate.
    \end{itemize}
  \end{itemize}
}

\frame{
  \frametitle{MFP vs MOP}
  \begin{itemize}[<+->]
  \item If MFP $\leq$ MOP?
    \begin{itemize}
    \item If so, then MFP $\leq$ MOP $\leq$ IDEAL, therefore MFP is safe.
    \end{itemize}
  \item Yes, but \ldots
  \item Requires two assumptions about the
    framework:
    \begin{itemize}
    \item ``Monotonicity.''
    \item Finite height  
    \item[] no infinite chains $\ldots < x_2 < x_1 < x < \ldots $
    \end{itemize}
  \end{itemize}
}

\frame{
  \frametitle{MFP vs MOP}
  \begin{itemize}[<+->]
  \item {\bf Intuition}: If we computed the MOP directly, we
    would compose functions along all paths, then take a big
    meet.
  \item But the MFP (iterative algorithm) alternates compositions
    and meets arbitrarily.
  \end{itemize}
}

\frame{
  \frametitle{Good News}
  \begin{itemize}[<+->]
  \item The frameworks we've studied so far are all monotone.
    \begin{itemize}
    \item Easy proof for functions in Gen-Kill form.
    \end{itemize}
  \item And they have finite height.
    \begin{itemize}
    \item Only a finite number of defs, variables, etc. in any program.
    \end{itemize}
  \end{itemize}
}

\frame{
  \frametitle{Two Paths to $B$ that Meet Early}
    \begin{center}
      \psset{unit=1mm}
      \begin{pspicture}(0,-10)(60,20)
        %\psframe(-30,-15)(90,30)
        \putnode{ent}{origin}{-22}{10}{\psframebox{\bbent}}
        \putnode{ox}{ent}{25}{10} {\psframebox[framesep=3]{}}
        \putnode{lx}{ox}{10}{0}{\scriptsize$\OUT=x$}
        \putnode{oy}{ent}{25}{-10}{\psframebox[framesep=3]{}}
        \putnode{ly}{oy}{10}{0}{\scriptsize$\OUT=y$}
        \putnode{ixy}{ent}{45}{0}{\psframebox[framesep=3]{}}
        \putnode{lixi}{ixy}{-11}{0}{\scriptsize$\IN=x\meet y$}
        \putnode{inf}{ixy}{30}{0}{\psframebox[framesep=3]{}}
        \putnode{b0}{inf}{15}{0}{\psframebox[framesep=1.8]{B}}
        \putnode{lb}{b0}{1}{-6}{\scriptsize$f(x\meet y)$}
        \putnode{rb}{b0}{2}{6}{\red\scriptsize$f(x)\meet f(y)$}

        \ncline[linestyle=dotted]{->}{ent}{ox}
        \ncline[linestyle=dotted]{->}{ent}{oy}

        \ncline{->}{ox}{ixy}
        \ncline{->}{oy}{ixy}

        \ncline[linestyle=dotted]{->}{ixy}{inf}
        \ncline[offset=5,nodesep=-3]{|-|}{ixy}{inf}
        \ncput*{\scriptsize$f$}
        \ncline{->}{inf}{b0}

        \nccurve[angleA=45,angleB=135,linecolor=red]{->}{ent}{b0}
        \ncput*{\red\scriptsize$f(x)$}
        \nccurve[angleA=-45,angleB=-135,linecolor=red]{->}{ent}{b0}
        \ncput*{\red\scriptsize$f(y
          )$}

      \end{pspicture}      
    \end{center}\pause
    \begin{itemize}[<+->]
    \item {\red MOP considers paths independently and combines at the
      last possible moment.}
    \item In MFP, Values x and y get combined too soon.
    \item Since $f(x\meet y) \leq f(x) \meet f(y)$, it is as we
      added non-existent paths.
    \end{itemize}
}

\frame{
  \frametitle{Distributive Frameworks}
  \begin{itemize}[<+->]
  \item Distributivity:
    \[ f(x\meet y) = f(x) \meet f(y)\]
  \item Stronger than Monotonicity
    \begin{itemize}
    \item Distributivity $\Rightarrow$ Monotonicity
    \item But the reverse is not true
    \end{itemize}
  \end{itemize}
}

\frame{
  \frametitle{Even More Good News!}
  \begin{itemize}[<+->]
  \item The 4 example frameworks are distributive.
  \item If a framework is distributive, then combining paths early doesn't hurt.
    \begin{itemize}
    \item MOP = MFP.
    \item That is, the iterative algorithm computes a solution
      that takes into account all and only the physical paths.
    \end{itemize}
  \end{itemize}
}

\end{document}
