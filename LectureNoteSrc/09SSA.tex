\mode<presentation>
{
  \usetheme{umbc4}
  %\setbeamercovered{dynamic}
}
\usepackage[english]{babel}
% or whatever

\usepackage[latin1]{inputenc}
% or whatever

\usepackage{times}
\usepackage[T1]{fontenc}
% Or whatever. Note that the encoding and the font should match. If T1
% does not look nice, try deleting the line with the fontenc.
%--------------------------------------------------------------------------------------------
%--- Other packages
\usepackage{pstricks}
\usepackage{pst-node}
\usepackage{pst-rel-points}
\usepackage{xspace}

%%%%%% Declarations Defined in custom-foils
%\setfootline{\insertshortauthor, \insertshortinstitute \hfill   \insertshorttitle \hfill \insertframenumber/\inserttotalframenumber} 
\newcommand{\mypart}[1]{%
  \frame[plain]{\mbox{}\vfill \psshadowbox{\huge #1} \vfill\mbox{}}
}
\author[karkare]{Amey Karkare \\ \url{karkare@cse.iitk.ac.in}}
\date[]{\scalebox{0.3}{\includegraphics{iitklogo.epsi}}}%
\institute[CSE, IITK]{\url{http://www.cse.iitk.ac.in/~karkare/cs738}\\Department of CSE, IIT Kanpur}
\title[CS738]{CS738: Advanced Compiler Optimizations}
\subtitle[]{}


\subtitle[]{\ \\{\LARGE Static Single Assignment (SSA)}\\}

\newcommand{\bbent}{{\it Entry}\xspace}
\newcommand{\bbext}{{\it Exit}\xspace}

\newcommand{\IN}{{\sf IN}\xspace}
\newcommand{\OUT}{{\sf OUT}\xspace}

\newcommand{\Pred}{{\sf PRED}\xspace}
\newcommand{\Succ}{{\sf SUCC}\xspace}

\newcommand{\Kill}{{\sf KILL}\xspace}
\newcommand{\Gen}{{\sf GEN}\xspace}

\newcommand{\cmt}[1]{{}}
\newcommand{\meet}{\ensuremath{\bigwedge}\xspace}
\newcommand{\join}{\ensuremath{\bigvee}\xspace}
\newcommand{\glb}{{\sf glb}\xspace}
\newcommand{\lub}{{\sf lub}\xspace}

%% Framed minipage %%%%%%%%%%%%%%%%%%%%%%%%%%%%%%%%%%%%
\newsavebox\TestBox
\newenvironment{frmminipage}[2][]
{\begin{lrbox}{\TestBox}\begin{minipage}[#1]{#2}}
{\end{minipage}\end{lrbox}\fbox{\usebox{\TestBox}}}


\newcommand{\mylabbox}[2]{{\small #1}\psframebox{#2}\phantom{\small #1}}

\newcommand{\tossa}{%
  \ensuremath{\stackrel{SSA\quad\ }{\scalebox{3}{$\Rightarrow$}}}}

\newcommand{\jstfy}{\mbox{\phantom{XX}}}

\newcommand{\dom}{\operatorname{dom}}
\newcommand{\sdom}{\operatorname{sdom}}
\newcommand{\idom}{\operatorname{idom}}

\newcommand{\DOM}{\operatorname{DOM}}
\newcommand{\DF}{\operatorname{DF}}

\begin{document}

\frame{\titlepage}
\frame{
  \frametitle{Agenda}
  \begin{itemize}
  \item SSA Form
  \item Constructing SSA form
  \item Properties and Applications
  \end{itemize}
}


\frame{
  \frametitle{SSA Form}
  \begin{itemize}[<+->]
  \item Developed by Ron Cytron, Jeanne Ferrante, Barry K. Rosen, Mark N. Wegman, and F. Kenneth Zadeck, 
    \begin{itemize}
    \item in 1980s while at IBM.
    \end{itemize}
  \item {\em Static Single Assignment} -- A variable is assigned only once in program text
    \begin{itemize}
    \item May be assigned multiple times if program is executed
    \end{itemize}
  \end{itemize}
}

\frame{
  \frametitle{What is SSA Form?}
  \begin{itemize}[<+->]
  \item An Intermediate Representation
  \item Sparse representation
    \begin{itemize}
    \item Definitions sites are directly associated with use sites
    \end{itemize}
  \item Advantage
    \begin{itemize}
    \item Directly access points where relevant data flow information is avaliable
    \end{itemize}
  \end{itemize}
}

\frame{
  \frametitle{SSA IR}
  \begin{itemize}[<+->]
  \item In SSA Form
    \begin{itemize}
    \item Each variable has exactly one definition
    \item[$\Rightarrow$] A use of a variable is reached by exactly one definition
    \end{itemize}
  \item Control flow like traditional programs
  \item Some {\em magic} is needed at {\em join} nodes
  \end{itemize}
}

\frame{
  \frametitle{Example}
  \begin{frmminipage}{.4\textwidth}
    \Large\tt
    i = 0;\\
    \ldots\\
    i = i + 1;\\
    \ldots\\
    j = i * 5;\\
    \ldots
  \end{frmminipage}
  \pause
  \hfill\tossa\hfill
  \begin{frmminipage}{.4\textwidth}
    \Large\tt
    i1 = 0;\\
    \ldots\\
    i2 = i1 + 1;\\
    \ldots\\
    j1 = i2 * 5;\\
    \ldots
  \end{frmminipage}
}

\frame{
  \frametitle{SSA Example}
  \begin{frmminipage}{.4\textwidth}
    \Large\tt
    i = \ldots;\\
    j = \ldots;\\
    if (i < 20) \\
    \jstfy i = i + j;\\
    else\\
    \jstfy j = j + 2;\\
    print i, j;
  \end{frmminipage}\hfill
  \begin{minipage}{.4\textwidth}
    \psset{unit=1mm}
    \begin{pspicture}(0,0)(45,30)
      %\psframe(0,0)(45,30)
      \putnode{n0}{origin}{23}{35}{\psframebox{
          \begin{tabular}{@{}c@{}}
            i = \ldots;\\
            j = \ldots;
          \end{tabular}
      }}
      \putnode{if0}{n0}{0}{-15}{\psframebox{if (i < 20)}}
      \putnode{n1}{if0}{-15}{-15}{\psframebox{
          \begin{tabular}{@{}c@{}}
            i = i + j;
          \end{tabular}
      }}
      \putnode{n2}{if0}{15}{-15}{\psframebox{
          \begin{tabular}{@{}c@{}}
            j = j + 2;
          \end{tabular}
      }}
      \putnode{n3}{if0}{0}{-30}{\psframebox{
          \begin{tabular}{@{}c@{}}
            print i, j;
          \end{tabular}
      }}
      \ncline{->}{n0}{if0}
      \ncline{->}{if0}{n1}
      \ncline{->}{if0}{n2}
      \ncline{->}{n1}{n3}
      \ncline{->}{n2}{n3}
    \end{pspicture}
  \end{minipage}
  
}

\frame{
  \frametitle{SSA Example}
  \begin{minipage}{.4\textwidth}
    \psset{unit=1mm}
    \begin{pspicture}(0,0)(45,30)
      %\psframe(0,0)(45,30)
      \putnode{n0}{origin}{23}{40}{\psframebox{
          \begin{tabular}{@{}c@{}}
            i = \ldots;\\
            j = \ldots;
          \end{tabular}
      }}
      \putnode{if0}{n0}{0}{-15}{\psframebox{if (i < 20)}}
      \putnode{n1}{if0}{-15}{-15}{\psframebox{
          \begin{tabular}{@{}c@{}}
            i = i + j;
          \end{tabular}
      }}
      \putnode{n2}{if0}{15}{-15}{\psframebox{
          \begin{tabular}{@{}c@{}}
            j = j + 2;
          \end{tabular}
      }}
      \putnode{n3}{if0}{0}{-35}{\psframebox{
          \begin{tabular}{@{}c@{}}
            \\ \\ print i, j;
          \end{tabular}
      }}
      \ncline{->}{n0}{if0}
      \ncline{->}{if0}{n1}
      \ncline{->}{if0}{n2}
      \ncline{->}{n1}{n3}
      \ncline{->}{n2}{n3}
    \end{pspicture}
  \end{minipage}
  \pause\hfill\tossa\hfill
  \begin{minipage}{.4\textwidth}
    \psset{unit=1mm}
    \begin{pspicture}(0,0)(45,30)
      %\psframe(0,0)(45,30)
      \putnode{n0}{origin}{23}{40}{\psframebox{
          \begin{tabular}{@{}c@{}}
            i1 = \ldots;\\
            j1 = \ldots;
          \end{tabular}
      }}
      \putnode{if0}{n0}{0}{-15}{\psframebox{if (i1 < 20)}}
      \putnode{n1}{if0}{-15}{-15}{\psframebox{
          \begin{tabular}{@{}c@{}}
            i2 = i1 + j1;
          \end{tabular}
      }}
      \putnode{n2}{if0}{15}{-15}{\psframebox{
          \begin{tabular}{@{}c@{}}
            j2 = j1 + 2;
          \end{tabular}
      }}
      \putnode{n3}{if0}{0}{-35}{\psframebox{
          \begin{tabular}{@{}c@{}}
            \onslide<3->{%
              i3 = $\phi$(i2, i1);\\
              j3 = $\phi$(j1, j2);\\
            }
            print i\onslide<4->{3}, j\onslide<4->{3};
          \end{tabular}
      }}
      \ncline{->}{n0}{if0}
      \ncline{->}{if0}{n1}
      \ncline{->}{if0}{n2}
      \ncline{->}{n1}{n3}
      \ncline{->}{n2}{n3}
    \end{pspicture}
  \end{minipage}
}

\frame{
  \frametitle{SSA Example}
  \begin{frmminipage}{.4\textwidth}
    \Large\tt
    i = \ldots;\\
    j = \ldots;\\
    if (i < 20) \\
    \jstfy i = i + j;\\
    else\\
    \jstfy j = j + 2;\\
    \\ \\ print i, j;
  \end{frmminipage}
  \hfill\tossa\hfill
  \begin{frmminipage}{.43\textwidth}
    \Large\tt
    i1 = \ldots;\\
    j1 = \ldots;\\
    if (i1 < 20) \\
    \jstfy i2 = i1 + j1;\\
    else\\
    \jstfy j2 = j1 + 2;\\
    i3 = $\phi$(i2, i1);\\
    j3 = $\phi$(j1, j2);\\
    print i3, j3;
  \end{frmminipage}
}

\frame{
  \frametitle{The {\em magic}: $\phi$ function}
  \begin{itemize}[<+->]
  \item $\phi$ is used for selection
    \begin{itemize}
    \item One out of multiple values at join nodes
    \end{itemize}
  \item Not every join node needs a $\phi$
    \begin{itemize}
    \item Needed only if multiple definitions reach the node
    \end{itemize}
  \item Examples?
  \end{itemize}
}

\frame{
  \frametitle{But\ldots What is $\phi$?}
  \begin{itemize}[<+->]
  \item What does $\phi$ operation mean in a machine code?
  \item $\phi$ is a conceptual entity
  \item Statically equivalent to chosing one of the arguments ``non-deterministicly'' 
    \item No direct translation to machine code
    \begin{itemize}
    \item typically mimicked using ``copy'' in predecessors
    \item Inefficient
    \item Practically, the inefficiency is compensated by dead
      code elimination and register allocation passes
    \end{itemize}
  \end{itemize}
}

\frame{
  \frametitle{Properties of $\phi$}
  \begin{itemize}[<+->]
  \item Placed only at the entry of a join node
  \item Multiple $\phi$-functions could be placed
    \begin{itemize}
    \item for multiple variables
    \item all such $\phi$ functions execute concurrently
    \end{itemize}
  \item $n$-ary $\phi$ function at $n$-way join node
  \item gets the value of $i$-th argument if control enters
    through $i$-th edge
    \begin{itemize}
    \item Ordering of $\phi$ arguments according to the edge
      ordering is important
    \end{itemize}
  \end{itemize}
}


\frame{
  \frametitle{SSA Example (revisit)}
  \begin{minipage}{.4\textwidth}
    \psset{unit=1mm}
    \begin{pspicture}(0,0)(45,30)
      %\psframe(0,0)(45,30)
      \putnode{n0}{origin}{23}{40}{\psframebox{
          \begin{tabular}{@{}c@{}}
            i = \ldots;\\
            j = \ldots;
          \end{tabular}
      }}
      \putnode{if0}{n0}{0}{-15}{\psframebox{if (i < 20)}}
      \putnode{n1}{if0}{-15}{-15}{\psframebox{
          \begin{tabular}{@{}c@{}}
            i = i + j;
          \end{tabular}
      }}
      \putnode{n2}{if0}{15}{-15}{\psframebox{
          \begin{tabular}{@{}c@{}}
            j = j + 2;
          \end{tabular}
      }}
      \putnode{n3}{if0}{0}{-35}{\psframebox{
          \begin{tabular}{@{}c@{}}
            \\ \\ print i, j;
          \end{tabular}
      }}
      \ncline{->}{n0}{if0}
      \ncline{->}{if0}{n1}
      \ncline{->}{if0}{n2}
      \ncline{->}{n1}{n3}
      \ncline{->}{n2}{n3}
    \end{pspicture}
  \end{minipage}
  \hfill\tossa\hfill
  \begin{minipage}{.4\textwidth}
    \psset{unit=1mm}
    \begin{pspicture}(0,0)(45,30)
      %\psframe(0,0)(45,30)
      \putnode{n0}{origin}{23}{40}{\psframebox{
          \begin{tabular}{@{}c@{}}
            i1 = \ldots;\\
            j1 = \ldots;
          \end{tabular}
      }}
      \putnode{if0}{n0}{0}{-15}{\psframebox{if (i1 < 20)}}
      \putnode{n1}{if0}{-15}{-15}{\psframebox{
          \begin{tabular}{@{}c@{}}
            i2 = i1 + j1;
          \end{tabular}
      }}
      \putnode{n2}{if0}{15}{-15}{\psframebox{
          \begin{tabular}{@{}c@{}}
            j2 = j1 + 2;
          \end{tabular}
      }}
      \putnode{n3}{if0}{0}{-35}{\psframebox{
          \begin{tabular}{@{}c@{}}
            \onslide<1->{%
              i3 = $\phi$(i2, i1);\\
              j3 = $\phi$(j1, j2);\\
            }
            print i\onslide<1->{3}, j\onslide<1->{3};
          \end{tabular}
      }}
      \ncline{->}{n0}{if0}
      \ncline{->}{if0}{n1}
      \ncline{->}{if0}{n2}
      \ncline{->}{n1}{n3}
      \ncline{->}{n2}{n3}
    \end{pspicture}
  \end{minipage}
}

\mypart{Construction of SSA Form}

\frame{
  \frametitle{Assumptions}
  \begin{itemize}[<+->]
  \item Only scalar variables
    \begin{itemize}
    \item Structures, pointers, arrays could be handled
    \item Refer to publications
    \end{itemize}
  \end{itemize}
}

\frame{
  \frametitle{Dominators}
  \begin{itemize}[<+->]
  \item Nodes $x$ and $y$ in flow graph
  \item $x$ \alert{dominates} $y$ if {\bf every} path from $\bbent$
    to $y$ goes through $x$
    \begin{itemize}
    \item $x \dom y$
    \item partial order?
    \end{itemize}
  \item $x$ \alert{strictly dominates} $y$ if  $x\dom y$ and $x \neq y$
    \begin{itemize}
    \item $x \sdom y$
    \end{itemize}
  \end{itemize}
}

\frame{
  \frametitle{Computing Dominators}
  \begin{itemize}[<+->]
  \item Equation
    \begin{eqnarray*}
      \DOM(N) &=& \{n\} \cup \left( \bigcap_{m\in\Pred(n)} \DOM(m)\right)
    \end{eqnarray*}
  \item Initial Conditions:
    \begin{eqnarray*}
      \DOM(n_\bbent) &=& \{n_\bbent\} \\
      \DOM(n) &=& N,  \forall n \in N - \{n_\bbent\}       
    \end{eqnarray*}
    where $N$ is the set of all nodes, $n_\bbent$ is the node
    corresponding to the $\bbent$ block.
  \item Note that efficient methods exist for computing dominators
  \end{itemize}
}

\frame{
  \frametitle{Immediate Dominators and Dominator Tree}
  \begin{itemize}[<+->]
  \item $x$ is \alert{immediate dominator} of $y$ if $x$ is the
    {\em closest strict dominator} of $y$
    \begin{itemize}
    \item unique, if it exists
    \item denoted $\idom[y]$
    \end{itemize}
  \item Dominator Tree
    \begin{itemize}
    \item A tree showing all immediate dominator relationships
    \end{itemize}
  \end{itemize}
}

\frame{
  \frametitle{Dominator Tree Example}
  \begin{minipage}{.4\textwidth}
    \psset{unit=1mm}
    \begin{pspicture}(0,0)(45,60)
      %\psframe(0,0)(45,60)
      \putnode{n0}{origin}{13}{55}{$B_0$}
      \putnode{n1}{n0}{0}{-10}{$B_1$}
      \putnode{n2}{n1}{-10}{-10}{$B_2$}
      \putnode{n3}{n1}{10}{-10}{$B_3$}
      \putnode{n4}{n3}{-7}{-10}{$B_4$}
      \putnode{n5}{n3}{7}{-10}{$B_5$}
      \putnode{n6}{n3}{0}{-20}{$B_6$}
      \putnode{n7}{n1}{0}{-40}{$B_7$}
      \ncline{->}{n0}{n1}
      \ncline{->}{n1}{n2}
      \ncline{->}{n1}{n3}
      \ncline{->}{n3}{n4}
      \ncline{->}{n3}{n5}
      \ncline{->}{n4}{n6}
      \ncline{->}{n5}{n6}
      \ncline{->}{n2}{n7}
      \ncline{->}{n6}{n7}
      \nccurve[ncurv=2]{->}{n7}{n1}
    \end{pspicture}
  \end{minipage}
  \begin{minipage}{.4\textwidth}
    \psset{unit=1mm}
    \begin{pspicture}(0,0)(45,60)
      %\psframe(0,0)(45,30)
      \putnode{n0}{origin}{23}{55}{$B_0$}
      \putnode{n1}{n0}{0}{-10}{$B_1$}
      \putnode{n2}{n1}{-10}{-10}{$B_2$}
      \putnode{n3}{n1}{10}{-10}{$B_3$}
      \putnode{n4}{n3}{-7}{-10}{$B_4$}
      \putnode{n5}{n3}{7}{-10}{$B_5$}
      \putnode{n6}{n3}{0}{-20}{$B_6$}
      \putnode{n7}{n1}{0}{-40}{$B_7$}
      \pause \ncline{->}{n0}{n1}
      \pause \ncline{->}{n1}{n2}
      \pause \ncline{->}{n1}{n3}
      \pause \ncline{->}{n1}{n7}
      \pause \ncline{->}{n3}{n4}
      \pause \ncline{->}{n3}{n5}
      \pause \ncline{->}{n3}{n6}
    \end{pspicture}
  \end{minipage}
}

\frame{
  \frametitle{Dominance Frontier: $\DF$}
  \begin{itemize}[<+->]
  \item Dominance Frontier of $x$ is set of all nodes $y$ s.t.
    \begin{itemize}
    \item $x$ \alert{dominates a predecessor} of $y$ AND
    \item $x$ \alert{does not strictly dominate} $y$
    \end{itemize}
  \item Denoted $\DF(x)$
  \item Why do you think $\DF(x)$ is important for any $x$?
    \begin{itemize}
    \item Think about the information originated in $x$. 
    \end{itemize}
  \end{itemize}
}

\frame{
  \frametitle{Computing $\DF$}
  \begin{itemize}
  \item $\operatorname{PARENT}(x)$ denotes parent of node $x$ in the
    dominator tree.
  \item $\operatorname{CHILDERN}(x)$ denotes childern of node $x$ in the
    dominator tree.
  \item \Pred and \Succ from CFG.
  \end{itemize}
  \begin{eqnarray*}
    \DF(x) &=& \DF_{\operatorname{local}}(x) \cup
    \left(\bigcup_{z \in \operatorname{CHILDERN}(x)} \DF_{\operatorname{up}}(z)\right)
    \\ & \\ \pause
    \DF_{\operatorname{local}}(x) &=&
    \{y \in \Succ(x) \mid \idom[y] \neq x\} \\ \pause
    \DF_{\operatorname{up}}(z) &=&
    \{y \in \DF(z) \mid \idom[y] \neq \operatorname{PARENT}(z)\}
  \end{eqnarray*}
}

\frame{
  \frametitle{Iterated Dominance Frontier}
  \begin{itemize}[<+->]
  \item Transitive closure of Dominance frontiers on a set of nodes
  \item Let $S$ be a set of nodes
    \begin{eqnarray*}
      \DF(S) &=& \bigcup_{x\in S} \DF(x) \\
      & \\ \pause
      \DF^1(S) &=& \DF(S) \\
      \DF^{i+1}(S) &=& \DF( S \cup \DF^i(S))
    \end{eqnarray*}
  \item $\DF^+(S)$ is the fixed point of $\DF^i$ computation.
  \end{itemize}
}

\frame{
  \frametitle{Minimal SSA Form Construction}
  \begin{itemize}[<+->]
  \item Compute $\DF^+$ set for each flow graph node
  \item Place \alert{trivial} $\phi$-functions for each variable in the node
    \begin{itemize}
    \item trivial $\phi$-function at $n$-ary join:
      $x = \phi(\overbrace{x, x, \ldots, x}^\text{$n$-times})$
    \end{itemize}
  \item Rename variables
  \item \alert{Why $\DF^+$? Why not only $\DF$?}
  \end{itemize}
}

\frame{
  \frametitle{Inserting $\phi$-functions}
  \tt
  \begin{tabbing}
    for\=each variable $v$ \{\\\pause
    \>$S = \bbent \cup \{ B_n \mid v \mbox{ defined in } B_n\}$\\\pause
    \>Compute $\DF^+(S)$\\\pause
    \>for\=each $n$ in $\DF^+(S)$ \{\\\pause
    \>\>insert $\phi$-function for $v$ at the start of $B_n$\\
    \>\}\\
    \}
  \end{tabbing}
}

\frame{
  \frametitle{Renaming Variables (Pseudo Code)}
  \begin{itemize}[<+->]
  \item Rename from the \bbent node recursively
    \begin{itemize}
    \item For each variable $x$, maintain a rename stack of $x
    \mapsto x_{\operatorname{version}}$ mapping
    \end{itemize}
  \item For node n
    \begin{itemize}
    \item For each assignment ($x = \ldots$) in $n$
      \begin{itemize}
      \item If non-$\phi$ assignment, rename any use of $x$ with
        the Top mapping of $x$ from the rename stack
      \item Push the mapping $x \mapsto x_i$ on the rename stack
      \item Replace lhs of the assignment by $x_i$
      \item $i = i + 1$
      \end{itemize}
    \end{itemize}
  \item For the successors of $n$
    \begin{itemize}
    \item Rename $\phi$ operands through \Succ edge index
    \end{itemize}
  \item Recursively rename all child nodes in the dominator tree 
  \item For each assignment ($x = \ldots$) in $n$ 
    \begin{itemize}
    \item Pop $x\mapsto \ldots$  from the rename stack
    \end{itemize}
  \end{itemize}
}

\end{document}
