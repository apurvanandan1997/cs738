\mode<presentation>
{
  \usetheme{umbc4}
  %\setbeamercovered{dynamic}
}
\usepackage[english]{babel}
% or whatever

\usepackage[latin1]{inputenc}
% or whatever

\usepackage{times}
\usepackage[T1]{fontenc}
% Or whatever. Note that the encoding and the font should match. If T1
% does not look nice, try deleting the line with the fontenc.
%--------------------------------------------------------------------------------------------
%--- Other packages
\usepackage{pstricks}
\usepackage{pst-node}
\usepackage{pst-rel-points}

\newcommand{\change}[1]{{\red #1}}
\newcommand{\myexample}{
  \begin{center}
    \psset{unit=1mm}
    \begin{pspicture}(0,0)(50,-78)
      %\psframe(0,0)(50,-78)
      \putnode{entry}{origin}{25}{5}{\psframebox{ENTRY}}
      
      \putnode{b1}{entry}{0}{-11}{\psframebox{a = 2}}
      \putnode{b2}{b1}{0}{-13}{\psframebox{b = 3}}
      \putnode{b3}{b2}{0}{-13}{\psframebox{a < b}}
      \putnode{b4}{b3}{-13}{-13}{\psframebox{c1 = 4}}
      \putnode{b5}{b3}{10}{-13}{\psframebox{c2 = 5}}
      \putnode{b6}{b3}{0}{-26}{\psframebox{c3 = $\phi$(c1, c2)}}

      \putnode{exit}{entry}{0}{-74}{\psframebox{EXIT}}

      \ncline{->}{entry}{b1}
      \ncline{->}{b1}{b2}
      \ncline{->}{b2}{b3}
      \ncline{->}{b3}{b4} \Bput[0.1]{true}
      \ncline{->}{b3}{b5} \Aput[0.1]{false}
      \ncline{->}{b4}{b6}
      \ncline{->}{b5}{b6}
      \ncline{->}{b6}{exit}

      \nccurve[linestyle=dotted,dotsep=0.3]{->}{b1}{b3}
      \nccurve[linestyle=dotted,dotsep=0.3,angle=180]{->}{b2}{b3}
      \nccurve[linestyle=dotted,dotsep=0.3,angle=180]{->}{b4}{b6}
      \nccurve[linestyle=dotted,dotsep=0.3]{->}{b5}{b6}
    \end{pspicture}
  \end{center}
}

%%%%%% Declarations Defined in custom-foils
%\setfootline{\insertshortauthor, \insertshortinstitute \hfill   \insertshorttitle \hfill \insertframenumber/\inserttotalframenumber} 
\newcommand{\mypart}[1]{%
  \frame[plain]{\mbox{}\vfill \psshadowbox{\huge #1} \vfill\mbox{}}
}
\author[karkare]{Amey Karkare \\ \url{karkare@cse.iitk.ac.in}}
\date[]{\scalebox{0.3}{\includegraphics{iitklogo.epsi}}}%
\institute[CSE, IITK]{\url{http://www.cse.iitk.ac.in/~karkare/cs738}\\Department of CSE, IIT Kanpur}
\title[CS738]{CS738: Advanced Compiler Optimizations}
\subtitle[]{}


\subtitle[]{\ \\{\LARGE Overview of Optimizations}\\}

\begin{document}

\frame{\titlepage}
\frame{
  \frametitle{Recap}
  \begin{itemize}[<+->]
  \item Optimizations
    \begin{itemize}
    \item To improve efficiency of generated executable (time, space, resources, \ldots)
    \item Maintain semantic equivalence
    \end{itemize}
  \item Two levels
    \begin{itemize}
    \item Machine Independent
    \item Machine Dependent 
    \end{itemize}
  \end{itemize}
}

\mypart{
  \begin{tabular}{c}
    Machine Independent \\ Code Optimizations
  \end{tabular}
}

\frame{
  \frametitle{Machine Independent Optimizations}
  \begin{itemize}[<+->]
  \item Scope of optimizations
    \begin{itemize}[<+->]
    \item Intraprocedural
      \begin{itemize}
      \item Local
      \item Global
      \end{itemize}
    \item Interprocedural
    \end{itemize}
  \end{itemize}
}

\frame{
  \frametitle{Local Optimizations}
  \begin{itemize}[<+->]
  \item Restricted to a basic block
  \item Simplifies the analysis
  \item Not all optimizations can be applied locally
    \begin{itemize}[<+->]
    \item E.g. Loop optimizations
  \end{itemize}
  \item Gains are also limited
  \item Simplify global/interprocedural optimizations
  \end{itemize}
}

\frame{
  \frametitle{Global Optimizations}
    \begin{itemize}[<+->]
    \item Typically restricted within a procedure/function
      \begin{itemize}[<+->]
      \item Could be restricted to a smaller scope, e.g. a loop
      \end{itemize}
    \item Most compiler implement up to global optimizations
    \item Well founded theory
    \item Practical gains
    \end{itemize}
}

\frame{
  \frametitle{Interprocedural Optimizations}
    \begin{itemize}[<+->]
    \item Spans multiple procedures, files
      \begin{itemize}[<+->]
      \item In some cases multiple languages!
      \end{itemize}
    \item Not as popular as global optimizations
      \begin{itemize}[<+->]
      \item No single theory applicable to all scenarios
      \item Time consuming
      \end{itemize}
    \end{itemize}
}

\mypart{
  \begin{tabular}{c}
    A Catalog of \\ Code Optimizations
  \end{tabular}
}

\frame{
  \frametitle{Compile-time Evaluation}
  \begin{itemize}[<+->]
  \item Move run-time actions to compile-time
  \item Constant Folding
    \[ \mbox{Volume} = \frac{4}{3}\times \pi
    \times\ r \times r\times r\]
    \begin{itemize}
    \item Compute $\frac{4}{3}\times \pi$ at compile-time
    \item Applied frequently for linearizing indices of
      multidimensional arrays
    \end{itemize}
  \item \alert{\psshadowbox{When should we NOT apply it?}}
  \end{itemize}
}

\frame{
  \frametitle{Compile-time Evaluation}
  \begin{itemize}
  \item<1-> Constant Propagation
    \begin{itemize}
    \item Replace a variable by its ``constant'' value
      \begin{center}
        \begin{tabular}{|l|} \hline
          \\i = 5 \\
          \vdots \\
          j = i * 4\\ \\ \hline
        \end{tabular}
        \pause can be replaced by
        \begin{tabular}{|l|} \hline
          \\ i = 5 \\
          \vdots \\
          j = 5 * 4\\ \\\hline
        \end{tabular}
      \end{center}\pause
    \item May result in the application of constant folding
    \end{itemize}\pause
  \item \alert{\psshadowbox{When should we NOT apply it?}}
  \end{itemize}
}

\frame{
  \frametitle{Common Subexpression Elimination}
  \begin{itemize}
  \item Reuse a computation if already ``available''
    \begin{center}
      \begin{tabular}{|l|} \hline
        \\x = u + v \\
        \vdots \\
        y = u + v\\ \\ \hline
      \end{tabular}
      \pause can be replaced by
      \begin{tabular}{|l|} \hline
        \\ t = u + v
        \\ x = t \\
        \vdots \\
        y = t\\ \\\hline
      \end{tabular}
    \end{center}\pause
  \item How to check if an expression is already available?\pause
  \item \alert{\psshadowbox{When should we NOT apply it?}}
  \end{itemize}
}

\frame{
  \frametitle{Copy Propagation}
  \begin{itemize}
  \item Replace (use of) a variable by  another variable
    \begin{itemize}
    \item If they are guaranteed to have the ``same value''
      \begin{center}
        \begin{tabular}{|l|} \hline
          \\i = k \\
          \vdots \\
          j = i * 4\\ \\ \hline
        \end{tabular}
        \pause can be replaced by
        \begin{tabular}{|l|} \hline
          \\ i = k \\
          \vdots \\
          j = k * 4\\ \\\hline
        \end{tabular}
      \end{center}\pause
    \item May result in dead code, common subexpression
    \end{itemize}\pause
  \item \alert{\psshadowbox{When should we NOT apply it?}}
  \end{itemize}
}

\frame{
  \frametitle{Code Movement}
  \begin{itemize}[<+->]
  \item Move the code around in a program
  \item Benefits
    \begin{itemize}
    \item Code size reduction
    \item Reduction in the frequency of execution
    \end{itemize}
  \item How to find out which code to move?
  \end{itemize}
}

\frame{
  \newcommand{\opp}{\mbox{$\oplus$}}
  \frametitle{Code Movement}
  \begin{itemize}
  \item Code size reduction
    \begin{itemize}
    \item Suppose the operator $\opp$ results in the generation of
      a large number of machine instructions. Then,
      \begin{center}
        \begin{tabular}{|l|} \hline
          \\
          if (a $<$ b)\\
          \ \ \ \ u = x\opp y \\
          else \\
          \ \ \ \ v = x\opp y \\
          \\ \hline
        \end{tabular}
        \pause can be replaced by
        \begin{tabular}{|l|} \hline
          \\
          t = x\opp y \\
          if (a $<$ b)\\
          \ \ \ \ u = t \\
          else \\
          \ \ \ \ v = t \\
          \\ \hline
        \end{tabular}
      \end{center}\pause
    \end{itemize}
  \item \alert{\psshadowbox{When should we NOT apply it?}}
  \end{itemize}
}

\frame{
  \frametitle{Code Movement}
  \begin{itemize}
  \item Execution frequency reduction
    \begin{center}
      \begin{tabular}{|l|} \hline
        \\
        if (a $<$ b)\\
        \ \ \ \ u = \ldots  \\
        else \\
        \ \ \ \ v = x *  y \\
        w = x * y \\
        \\ \hline
        \end{tabular}
        \pause can be replaced by
        \begin{tabular}{|l|} \hline
          \\
          if (a $<$ b)\\
          \ \ \ \ u = \ldots \\
          \ \ \ \ t = x *  y \\
          else \\
          \ \ \ \ t = x * y \\
          \ \ \ \ v = t \\
          w = t \\
          \\ \hline
        \end{tabular}
    \end{center} \pause
  \item \alert{\psshadowbox{When should we NOT apply it?}}
  \end{itemize}
}


\frame{
  \frametitle{Loop Invariant Code Movement}
  \begin{itemize}
  \item Move loop invariant code out of the loop
    \begin{center}
      \begin{tabular}{|l|} \hline
        \\
        for (\ldots) \{ \\
        \ \ \ \ \ldots  \\
        \ \ \ \ u = a + b \\
        \ \ \ \ \ldots \\
        \} \\ \\ \hline
      \end{tabular}
      \pause can be replaced by
      \begin{tabular}{|l|} \hline
        \\
        t = a + b \\
        for (\ldots) \{ \\
        \ \ \ \ \ldots  \\
        \ \ \ \ u = t \\
        \ \ \ \ \ldots \\
        \} \\ \\ \hline
      \end{tabular}
    \end{center} \pause
  \item \alert{\psshadowbox{When should we NOT apply it?}}
  \end{itemize}
}

\frame{
  \frametitle{Code Movement}
  \begin{itemize}[<alert@+|+->]
  \item[]{Safety of code motion}
  \item[]{Profitability of code motion}
  \end{itemize}
}

\frame{
  \frametitle{Other Optimizations}
  \begin{itemize}[<+->]
  \item Dead code elimination
    \begin{itemize}
    \item Remove unreachable and/or unused code.
    \item \alert{Can we always do it?}
    \item \alert{Is there ever a need to introduce unused code?}
    \end{itemize}
  \item Strength Reduction
    \begin{itemize}
    \item Use of {\em low strength} operators in place of {\em high} strength ones.
      \begin{itemize}
      \item $i*i$ instead of $i**\ 2$, pow($i$, 2)
      \item $i<<1$ instead of $i*2$ 
      \end{itemize}
    \item Typically performed for integers only -- \alert{Why?}
    \end{itemize}
  \end{itemize}
}

\frame{
  \frametitle{Agenda}
  \begin{itemize}[<+->]
  \item Static analysis and compile-time optimizations
  \item For the next few lectures
  \item {\em Intraprocedural} Data Flow Analysis
    \begin{itemize}
    \item Classical Examples
    \item Components
    \end{itemize}
  \end{itemize}
}

\frame{
  \frametitle{Assumptions}
  \begin{itemize}[<+->]
  \item Intraprocedural: Restricted to a single function
  \item Input in 3-address format
  \item \alert{Unless otherwise specified}
  \end{itemize}
}

\frame{
  \frametitle{3-address Code Format}
  \begin{itemize}[<+->]
  \item Assignments
    \begin{itemize}
    \item[] x = y op z
    \item[] x = op y
    \item[] x = y
    \end{itemize}
  \item Jump/control transfer
    \begin{itemize}
    \item[] goto L
    \item[] if x relop y goto L
    \end{itemize}
  \item Statements can have label(s)
    \begin{itemize}
    \item[] L: \ldots
    \end{itemize}
  \item \alert{Arrays, Pointers and Functions to be added later when needed}
  \end{itemize}
}
\end{document}
