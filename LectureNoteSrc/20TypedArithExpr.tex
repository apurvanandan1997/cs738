\mode<presentation>
{
  %\usetheme{umbc4}
  %\setbeamercovered{dynamic}
}
\usepackage[english]{babel}
% or whatever

\usepackage[latin1]{inputenc}
% or whatever

\usepackage{times}
\usepackage[T1]{fontenc}
\usepackage{etex}
% Or whatever. Note that the encoding and the font should match. If T1
% does not look nice, try deleting the line with the fontenc.
%--------------------------------------------------------------------------------------------
%--- Other packages
\usepackage{pstricks}
\usepackage{pst-node}
\usepackage{pst-rel-points}
\usepackage{bcprules}
\usepackage{graphicx}
\newcommand{\change}[1]{{\red #1}}
%%%%%% Declarations Defined in custom-foils
%\setfootline{\insertshortauthor, \insertshortinstitute \hfill   \insertshorttitle \hfill \insertframenumber/\inserttotalframenumber} 
\newcommand{\mypart}[1]{%
  \frame[plain]{\mbox{}\vfill \psshadowbox{\huge #1} \vfill\mbox{}}
}
\author[karkare]{Amey Karkare \\ \url{karkare@cse.iitk.ac.in}}
\date[]{\scalebox{0.3}{\includegraphics{iitklogo.epsi}}}%
\institute[CSE, IITK]{\url{http://www.cse.iitk.ac.in/~karkare/cs738}\\Department of CSE, IIT Kanpur}
\title[CS738]{CS738: Advanced Compiler Optimizations}
\subtitle[]{}


\subtitle[Types]{\ \\{\LARGE Typed Arithmetic Expressions}}

\begin{document}

\frame{\titlepage}

\frame{
  \frametitle{Reference Book}
  Types and Programming Languages by Benjamin C. Pierce
}

\newcommand{\tm}{\ensuremath{t}}
\newcommand{\uif}{\mbox{\tt if}}
\newcommand{\uthen}{\mbox{\tt then}}
\newcommand{\uelse}{\mbox{\tt else}}
\newcommand{\usucc}{\mbox{\tt succ}}
\newcommand{\upred}{\mbox{\tt pred}}
\newcommand{\uiszero}{\mbox{\tt iszero}}
\newcommand{\utrue}{\mbox{\tt true}}
\newcommand{\ufalse}{\mbox{\tt false}}

\frame{
  \frametitle{Recap: Untyped Arithmetic Expression Language}
  \begin{tabular}{lp{.4\textwidth}l}
    \tm\ :=&& {\em -- terms} \\
    & \utrue & {\em -- constant true} \\
    & \ufalse & {\em -- constant false} \\
    & \uif\ \tm\ \uthen\ \tm\ \uelse\ \tm & {\em -- conditional} \\
    & 0 &  {\em -- constant zero} \\
    & \usucc\ \tm & {\em -- successor} \\
    & \upred\ \tm & {\em -- predecessor} \\
    & \uiszero\ \tm & {\em -- zero test}
  \end{tabular}
}

\newcommand{\val}{\ensuremath{v}}
\frame{
  \frametitle{Recap: The Set of Values}
  \begin{tabular}{lp{.4\textwidth}l}
    \val\ :=&& {\em -- values} \\
    & \utrue & {\em -- value true} \\
    & \ufalse & {\em -- value false} \\
    & 0 &  {\em -- value zero} \\
    & \usucc\ \val & {\em -- successor value}
  \end{tabular}
}

\newcommand{\ty}{\ensuremath{T}}
\newcommand{\bool}{\mbox{Bool}}
\newcommand{\nat}{\mbox{Nat}}
\frame{
  \frametitle{Let's add Types to the Language}
    \begin{tabular}{lp{.4\textwidth}l}
    \ty\ :=&& {\em -- Types} \\
      \pause & \bool & {\em -- Booleans} \\
      \pause & \nat & {\em -- Natural Numbers} \\
  \end{tabular}

}

\frame{
  \frametitle{The Typing Relation}
  \begin{itemize}[<+->]
  \item A set of rules assigning types to terms
  \item $\vdash \tm:\ty$ denotes ``term $\tm$ has type $\ty$''
  \end{itemize}\pause
  \infax{0 : \nat}\pause
  \infrule{\tm_1 : \nat}{\usucc\ \tm_1 : \nat}\pause
  \infrule{\tm_1 : \nat}{\upred\ \tm_1 : \nat}\pause
  \infrule{\tm_1 : \nat}{\uiszero\ \tm_1 : \bool}
}

\frame{
  \frametitle{The Typing Relation (contd\ldots)}
  \begin{itemize}
  \item A set of rules assigning types to terms
  \item $\vdash \tm:\ty$ denotes ``term $\tm$ has type $\ty$''
  \end{itemize}
  \infax{\utrue : \bool}\pause
  \infax{\ufalse : \bool}\pause
  \infrule{\tm_1 : \bool \andalso \tm_2: \ty \andalso \tm_3:\ty}{ \uif\ \tm_1\ \uthen\ \tm_2\ \uelse\ \tm_3 : \ty}
}

\frame{
  \frametitle{The Typing Relation: Definition}
  \begin{itemize}[<+->]
  \item The {\em typing relation} for arithmetic expressions is
    the smallest binary relation between terms and types
    satisfying all instances of the rules defined earlier.
  \item A term $\tm$ is {\em typable} (or {\em well typed}) if there is some $\ty$ such that $\tm:\ty$.
  \end{itemize}
}

\frame{
  \frametitle{Inversion of the Typing Relation}
  \begin{itemize}[<+->]
  \item If $\vdash 0 : R$, then $R = \nat$.
  \item If $\vdash \usucc\ \tm_1: R$, then $R = \nat$ and $\vdash \tm_1:\nat$.
  \item If $\vdash \upred\ \tm_1: R$, then $R = \nat$ and $\vdash \tm_1:\nat$.
  \item If $\vdash \uiszero\ \tm_1: R$, then $R = \bool$ and $\vdash \tm_1:\nat$.
  \item If $\vdash \utrue : R$, then $R = \bool$.
  \item If $\vdash \ufalse : R$, then $R = \bool$.
  \item If $\Gamma \vdash \uif\ \tm_1\ \uthen\ \tm_2\ \uelse\ \tm_3 : R$, then
    \begin{itemize}
    \item $\Gamma\vdash \tm_1 : \bool$
    \item $\Gamma\vdash \tm_2 : R$
    \item $\Gamma\vdash \tm_3 : R$
    \end{itemize}
  \end{itemize}
}

\frame{
  \frametitle{Uniqueness of Types}
  \begin{itemize}[<+->]
  \item Every term $\tm$ has at most one type.
  \item If $\tm$ is typeable, then its type is unique. 
  \item Moreover, there is just one derivation of this typing
    built from the inference rules.
  \end{itemize}
}

\frame{
  \frametitle{Safety = Preservation + Progress}
  \begin{itemize}[<+->]
  \item The type system is {\em safe} (also called {\em sound})
  \item Well-typed programs do not ``go wrong.''
    \begin{itemize}
    \item Do not reach a ``stuck state.''
    \end{itemize}
  \item {\bf Progress:} A well-typed term is not stuck.
    \begin{itemize}
    \item If $\vdash \tm:\ty$, then $\tm$ is either a value or
      there exists some $\tm'$ such that $\tm\rightarrow\tm'$.
    \end{itemize}
  \item {\bf Preservation:} If a well-typed term takes a step of
    evaluation, then the resulting term is also well-typed.
    \begin{itemize}
    \item If $\vdash \tm:\ty$ and $\tm\rightarrow\tm'$, then
      $\vdash \tm' : \ty$.
    \end{itemize}
 \end{itemize}
}
\end{document}

