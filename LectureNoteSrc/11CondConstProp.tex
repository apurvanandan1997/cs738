\mode<presentation>
{
  %\usetheme{umbc4}
  %\setbeamercovered{dynamic}
}
\usepackage[english]{babel}
% or whatever

\usepackage[latin1]{inputenc}
% or whatever

\usepackage{times}
\usepackage[T1]{fontenc}
% Or whatever. Note that the encoding and the font should match. If T1
% does not look nice, try deleting the line with the fontenc.
%--------------------------------------------------------------------------------------------
%--- Other packages
\usepackage{pstricks}
\usepackage{pst-node}
\usepackage{pst-rel-points}

\newcommand{\change}[1]{{\red #1}}
\newcommand{\myexample}{
  \begin{center}
    \psset{unit=1mm}
    \begin{pspicture}(0,0)(50,-78)
      %\psframe(0,0)(50,-78)
      \putnode{entry}{origin}{25}{5}{\psframebox{ENTRY}}
      
      \putnode{b1}{entry}{0}{-11}{\psframebox{a = 2}}
      \putnode{b2}{b1}{0}{-13}{\psframebox{b = 3}}
      \putnode{b3}{b2}{0}{-13}{\psframebox{a < b}}
      \putnode{b4}{b3}{-13}{-13}{\psframebox{c1 = 4}}
      \putnode{b5}{b3}{10}{-13}{\psframebox{c2 = 5}}
      \putnode{b6}{b3}{0}{-26}{\psframebox{c3 = $\phi$(c1, c2)}}

      \putnode{exit}{entry}{0}{-74}{\psframebox{EXIT}}

      \ncline{->}{entry}{b1}
      \ncline{->}{b1}{b2}
      \ncline{->}{b2}{b3}
      \ncline{->}{b3}{b4} \Bput[0.1]{true}
      \ncline{->}{b3}{b5} \Aput[0.1]{false}
      \ncline{->}{b4}{b6}
      \ncline{->}{b5}{b6}
      \ncline{->}{b6}{exit}

      \nccurve[linestyle=dotted,dotsep=0.3]{->}{b1}{b3}
      \nccurve[linestyle=dotted,dotsep=0.3,angle=180]{->}{b2}{b3}
      \nccurve[linestyle=dotted,dotsep=0.3,angle=180]{->}{b4}{b6}
      \nccurve[linestyle=dotted,dotsep=0.3]{->}{b5}{b6}
    \end{pspicture}
  \end{center}
}

%%%%%% Declarations Defined in custom-foils
%\setfootline{\insertshortauthor, \insertshortinstitute \hfill   \insertshorttitle \hfill \insertframenumber/\inserttotalframenumber} 
\newcommand{\mypart}[1]{%
  \frame[plain]{\mbox{}\vfill \psshadowbox{\huge #1} \vfill\mbox{}}
}
\author[karkare]{Amey Karkare \\ \url{karkare@cse.iitk.ac.in}}
\date[]{\scalebox{0.3}{\includegraphics{iitklogo.epsi}}}%
\institute[CSE, IITK]{\url{http://www.cse.iitk.ac.in/~karkare/cs738}\\Department of CSE, IIT Kanpur}
\title[CS738]{CS738: Advanced Compiler Optimizations}
\subtitle[]{}


\subtitle[]{\ \\{\LARGE Sparse Conditional Constant Propagation}}

\begin{document}

\frame{\titlepage}
\frame{
  \frametitle{Sparse Simple Constant Propagation (SSC)}
  \begin{itemize}[<+->]
  \item Improved analysis time over Simple Constant
    Propagation
  \item Finds all simple constant
    \begin{itemize}[<+->]
    \item Same class as Simple Constant Propagation
    \end{itemize}
  \end{itemize}
}

\frame{
  \frametitle{Motivating Example}
  {\scriptsize Dashed edges denote SSA \\def-use chains}
  \vspace*{-6mm}\myexample
}

\frame{
  \frametitle{Preparations for SSC Analysis}
  \begin{itemize}[<+->]
  \item Convert the program to SSA form
  \item One statement per basic block
  \item Add connections called {\em SSA edges}
    \begin{itemize}[<+->]
    \item Connect (unique) definition point of a variable to
      its use points 
    \item Same as {\em def-use} chains
    \end{itemize}
  \end{itemize}
}

\newcommand{\wl}{\mbox{\em WL}}
\frame{
  \frametitle{SSC Algorithm: Initialization}
  \begin{itemize}[<+->]
  \item Evaluate expressions involving constants only and
    assign the value ($c$) to variable on LHS
  \item If expression can not be evaluated at compile time,
    assign $\bot$
  \item Else (for expression contains variables) assign $\top$
  \item Initialize worklist \wl\  with SSA edges whose def is
    not $\top$ 
  \item Algorithm terminates when \wl\  is empty
  \end{itemize}
}

\frame{
  \frametitle{SSC Algorithm: Iterative Actions}
  \begin{itemize}[<+->]
  \item Take an SSA edge $E$ out of \wl
  \item Take meet of the value at def end and the use end of
    $E$ for the variable defined at def end
  \item If the meet value is different from use value,
    replace the use by the meet
  \item Recompute the def $d$ at the use end of $E$
  \item If the recomputed value is {\em lower} than the
    stored value, add all SSA edges originating at $d$
  \end{itemize}
}

\frame{
  \frametitle{Meet for $\phi$-function}
  $$v = \phi(v_1, v_2, \ldots, v_k) $$ \\
  $$\Rightarrow \mbox{ValueOf}(v) = v_1 \wedge v_2 \wedge \ldots \wedge v_n$$
}

\frame{
  \frametitle{SSC Algorithm: Complexity}
    \begin{itemize}[<+->]
    \item Height of CP lattice  = 2
    \item Each SSA edge is examined at most
      twice, for each lowering
    \item Theoritical size of SSA graph: $O(V \times E)$
    \item Practical size: linear in the program size
    \end{itemize}
}


\frame{
  \frametitle{SSC: Practice Example}
  \myexample
}

\frame{
  \frametitle{SSC: Practice Example}
  What if we change ``c1 = 4'' to ``c1 = 5''?
}

\frame{
  \frametitle{Sparse Condtional Constant Propagation (SCC)}
  \begin{itemize}[<+->]
  \item Constant Propagation with {\em unreachable code
    elimination}
  \item Ignore definitions that reach a use via a
    non-executable edge
  \end{itemize}
}

\frame{
  \frametitle{SCC Algorithm: Key Idea}
  $$v = \phi(v_1, v_2, \ldots, v_k)$$ \\
  $$\Rightarrow \mbox{ValueOf}(v) = \bigwedge_{i\in
    Executable Path}
  v_i$$\\

We ignore paths that are not ``yet'' marked executable
}

\newcommand{\fwl}{\mbox{\em FWL}}
\newcommand{\swl}{\mbox{\em SWL}}
\newcommand{\eflag}{{\it ExecutableFlag}}
\frame{
  \frametitle{SCC Algorithm: Preparations}
  \begin{itemize}[<+->]
  \item Two Worklists
    \begin{itemize}[<+->]
    \item Flow Worklist (\fwl)
      \begin{itemize}[<+->]
      \item Worklist of flow graph edges
      \end{itemize}
    \item SSA Worklist (\swl)
      \begin{itemize}[<+->]
      \item Worlist of SSA graph edges
      \end{itemize}
    \end{itemize}
  \item Execution Halts when {\bf both} worklists are empty
  \item Associate a flag, the {{\eflag}}, with
    every flow graph edge to control the evaluation of
    $\phi$-function in the destination node
  \end{itemize}
}

\frame{
  \frametitle{SCC Algorithm: Initialization}
  \begin{itemize}[<+->]
  \item Initialize \fwl\ to contain edges leaving ENTRY node
  \item Initialize \swl\ to empty
  \item Each {\eflag} is false initially
  \item Each value is $\top$ initially (Optimistic) 
  \end{itemize}
}

\frame{
  \frametitle{SCC Algorithm: Iterations}
  \begin{itemize}[<+->]
  \item Remove an item from either worklist
  \item process the item (described next)
  \end{itemize}
}

\frame{
  \frametitle{SCC Algorithm: Processing \fwl\ Item}
  \begin{itemize}[<+->]
  \item Item is flow graph edge
  \item If {\eflag} is true, do nothing
  \item Otherwise
    \begin{itemize}[<+->]
    \item Mark the {\eflag} as true
    \item {\bf Visit-$\phi$} for all $\phi$-functions in the
      destination
    \item If only one of the {\eflag}s of incoming flow graph
      edges for dest is true (dest visted for the first time),
      then {\bf VisitExpression} for all expressions in dest
    \item If the dest contains only one outgoing flow graph
      edge, add that edge to \fwl
    \end{itemize}
  \end{itemize}
}

\frame{
  \frametitle{SCC Algorithm: Processing \swl\ Item}
  \begin{itemize}[<+->]
  \item Item is SSA edge
  \item If dest is a $\phi$-function, {\bf
    Visit-$\phi$}
  \item If dest is an expression and any of {\eflag}s
    for the incoming flow graph edges of dest is true,
    perform {\bf VisitExpression}
  \end{itemize}
}

\frame{
  \frametitle{SCC Algorithm: Visit-$\phi$}
  $$v = \phi(v_1, v_2, \ldots, v_k)$$ \\
  \begin{itemize}[<+->]
  \item If $i^{th}$ incoming edge's \eflag\ is true,
    $val_i = \mbox{ValueOf}(v_i)$ else $val_i = \top$
  \item $\mbox{ValueOf}(v) = \bigwedge_i val_i$
  \end{itemize}
}

\frame{
  \frametitle{SCC Algorithm: VisitExpression}
  \begin{itemize}[<+->]
  \item Evaluate the expression using values of operands and
    rules for operators
  \item If the result is same as old, nothing to do
  \item Otherwise
    \begin{itemize}[<+->]
    \item If the expression is part of assignment, add all
      outgoing SSA edges to \swl
    \item if the expression controls a conditional branch,
      then
      \begin{itemize}[<+->]
      \item if the result is $\bot$, add all outgoing flow
        edges to \fwl
      \item if the value is constant $c$, only the
        corresponding flow graph edge is added to \fwl
      \item Value can not be $\top$ (why?)
      \end{itemize}
    \end{itemize}
  \end{itemize}
}

\frame{
  \frametitle{SCC Algorithm: Complexity}
  \begin{itemize}[<+->]
  \item Each SSA edge is examined twice
  \item Flow graph nodes are visited once for every incoming
    edge
  \item Complexity = O(\# of SSA edges + \# of flow graph edges) 
  \end{itemize}
}

\frame{
  \frametitle{SCC Algorithm: Correctness and Precision}
  \begin{itemize}[<+->]
  \item SCC is conservative
    \begin{itemize}[<+->]
    \item Never labels a variable value as a constant
    \end{itemize}
  \item SCC is at least as powerful as Conditional Constant
    Propagation (CC)
    \begin{itemize}[<+->]
    \item Finds all constants as CC does
    \end{itemize}
  \item PROOFs: In paper {\bf Constant propagation with
    conditional branches} by {\bf Mark N. Wegman, F. Kenneth
    Zadeck}, ACM TOPLAS 1991.

  \end{itemize}
}

\frame{
  \frametitle{Practice Example}
  \myexample
}

\end{document}
