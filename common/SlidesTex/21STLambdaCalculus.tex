\mode<presentation>
{
  \usetheme{umbc4}
%  \setbeamercovered{dynamic}
}
\usepackage[english]{babel}
% or whatever

\usepackage[latin1]{inputenc}
% or whatever

\usepackage{times}
\usepackage[T1]{fontenc}
\usepackage{etex}
% Or whatever. Note that the encoding and the font should match. If T1
% does not look nice, try deleting the line with the fontenc.
%--------------------------------------------------------------------------------------------
%--- Other packages
\usepackage{pstricks}
\usepackage{pst-node}
\usepackage{pst-rel-points}
\usepackage{bcprules}
\usepackage{graphicx}
\newcommand{\change}[1]{{\red #1}}
%%%%%% Declarations Defined in custom-foils
%\setfootline{\insertshortauthor, \insertshortinstitute \hfill   \insertshorttitle \hfill \insertframenumber/\inserttotalframenumber} 
\newcommand{\mypart}[1]{%
  \frame[plain]{\mbox{}\vfill \psshadowbox{\huge #1} \vfill\mbox{}}
}
\author[karkare]{Amey Karkare \\ \url{karkare@cse.iitk.ac.in}}
\date[]{\scalebox{0.3}{\includegraphics{iitklogo.epsi}}}%
\institute[CSE, IITK]{\url{http://www.cse.iitk.ac.in/~karkare/cs738}\\Department of CSE, IIT Kanpur}
\title[CS738]{CS738: Advanced Compiler Optimizations}
\subtitle[]{}


\subtitle[Types]{\ \\{\LARGE Simply Typed Lambda Calculus}}

\begin{document}

\frame{\titlepage}

\frame{
  \frametitle{Reference Book}
  Types and Programming Languages by Benjamin C. Pierce
}
\newcommand{\tm}{\ensuremath{\mathsf t}}
\newcommand{\ty}{\ensuremath{T}}
\newcommand{\bool}{\mbox{Bool}}
\newcommand{\nat}{\mbox{Nat}}
\frame{
  \frametitle{Simple Types over \bool}
  \begin{tabular}{rcl@{\qquad}r}
    $\ty$ & := &          &-- Types\\ \pause
    & & \bool &-- Boolean Type\\ \pause
    & & $\ty \rightarrow\ \ty $     &-- Function Type\\ \pause
  \end{tabular}

{\em type constructor} $\rightarrow$ is right-associative, i.e., $\ty_1\rightarrow \ty_2 \rightarrow \ty_3$ stands for  $\ty_1\rightarrow (\ty_2 \rightarrow \ty_3)$
}

\frame{
  \frametitle{Examples}
  For each of the type below, write a function (in your favourite programming language) that has the required type:
  \begin{itemize}[<+->]
  \item $\bool\rightarrow\bool$
  \item $\bool\rightarrow\bool\rightarrow\bool$
  \item $(\bool\rightarrow\bool)\rightarrow\bool$
  \item $(\bool\rightarrow\bool)\rightarrow\bool\rightarrow\bool$
  \item $(\bool\rightarrow\bool\rightarrow\bool)\rightarrow\bool$
  \item $(\bool\rightarrow\bool\rightarrow\bool)\rightarrow\bool$
  \item $((\bool\rightarrow\bool)\rightarrow\bool)\rightarrow\bool$
  \end{itemize}
}

\newcommand{\uif}{\mbox{\tt if}}
\newcommand{\uthen}{\mbox{\tt then}}
\newcommand{\uelse}{\mbox{\tt else}}
\newcommand{\usucc}{\mbox{\tt succ}}
\newcommand{\upred}{\mbox{\tt pred}}
\newcommand{\uiszero}{\mbox{\tt iszero}}
\newcommand{\utrue}{\mbox{\tt true}}
\newcommand{\ufalse}{\mbox{\tt false}}
\frame{
  \frametitle{The Abstract Syntax}
  Simply Typed $\lambda$-terms with conditions and Booleans
  \begin{tabular}{rcl@{\qquad}r}
    $\tm$ & := & $x$         & {\em -- Variable}\\ \pause
    & $\mid$ & $\lambda x:\ty.\ \tm$ &{\em -- Abstraction}\\ \pause
    & $\mid$& $\tm\ \tm $     &{\em -- Application} \\ \pause
    & $\mid$&\utrue & {\em -- constant true} \\ \pause
    & $\mid$&\ufalse & {\em -- constant false} \\ \pause
    & $\mid$&\uif\ \tm\ \uthen\ \tm\ \uelse\ \tm & {\em -- conditional} \\
  \end{tabular}
}

\newcommand{\val}{\ensuremath{v}}
\frame{
  \frametitle{Recap: The Set of Values}
  \begin{tabular}{rclp{.4\textwidth}l}
    \val\ &:=& & {\em -- values} \\
    && $\lambda x:\ty.\ \tm$ &{\em -- Abstraction Value}\\ \pause
    & $\mid$ & \utrue & {\em -- value true} \\\pause
    & $\mid$ & \ufalse & {\em -- value false} \\\
  \end{tabular}
}

\frame{
  \frametitle{Evaluation}
  \infrule[E-App1]{\tm_1 \rightarrow \tm'_1}{\tm_1\ \tm_2 \rightarrow \tm_1'\ \tm_2} \pause

  \infrule[E-App2]{\tm_2 \rightarrow \tm'_2}{\val\ \tm_2 \rightarrow \val\ \tm'_2} \pause

  \infax[E-AppAbs]{(\lambda x:\ty_1.\ \tm_1)\val_2 \rightarrow [x\mapsto \val_2]\tm_1}
}

\frame{
  \frametitle{The Typing Relation}
  \begin{itemize}[<+->]
  \item A {\em Typing Context} or {\em Type Environment}, $\Gamma$, is a sequence of variables with their types
  \item $\Gamma, x:\ty$ denotes extending $\Gamma$ with a new variable $x$ having type $\ty$
    \begin{itemize}
    \item The name $x$ is assumed to be distinct from any
      existing names in $\Gamma$
    \end{itemize}
  \end{itemize}
}

\frame{
  \frametitle{The Typing Relation}
  \infrule[T-Abs]{\Gamma,x:\ty_1 \vdash \tm_2:\ty_2}{\Gamma\vdash\lambda x:\ty_1.\ \tm_2 : \ty_1\rightarrow \ty_2} \pause

  \infrule[T-Var]{x:\ty \in \Gamma}{\Gamma\vdash x:\ty} \pause

  \infrule[T-App]{\Gamma\vdash \tm_1:\ty_1\rightarrow\ty_2\andalso\Gamma\vdash \tm_2:\ty_1}{\Gamma\vdash \tm_1\ \tm_2:\ty_2}
}

\frame{
  \frametitle{Inversion of the Typing Relation}
  \begin{itemize}[<+->]
  \item If $\Gamma \vdash x : R$, then $x:R \in \Gamma$.
  \item If $\Gamma \vdash \lambda x:\ty_1.\ \tm_2 : R$, then $R = \ty_1\rightarrow R_2$ for some $R_2$ with $\Gamma,x:\ty_1\vdash \tm_2:R_2$.
  \item If $\Gamma \vdash \tm_1\ \tm_2 : R$, then
    $\exists\ty_1\ s.t.\ \Gamma\vdash \tm_1:\ty_1\rightarrow R$ and
    $\Gamma\vdash\tm_2:\ty_1$.
  \item If $\Gamma \vdash \utrue : R$, then $R = \bool$.
  \item If $\Gamma \vdash \ufalse : R$, then $R = \bool$.
  \item If $\Gamma \vdash \uif\ \tm_1\ \uthen\ \tm_2\ \uelse\ \tm_3 : R$, then
    \begin{itemize}
    \item $\Gamma\vdash \tm_1 : \bool$
    \item $\Gamma\vdash \tm_2 : R$
    \item $\Gamma\vdash \tm_3 : R$
    \end{itemize}
  \end{itemize}
}

\frame{
  \frametitle{Exercises}
  \begin{itemize}[<+->]
  \item For each of the term $\tm$ below, find context $\Gamma$ and type $\ty$ such that \[\Gamma\vdash\tm : \ty\]
    \begin{itemize}
    \item $\tm$ is $\lambda x.\ x$
    \item $\tm$ is $(\ (x\ z)\ (y\ z)\ )$
    \item $\tm$ is $\lambda y.\ x$
    \item $\tm$ is $x\ x$
    \end{itemize}
  \end{itemize}
}


\frame{
  \frametitle{Uniqueness of Types}
  \begin{itemize}[<+->]
  \item In a given type context $\Gamma$, A term $\tm$, such that
    the free variables of $\tm$ are in $\Gamma$, has at most one
    type.
  \item If $\tm$ is typeable, then its type is unique. 
  \item Moreover, there is just one derivation of this typing
    built from the inference rules.
  \end{itemize}
}


\frame{
  \frametitle{Some Properties}
  \begin{itemize}[<+->] 
  \item {\bf Permutation:} If $\Gamma\vdash \tm:\ty$ and $\Delta$ is a permutation of $\Gamma$, then $\Delta\vdash \tm:\ty$.
    \begin{itemize}
    \item The derivation with $\Delta$ has the same depth as the
      derivation with $\Gamma$.
    \end{itemize}
  \item {\bf Weakening:} If $\Gamma\vdash \tm:\ty$ and $x \not\in \mbox{domain}(\Gamma)$, then $\Gamma, x:S \vdash \tm:\ty$.
    \begin{itemize}
    \item The derivation with $\Gamma, x:S$ has the same depth as the
      derivation with $\Gamma$.
    \end{itemize}
  \end{itemize}
}

\frame{
  \frametitle{Progress}
  \begin{itemize}[<+->]
  \item {\bf Progress:} A well-typed term is not stuck.
    \begin{itemize}
    \item If $\vdash \tm:\ty$, then $\tm$ is either a value or
      there exists some $\tm'$ such that $\tm\rightarrow\tm'$.
    \end{itemize}
  \end{itemize}
}
\frame{
  \frametitle{Preservation}
  \begin{itemize}[<+->]
  \item {\bf Preservation of Types under Substitution:} 
    If  $\Gamma, x:S \vdash \tm:\ty$ and $\Gamma \vdash s:S$, then
    $\Gamma \vdash [x\mapsto s]\tm : \ty$.
  \item {\bf Preservation:} If a well-typed term takes a step of
    evaluation, then the resulting term is also well-typed.
    \begin{itemize}
    \item If $\Gamma \vdash \tm:\ty$ and $\tm\rightarrow\tm'$, then
      $\Gamma \vdash \tm' : \ty$.
    \end{itemize}
  \end{itemize}
}
\end{document}
